%https://github.com/SublimeText/LaTeXTools/issues/1439
%!TEX output_directory=build

% Allows you to write your thesis both in English and Portuguese
% https://tex.stackexchange.com/questions/5076/is-it-possible-to-keep-my-translation-together-with-original-text
\newif\ifenglish\englishfalse
\newif\ifadvisor\advisorfalse

% Uncomment the line `\englishtrue` to set the document default language to English
% \englishtrue
\advisortrue

% https://tex.stackexchange.com/questions/131002/how-to-expand-ifthenelse-so-that-it-can-be-used-in-parshape
\newcommand{\lang}[2]{\ifenglish#1\else#2\fi}
\newcommand{\advisor}[2]{\ifadvisor#1\else#2\fi}

% https://tex.stackexchange.com/questions/385895/how-to-make-passoptionstopackage-add-the-option-as-the-last
% https://tex.stackexchange.com/questions/484400/changing-the-cleveref-package-language-conjunction-definition
% https://tex.stackexchange.com/questions/516058/why-isnt-my-biblatex-language-changing-when-passing-the-language-on-my-document
\ifenglish
    \PassOptionsToPackage{brazil,main=english,spanish,french}{babel}
\else
    \PassOptionsToPackage{main=brazil,english,spanish,french}{babel}
\fi

% Simple alias for English and Portuguese words
% https://tex.stackexchange.com/questions/513019/argument-of-bbltempd-has-an-extra
\newcommand{\brazilword}[1]{\protect\foreignlanguage{brazil}{#1}}
\newcommand{\englishword}[1]{\protect\foreignlanguage{english}{#1}}

% Allow you to write `Evandro's house` in latex as `Evandro\s house` instead of `Evandro\textquotesingle{}s house`
% https://tex.stackexchange.com/questions/31091/space-after-latex-commands
\newcommand{\s}[0]{\textquotesingle{}s\xspace}
\newcommand{\q}[0]{\textquotesingle{}\xspace}

% Uncomment the following line if you want to use other biblatex settings
% \PassOptionsToPackage{style=numeric,repeatfields=true,backend=biber,backref=true,citecounter=true}{biblatex}
\documentclass[
\lang{english}{brazilian,brazil}, % https://tex.stackexchange.com/questions/484400/changing-the-cleveref-package-language-conjunction-definition
12pt, % Padrão UFSC para versão final
a4paper, % Padrão UFSC para versão final
oneside, % Impressão nos dois lados da folha
chapter=TITLE, % Título de capítulos em caixa alta
section=TITLE, % Título de seções em caixa alta
]{setup/ufscthesisx}

% Utilize o arquivo aftertext/references.bib para incluir sua bibliografia.
% http://tug.ctan.org/tex-archive/macros/latex/contrib/cleveref/cleveref.pdf
\addbibresource{aftertext/references.bib}

% https://www.overleaf.com/learn/latex/Inserting_Images
\graphicspath{{pictures/}}

% ======================= DADOS DO AUTOR E DO TRABALHO =======================
% Preencha com seus dados
\autor{\brazilword{João Pedro Schmidt Cordeiro}}
\titulo{\lang{Generation of Pseudo-Random Numbers and Primality Tests}{Geração de Números Pseudo-Aleatórios e Testes de Primalidade}}

% Se houver subtítulo, descomente a linha abaixo
% \subtitulo{\lang{Subtitle}{Subtítulo}}

% Siglas para grau de formação Dr./Dra., Me./Ma, Bel. Bela. (inglês: PhD., MSc., Bs.)
\orientador[\lang{Supervisor}{Orientador(a)}]{\brazilword{Nome do Orientador(a)}, \lang{Phd.}{Dr.}}

% Se houver coorientador, descomente a linha abaixo
% \coorientador[\lang{Co-supervisor}{Coorientador(a)}]{\brazilword{Nome do coorientador(a)}, \lang{Phd.}{Dr.}}

% Preencher com o nome do Coordenador de TCCs/Teses do seu curso
\coordenador[\lang{Coordinator}{Coordenador(a)}]{\brazilword{Nome do Coordenador(a)}, \lang{Phd.}{Dr.}}

% Local da sua defesa
\local{\brazilword{Florianópolis, Santa Catarina} -- \lang{Brazil}{Brasil}}

% Ano da sua defesa
\ano{2025}
\biblioteca{\lang{University Library}{Biblioteca Universitária}}

% Sigla da sua instituição
\instituicaosigla{UFSC}
\instituicao{\brazilword{Universidade Federal de Santa Catarina}}

% Preencha com Tese, Dissertação, Monografia ou Trabalho de Conclusão de Curso, Bachelor's Thesis, etc
\tipotrabalho{\lang{Bachelor\s Thesis}{Trabalho de Conclusão de Curso}}

% Se houver Área de Concentração, descomente a linha abaixo
% \area{\lang{Information Security}{Segurança da Informação}}

% Preencha com Doutor, Bacharel ou Mestrando
\formacao{\lang
    {Bachelor of Science degree in Computer Science}
    {Bacharel em Ciências da Computação}%
}
\programa{\lang
    {Undergraduate Program in Computer Science}
    {Programa de Graduação em Ciências da Computação}%
}

% Preencha com Departamento de XXXXXX, Centro de XXXXXX
\centro{\lang
    {INE -- Department of Informatics and Statistics, CTC -- Technological Center}
    {INE -- Departamento de Informática e Estatística, CTC -- Centro Tecnológico}%
}

% Preencha com Campus XXXXXX     ou     Centro de XXXXXX
\campus{\brazilword{Campus Reitor João David Ferreira Lima}}

% Data da sua defesa
\data{\lang{30 of march of}{30 de março de} 2025}

% O preambulo deve conter tipo do trabalho, objetivo, nome da instituição e a área de concentração.
\preambulo{\lang%
    {%
        \imprimirtipotrabalho~submitted to the \imprimirprograma~of
        \imprimirinstituicao~for degree acquirement in \imprimirformacao.%
    }{%
        \imprimirtipotrabalho~submetido ao \imprimirprograma~da
        \imprimirinstituicao~para a obtenção do Grau de \imprimirformacao.%
    }%
}

% Allows you to use ~= instead of `\hyp{}`
% https://tex.stackexchange.com/questions/488008/how-to-create-an-alternative-to-shortcut-or-hyp
% https://tex.stackexchange.com/questions/405718/depending-on-babel-language-setting-i-get-biblatex-error-argument-of-language
% https://tex.stackexchange.com/questions/340661/argument-of-languageactivearg-has-an-extra-i-use-includegraphics-and-r
\useshorthands{~}\defineshorthand{~=}{\hyp{}}

% ======================= PALAVRAS-CHAVE =======================
% Adicione suas palavras-chave para o documento
\palavraschaveufsc{palavraschaveingles}   {Information Security}
\palavraschaveufsc{palavraschaveportugues}{Segurança da Informação}

\palavraschaveufsc{palavraschaveingles}   {Cryptography}
\palavraschaveufsc{palavraschaveportugues}{Criptografia}

\palavraschaveufsc{palavraschaveingles}   {Ethics}
\palavraschaveufsc{palavraschaveportugues}{Ética}

\hypersetup
{
    pdfsubject={Document Abstract},
    pdfcreator={LaTeX with abnTeX2 for UFSC},
    pdftitle={\imprimirtitulo},
    pdfauthor={\imprimirautor},
    pdfkeywords={\lang{\palavraschaveinglessemitem}{\palavraschaveportuguessemitem}},
}

% Altere o arquivo 'settings.tex' para incluir customizações de aparência da sua tese
%----------------------------------------------------------------------------------------
%   Thesis Tweaks and Utilities
%----------------------------------------------------------------------------------------

% Add the lipsum package for generating dummy text
\usepackage{lipsum}

% Add the float package to use [H] placement specifier for tables and figures
\usepackage{float}

\makeatletter


% Uncomment this if you are debugging pages' badness Underfull & Overflow
% https://tex.stackexchange.com/questions/115908/geometry-showframe-landscape
% https://tex.stackexchange.com/questions/387077/what-is-the-difference-between-usepackageshowframe-and-usepackageshowframe
% https://tex.stackexchange.com/questions/387257/how-to-do-the-memoir-headings-fix-but-not-have-my-text-going-over-the-page-botto
% https://tex.stackexchange.com/questions/14508/print-page-margins-of-a-document
% \usepackage[showframe,pass]{geometry}

% To use the font Times New Roman, instead of the default LaTeX font
% more up-to-date than '\usepackage{mathptmx}'
% \usepackage{newtxtext}
% \usepackage{newtxmath}

% https://tex.stackexchange.com/questions/182569/how-to-manually-set-where-a-word-is-split
\hyphenation{Ge-la-im}
\hyphenation{Cis-la-ghi}

% Add missing translations for Portuguese
% https://tex.stackexchange.com/questions/8564/what-is-the-right-way-to-redefine-macros-defined-by-babel
\@ifpackageloaded{babel}{\@ifpackagewith{babel}{brazil}{\addto\captionsbrazil{%
  \renewcommand{\mytextpreliminarylistname}{Breve Sumário}
}}{}}{}
\@ifundefined{advisor}{\newcommand{\advisor}[2]{#1}}{}

% Selects a sans serif font family
% \renewcommand{\sfdefault}{cmss}

% Selects a monospaced (“typewriter”) font family
% \renewcommand{\ttdefault}{cmtt}

% Spacing between lines and paragraphs
% https://tex.stackexchange.com/questions/70212/ifpackageloaded-question
\@ifclassloaded{memoir}
{
  % New custom chapter style VZ14, see other chapters styles in:
  % http://repositorios.cpai.unb.br/ctan/info/latex-samples/MemoirChapStyles/MemoirChapStyles.pdf
  \newcommand\thickhrulefill{\leavevmode \leaders \hrule height 1ex \hfill \kern \z@}
  \makechapterstyle{VZ14} { %
    % \thispagestyle{empty}
    \setlength\beforechapskip{50pt}
    \setlength\midchapskip{20pt}
    \setlength\afterchapskip{20pt}
    \renewcommand\chapternamenum{}
    \renewcommand\printchaptername{}
    \renewcommand\chapnamefont{\Huge\scshape}
    \renewcommand\printchapternum {%
      \chapnamefont\null\thickhrulefill\quad
      \@chapapp\space\thechapter\quad\thickhrulefill
    }
    \renewcommand\printchapternonum {%
      \par\thickhrulefill\par\vskip\midchapskip
      \hrule\vskip\midchapskip
    }
    \renewcommand\chaptitlefont{\huge\scshape\centering}
    \renewcommand\afterchapternum {%
      \par\nobreak\vskip\midchapskip\hrule\vskip\midchapskip
    }
    \renewcommand\afterchaptertitle {%
      \par\vskip\midchapskip\hrule\nobreak\vskip\afterchapskip
    }
  }

  % Apply the style `VZ14` just created
  % \chapterstyle{VZ14}

  % http://mirrors.ibiblio.org/CTAN/macros/latex/contrib/memoir/memman.pdf
  \setlength\beforechapskip{0pt}
  \setlength\midchapskip{15pt}
  \setlength\afterchapskip{15pt}

  % Memoir: Warnings "The material used in the headers is too large" w/ accented titles
  % https://tex.stackexchange.com/questions/387293/how-to-change-the-page-layout-with-memoir
  \setheadfoot{30.0pt}{\footskip}
  \checkandfixthelayout
}{}

% Controlling the spacing between one paragraph and another
% Default value for UFSC 0.0cm
\setlength{\parskip}{\advisor{0.0cm}{0.2cm}}

% Paragraph size is given by
% Default value for UFSC 1.5cm
% \setlength{\parindent}{1.3cm}

% https://tex.stackexchange.com/questions/148647/how-to-remove-space-before-enumerate
% https://tex.stackexchange.com/questions/433543/behaviour-of-enumitem-setlist
\advisor{}{
    \setlist*[enumerate]{label=\arabic*,}
    \setlist*[enumerateoptional]{label=\arabic*,}

    % https://tex.stackexchange.com/questions/24454/space-after-float-with-h
    % https://tex.stackexchange.com/questions/23313/how-can-i-reduce-padding-after-figure
    \AtBeginEnvironment{figure}{
      \setlength{\intextsep}{5pt} % Vertical space above & below [h] floats
      % \setlength{\textfloatsep}{10pt} % Vertical space below (above) [t] ([b]) floats
      % \setlength{\abovecaptionskip}{10pt}
      % \setlength{\belowcaptionskip}{5pt}
    }

    % Patch the `abntex2` citacao environment removing the extra space from its top
    % https://tex.stackexchange.com/questions/300340/topsep-itemsep-partopsep-and-parsep-what-does-each-of-them-mean-and-wha
    \xpatchcmd{\citacao}
    {\list{}}
    {\list{}{\topsep=0pt}}
    {}
    {\FAILEDPATCHINGCITACAO}
}


% Color settings across the document
\@ifpackageloaded{xcolor}
{
  % RGB colors in absolute values from 0 to 255 by using `RGB` tag
  \definecolor{darkblue}{RGB}{26,13,178}

  % Colors names definitions as RGB colors in percentage notation by using `rgb` tag
  \definecolor{mygreen}{rgb}{0,0.6,0}
  \definecolor{mygray}{rgb}{0.5,0.5,0.5}
  \definecolor{mymauve}{rgb}{0.58,0,0.82}
  \definecolor{figcolor}{rgb}{1,0.4,0}
  \definecolor{tabcolor}{rgb}{1,0.4,0}
  \definecolor{eqncolor}{rgb}{1,0.4,0}
  \definecolor{linkcolor}{rgb}{1,0.4,0}
  \definecolor{citecolor}{rgb}{1,0.4,0}
  \definecolor{seccolor}{rgb}{0,0,1}
  \definecolor{abscolor}{rgb}{0,0,1}
  \definecolor{titlecolor}{rgb}{0,0,1}
  \definecolor{biocolor}{rgb}{0,0,1}
  \definecolor{blue}{RGB}{41,5,195}

  % PDF Hyperlinks settings
  \@ifpackageloaded{hyperref}
  {
    \hypersetup
    {
      colorlinks=true,     % false: boxed links; true: colored links
      linkcolor=darkblue,  % color of internal links
      citecolor=darkblue, % color of links to bibliography
      filecolor=black,     % color of file links
      urlcolor=\advisor{black}{darkgreen},
      bookmarksdepth=4,
      pdfencoding=auto,%
      psdextra,
    }
  }
}{}


% Filtering and Mapping Bibliographies
% \DeclareFieldFormat{url}{Disponível~em:\addspace\url{#1}}

% https://tex.stackexchange.com/questions/517526/how-to-make-biblatex-url-links-generated-with-brackets-around-it-url-correctly
\DeclareFieldFormat{url}{\bibstring{urlfrom}\addcolon\space\textless\url{#1}\textgreater}
\DefineBibliographyStrings{brazil}{urlfrom = {Disponível em}}
\DefineBibliographyStrings{english}{urlfrom = {Available from}}

% https://tex.stackexchange.com/questions/391695/is-possible-to-remove-the-link-color-of-the-comma-on-the-citation-link
% \DeclareFieldFormat{citehyperref}{#1}

% % https://tex.stackexchange.com/questions/203764/reduce-font-size-of-bibliography-overfull-bibliography
% \newcommand{\bibliographyfontsize}{\fontsize{10.0pt}{10.5pt}\selectfont}
% \renewcommand*{\bibfont}{\bibliographyfontsize}

% Uncomment this to insert the abstract into your bibliography entries when the abstract is available
% https://tex.stackexchange.com/questions/398666/how-to-correctly-insert-and-justify-abstract
\ifadvisor
\else
  \DeclareFieldFormat{abstract}%
  {%
    \par\justifying
    \begin{adjustwidth}{1cm}{}
      \textbf{\bibsentence\bibstring{abstract}:} #1
    \end{adjustwidth}
  }
  \renewbibmacro*{finentry}%
  {%
    \iffieldundef{abstract}
    {\finentry}
    {\finentrypunct
      \printfield{abstract}%
      \renewcommand*{\finentrypunct}{}%
      \finentry
    }
  }

  % Backref package settings, pages with citations in bibliography
  \newcommand{\biblatexcitedntimes}{\autocap{c}ited \arabic{citecounter} times}
  \newcommand{\biblatexcitedonetime}{\autocap{c}ited one time}
  \newcommand{\biblatexcitednotimes}{\autocap{n}o citation in the text}

  \@ifpackageloaded{babel}{\@ifpackagewith{babel}{brazil}{\addto\captionsbrazil{%
    \renewcommand{\biblatexcitedntimes}{\autocap{c}itado \arabic{citecounter} vezes}
    \renewcommand{\biblatexcitedonetime}{\autocap{c}itado uma vez}
    \renewcommand{\biblatexcitednotimes}{\autocap{n}enhuma citação no texto}
  }}{}}{}
  \@ifpackageloaded{biblatex}
  {%
    % https://tex.stackexchange.com/questions/483707/how-to-detect-whether-the-option-citecounter-was-enabled-on-biblatex
    \ifx\blx@citecounter\relax
      \message{Is citecounter defined? NO!^^J}
    \else
      \message{Is citecounter defined? YES!^^J}
      \ifbacktracker
        \message{Is backtracker defined? YES!^^J}
        \renewbibmacro*{pageref}
        {%
          % https://tex.stackexchange.com/questions/516054/how-to-use-a-dot-to-separate-my-new-bibliography-entry
          \renewcommand*{\bibpagerefpunct}{\addperiod\space}%
          \iflistundef{pageref}%
          {\printtext{\biblatexcitednotimes}}
          {%
            \printtext
            {%
              \ifnumgreater{\value{citecounter}}{1}
                {\biblatexcitedntimes}
                {\biblatexcitedonetime}%
            }%
            \setunit{\addspace}%
            \ifnumgreater{\value{pageref}}{1}
              {\bibstring{backrefpages}\ppspace}
              {\bibstring{backrefpage}\ppspace}%
            \printlist[pageref][-\value{listtotal}]{pageref}%
          }%
        }

        \DefineBibliographyStrings{brazil}
        {
          backrefpage  = {na página},
          backrefpages = {nas páginas},
        }

        \DefineBibliographyStrings{english}
        {
          backrefpage  = {on page},
          backrefpages = {on pages},
        }
      \else
        \message{Is backtracker defined? NO!^^J}
      \fi
    \fi
  }{}
\fi


% https://tex.stackexchange.com/questions/516056/why-an-empty-or-not-biblatex-declaresourcemap-is-removing-my-bibliography-acces
% https://github.com/abntex/biblatex-abnt/pull/56/files
\DeclareStyleSourcemap{%% >>>2
  % This maps some fields used in abntex2cite to biblatex fields.
  \maps[datatype=bibtex]{%
    \map{%
      \step[fieldsource=conference-number,fieldtarget=number]%
      \step[fieldsource=conference-year,fieldtarget=eventdate]%
      \step[fieldsource=conference-location,fieldtarget=venue]%
      \step[fieldsource=conference-number,fieldtarget=number]%
      \step[fieldsource=org-short,fieldtarget=shortauthor]%
      \step[fieldsource=urlaccessdate,fieldtarget=urldate]%
      \step[fieldsource=year-presented,fieldtarget=eventyear]%
      \step[fieldsource=furtherresp,fieldtarget=titleaddon]%
      \step[typesource=journalpart,typetarget=supperiodical]%
    }%
    \map[overwrite=false]{%
      \step[fieldsource=reprinted-from, final]%
      \step[fieldset=related, origfieldval]%
    }%
    \map[overwrite=false]{%
      \step[fieldsource=reprinted-text, final]%
      \step[fieldset=relatedtype, fieldvalue={reprintfrom}]%
    }%
    \map{%
      \pertype{patent}% Use the organization as sourcekey for patents
      \step[fieldsource=organization, final]%
      \step[fieldset=sortkey, origfieldval]%
    }%
    \map[overwrite=false]{%
      \pertype{thesis}%
      \pertype{phdthesis}%
      \pertype{mastersthesis}%
      \pertype{monography}%
      \step[fieldset=bookpagination, fieldvalue={sheet}]%
    }%
    % remove fields that are always useless
    \map{
      % \step[fieldset=abstract, null]
      \step[fieldset=pagetotal, null]
    }
    % % remove URLs for types that are primarily printed
    % \map{
    %   \pernottype{software}
    %   \pernottype{online}
    %   \pernottype{report}
    %   \pernottype{techreport}
    %   \pernottype{standard}
    %   \pernottype{manual}
    %   \pernottype{misc}
    %   \step[fieldset=url, null]
    %   \step[fieldset=urldate, null]
    % }
    \map{
      \pertype{inproceedings}
      % remove mostly redundant conference information
      \step[fieldset=venue, null]
      \step[fieldset=eventdate, null]
      \step[fieldset=eventtitle, null]
      % do not show ISBN for proceedings
      \step[fieldset=isbn, null]
      % Citavi bug
      \step[fieldset=volume, null]
    }
  }%
}% <<<2


% https://tex.stackexchange.com/questions/14314/changing-the-font-of-the-numbers-in-the-toc-in-the-memoir-class
\renewcommand{\cftpartfont}{\ABNTEXpartfont\color{black}}
\renewcommand{\cftpartpagefont}{\ABNTEXpartfont\color{black}}

\renewcommand{\cftchapterfont}{\ABNTEXchapterfont\color{black}}
\renewcommand{\cftchapterpagefont}{\ABNTEXchapterfont\color{black}}

\renewcommand{\cftsectionfont}{\ABNTEXsectionfont\color{black}}
\renewcommand{\cftsectionpagefont}{\ABNTEXsectionfont\color{black}}

\renewcommand{\cftsubsectionfont}{\ABNTEXsubsectionfont\color{black}}
\renewcommand{\cftsubsectionpagefont}{\ABNTEXsubsectionfont\color{black}}

\renewcommand{\cftsubsubsectionfont}{\ABNTEXsubsubsectionfont\color{black}}
\renewcommand{\cftsubsubsectionpagefont}{\ABNTEXsubsubsectionfont\color{black}}

\renewcommand{\cftparagraphfont}{\ABNTEXsubsubsubsectionfont\color{black}}
\renewcommand{\cftparagraphpagefont}{\ABNTEXsubsubsubsectionfont\color{black}}

% Memoir has another mechanism for the job: \cftsetindents{‹kind›}{indent}{numwidth}. Here kind is
% chapter, section, or whatever; the indent specifies the 'margin' before the entry starts; and the
% width is of the box into which the number is typeset (so needs to be wide enough for the largest
% number, with the necessary spacing to separate it from what comes after it in the line.
% http://www.tex.ac.uk/FAQ-tocloftwrong.html
% https://tex.stackexchange.com/questions/264668/memoir-indentation-of-unnumbered-sections-in-table-of-contents
% https://tex.stackexchange.com/questions/394227/memoir-toc-indent-the-second-line-by-numberspace
%
% `\cftlastnumwidth` and these `\cftsetindents` are defined by the abntex2 class,
% obeying the `ABNTEXsumario-abnt-6027-2012`. \newlength{\cftlastnumwidth}
% \setlength{\cftlastnumwidth}{\cftsubsubsectionnumwidth}
% \addtolength{\cftlastnumwidth}{-1em}

% http://www.tex.ac.uk/FAQ-tocloftwrong.html
% Use \setlength\cftsectionnumwidth{4em} to override all these values at once
\ifadvisor
\else
  \makechapterstyle{fixedabntex2indentation}
  {%
    \renewcommand{\chapterheadstart}{}
    \setlength{\beforechapskip}{20pt}
    \setlength{\midchapskip}{20pt}
    \setlength{\afterchapskip}{15pt}

    \ifx \chapternamenumlength \undefined
      \newlength{\chapternamenumlength}
    \fi

    % tamanhos de fontes de chapter e part
    \ifthenelse{\equal{\ABNTEXisarticle}{true}}{%
      \setlength\beforechapskip{\baselineskip}%
      \renewcommand{\chaptitlefont}{\ABNTEXsectionfont\ABNTEXsectionfontsize}%
    }{%else
       \setlength{\beforechapskip}{0pt}%
       \renewcommand{\chaptitlefont}{\ABNTEXchapterfont\ABNTEXchapterfontsize}%
    }

    \renewcommand{\chapnumfont}{\chaptitlefont}
    \renewcommand{\parttitlefont}{\ABNTEXpartfont\ABNTEXpartfontsize}
    \renewcommand{\partnumfont}{\ABNTEXpartfont\ABNTEXpartfontsize}
    \renewcommand{\partnamefont}{\ABNTEXpartfont\ABNTEXpartfontsize}

    % tamanhos de fontes de section, subsection, subsubsection e subsubsubsection
    \setsecheadstyle{\ABNTEXsectionfont\ABNTEXsectionfontsize\ABNTEXsectionupperifneeded}
    \setsubsecheadstyle{\ABNTEXsubsectionfont\ABNTEXsubsectionfontsize\ABNTEXsubsectionupperifneeded}
    \setsubsubsecheadstyle{\ABNTEXsubsubsectionfont\ABNTEXsubsubsectionfontsize\ABNTEXsubsubsectionupperifneeded}
    \setsubsubsubsecheadstyle{\ABNTEXsubsubsubsectionfont\ABNTEXsubsubsubsectionfontsize\ABNTEXsubsubsubsectionupperifneeded}

    % Impressão do número do capítulo
    \renewcommand{\chapternamenum}{}

    % Impressão do nome do capítulo
    \renewcommand{\printchaptername}{%
       \chaptitlefont%
       \ifthenelse{\boolean{abntex@apendiceousecao}}{\appendixname}{}%
    }

    % Impressão do título do capítulo
    \def\printchaptertitle##1{%
      \chaptitlefont%
      \ifthenelse{\boolean{abntex@innonumchapter}}{\centering\ABNTEXchapterupperifneeded{##1}}{%
      \ifthenelse{\boolean{abntex@apendiceousecao}}{%
        \centering%
        \settowidth{\chapternamenumlength}{\printchaptername\printchapternum\afterchapternum}%
        \ABNTEXchapterupperifneeded{##1}%
      }{%
        \settowidth{\chapternamenumlength}{\printchaptername\printchapternum\afterchapternum}%
        \parbox[t]{\columnwidth-\chapternamenumlength}{\ABNTEXchapterupperifneeded{##1}}}%
      }%
    }

    % https://tex.stackexchange.com/questions/264668/memoir-indentation-of-unnumbered-sections-in-table-of-contents
    \renewcommand{\tocinnonumchapter}{%
      \addtocontents{toc}{\cftsetindents{chapter}{2.5em}{2em}}%
      \cftinserthook{toc}{A}}

    % Impressão do número do capítulo (no capítulo e não toc)
    \renewcommand{\printchapternum}{%
      \setboolean{abntex@innonumchapter}{false}%
      \chapnumfont%
      ~~\thechapter~%
      \ifthenelse{\boolean{abntex@apendiceousecao}}{%
        \tocinnonumchapter%
        ~\ABNTEXcaptiondelim~~%
      }{}%
    }

    \renewcommand{\ABNTEXcaptiondelim}{~\textendash~}
    \renewcommand{\afterchapternum}{}

    % Impressão do capítulo não numerado
    \renewcommand\printchapternonum{%
      \setboolean{abntex@innonumchapter}{true}%
    }
  }
  \chapterstyle{fixedabntex2indentation}

  \cftsetindents{part}          {0em} {3em}
  \cftsetindents{chapter}       {0em} {3em}
  \cftsetindents{section}       {0em} {4.3em}
  \cftsetindents{subsection}    {0em} {5.2em}
  \cftsetindents{subsubsection} {0em} {5.1em}
  \cftsetindents{paragraph}     {0em} {6.0em}
  \cftsetindents{subparagraph}  {0em} {7.0em}
\fi


\makeatother



\begin{document}
    % FIXME: Comment this after finishing the thesis, so you can start fixing the \flushbottom vs \raggedbottom
    % https://tex.stackexchange.com/questions/65355/flushbottom-vs-raggedbottom
    \raggedbottom

    % https://tex.stackexchange.com/questions/4705/double-space-between-sentences
    \frenchspacing

    % Uncomment this to put a ←← | ← (Go To Top/Go Back) on each section header
    \advisor{}{\addGoToSummary}

    % ELEMENTOS PRÉ-TEXTUAIS
    % Capa
    \pretextual
    % https://tex.stackexchange.com/questions/227711/blank-page-after-titlingpage
    \advisor{}{\AtBeginShipoutNext{\AtBeginShipoutNext{\AtBeginShipoutDiscard}}}
    \imprimircapa

    % ======================= CONTEÚDO DO DOCUMENTO =======================
    \textual
    
    % Incluindo os capítulos do trabalho
    A segurança da informação tradicionalmente tem sido abordada como um conjunto de desafios técnicos: como proteger sistemas contra invasores, como garantir a confidencialidade, integridade e disponibilidade de dados, e como implementar medidas de proteção eficazes. No entanto, à medida que sistemas computacionais se tornam mais integrados à sociedade e assumem papéis mais críticos na mediação de aspectos fundamentais da vida humana, torna-se evidente que a segurança da informação não é apenas um domínio técnico, mas também profundamente ético.

Esta análise busca explorar as dimensões éticas dessa segurança, examinando como princípios morais fundamentais devem guiar as práticas profissionais neste campo. O estudo se concentra especificamente em um sistema de clusters empresariais, composto por quatro clusters (três de desenvolvimento e um de produção), onde aplicações são executadas em containers orquestrados via Kubernetes. Como administrador deste sistema, enfrento diariamente dilemas éticos que vão além das questões técnicas, envolvendo o acesso a dados sensíveis de clientes, a gestão de recursos computacionais e a supervisão de equipes de desenvolvimento.

O objetivo é demonstrar que uma abordagem puramente técnica à segurança da computação é insuficiente; os profissionais da área precisam incorporar considerações éticas em todas as facetas de seu trabalho, desde o desenvolvimento de sistemas até a resposta a incidentes.

A premissa central deste documento é que a ética não é um componente opcional ou secundário da segurança da computação, mas constitui sua própria essência. Como argumenta \citeauthor{spinello2013cyberethics}, "a ética da segurança da informação não é meramente sobre seguir regras ou códigos de conduta, mas sobre cultivar uma sensibilidade moral que permita aos profissionais reconhecer e responder apropriadamente às dimensões éticas de seu trabalho" \cite{spinello2013cyberethics}.

Esta análise está estruturada em quatro seções principais. Primeiramente, examinaremos a responsabilidade social e profissional inerente ao trabalho em segurança da computação. Em seguida, analisaremos como os princípios éticos estabelecidos pela Association for Computing Machinery (ACM) se aplicam especificamente aos desafios de segurança. A terceira seção explorará os impactos éticos das falhas de segurança, destacando como vulnerabilidades e brechas afetam indivíduos e sociedades além das consequências técnicas imediatas. Finalmente, abordaremos as dimensões éticas do gerenciamento de riscos em segurança da computação, analisando como decisões sobre quais riscos aceitar, mitigar ou transferir refletem valores e prioridades éticas.

Através desta análise, espera-se cultivar uma compreensão mais profunda das responsabilidades éticas dos profissionais de segurança da computação e contribuir para o desenvolvimento de práticas que não apenas protejam sistemas e dados, mas também respeitem e promovam valores humanos fundamentais. 
    
    \chapter{Números Aleatórios}\label{chap:numeros-aleatorios}

\section{Xorshift}

\subsection{Descrição}

O algoritmo Xorshift é uma classe de geradores de números pseudoaleatórios (PRNGs) desenvolvida pelo matemático George Marsaglia e publicada em 2003 no Journal of Statistical Software \cite{marsaglia2003xorshift}. O Xorshift pertence à família dos registradores de deslocamento com retroalimentação linear (LFSRs) \cite{brent2004note}, mas foi projetado para permitir implementações particularmente eficientes em software.

O princípio fundamental do Xorshift é a geração de números pseudoaleatórios através de uma sequência de operações bit a bit simples: o XOR (ou exclusivo) de um número com uma versão deslocada de si mesmo. Este processo é repetido algumas vezes em cada ciclo de geração.

A fórmula básica do algoritmo pode ser expressa da seguinte forma \cite{marsaglia2003xorshift}:
\begin{enumerate}
    \item Inicia-se com um valor de estado inicial não-nulo $x$
    \item Aplica-se $x \oplus= x \ll a$ (desloca $x$ à esquerda por $a$ bits e faz XOR com o valor original)
    \item Aplica-se $x \oplus= x \gg b$ (desloca $x$ à direita por $b$ bits e faz XOR com o valor resultante)
    \item Aplica-se $x \oplus= x \ll c$ (desloca $x$ à esquerda por $c$ bits e faz XOR com o valor resultante)
    \item O resultado é o novo valor de $x$, que também se torna o estado para a próxima iteração
\end{enumerate}

Os parâmetros $a$, $b$ e $c$ devem ser escolhidos cuidadosamente para garantir que o algoritmo atinja seu período máximo $(2^n-1)$ \cite{marsaglia2003xorshift, brent2004note}, onde $n$ é o tamanho do estado em bits. Existem versões de 32 bits, 64 bits e 128 bits, cada uma com períodos diferentes.

Uma característica notável do Xorshift é sua extrema eficiência em implementações de software modernas, exigindo apenas algumas operações de bits por número gerado \cite{marsaglia2003xorshift}. Isso o torna um dos PRNGs mais rápidos disponíveis, requerindo código mínimo e pouco estado interno.

Apesar de sua simplicidade, as versões básicas do Xorshift não passam em todos os testes estatísticos de aleatoriedade sem refinamentos adicionais \cite{panneton2005xorshift, vigna2016experimental}. Para melhorar a qualidade estatística, variantes como Xorshift*, Xorshift+ e Xoroshiro foram desenvolvidas \cite{vigna2016experimental, blackman2018scrambled}, adicionando operações não-lineares (como multiplicação ou adição) ao resultado final.

Na prática, o algoritmo Xorshift oferece um excelente equilíbrio entre velocidade, simplicidade de implementação e qualidade estatística suficiente para muitas aplicações \cite{vigna2016experimental, blackman2018scrambled}, embora não seja adequado para uso criptográfico devido à sua natureza determinística e previsível quando o estado interno é conhecido \cite{panneton2005xorshift}.

\subsection{Implementação}

Para este estudo, implementamos quatro variantes do algoritmo Xorshift: Xorshift32, Xorshift64, Xorshift128 e Xorshift128Plus. Cada implementação utiliza diferentes tamanhos de estado interno e parâmetros de deslocamento recomendados na literatura.

A implementação completa do algoritmo Xorshift pode ser encontrada no Apêndice \ref{apx:xorshift-impl}, incluindo todas as variantes mencionadas. O código está escrito em Python e foi projetado para máxima clareza, mantendo a eficiência característica do algoritmo original.

As principais características de nossa implementação incluem:
\begin{itemize}
    \item Suporte para diferentes tamanhos de estado (32, 64 e 128 bits)
    \item Manipulação adequada de sementes iniciais
    \item Garantia de período completo através de parâmetros de deslocamento otimizados
    \item Métodos auxiliares para geração de números de ponto flutuante no intervalo [0,1)
\end{itemize}

Adicionalmente, criamos um código de teste que avalia o desempenho de cada variante na geração de números pseudoaleatórios de diferentes tamanhos (40 a 4096 bits). Este código também está disponível no Apêndice \ref{apx:xorshift-impl}.

\subsection{Experimento}

\begin{table}[H]
\centering
\caption{Desempenho do algoritmo Xorshift para geração de números pseudoaleatórios}
\label{tab:xorshift-performance}
\begin{tabular}{|l|c|c|}
\hline
\textbf{Algoritmo} & \textbf{Tamanho do Número} & \textbf{Tempo para gerar (ms)} \\
\hline
Xorshift32 & 40 bits & 0,0011 ms \\
Xorshift32 & 56 bits & 0,0009 ms \\
Xorshift32 & 80 bits & 0,0013 ms \\
Xorshift32 & 128 bits & 0,0016 ms \\
Xorshift32 & 168 bits & 0,0024 ms \\
Xorshift32 & 224 bits & 0,0027 ms \\
Xorshift32 & 256 bits & 0,0031 ms \\
Xorshift32 & 512 bits & 0,0059 ms \\
Xorshift32 & 1024 bits & 0,0124 ms \\
Xorshift32 & 2048 bits & 0,0250 ms \\
Xorshift32 & 4096 bits & 0,0547 ms \\
\hline
Xorshift64 & 40 bits & 0,0005 ms \\
Xorshift64 & 56 bits & 0,0005 ms \\
Xorshift64 & 80 bits & 0,0008 ms \\
Xorshift64 & 128 bits & 0,0008 ms \\
Xorshift64 & 168 bits & 0,0011 ms \\
Xorshift64 & 224 bits & 0,0015 ms \\
Xorshift64 & 256 bits & 0,0015 ms \\
Xorshift64 & 512 bits & 0,0029 ms \\
Xorshift64 & 1024 bits & 0,0058 ms \\
Xorshift64 & 2048 bits & 0,0115 ms \\
Xorshift64 & 4096 bits & 0,0261 ms \\
\hline
Xorshift128 & 40 bits & 0,0011 ms \\
Xorshift128 & 56 bits & 0,0012 ms \\
Xorshift128 & 80 bits & 0,0016 ms \\
Xorshift128 & 128 bits & 0,0019 ms \\
Xorshift128 & 168 bits & 0,0028 ms \\
Xorshift128 & 224 bits & 0,0033 ms \\
Xorshift128 & 256 bits & 0,0040 ms \\
Xorshift128 & 512 bits & 0,0074 ms \\
Xorshift128 & 1024 bits & 0,0156 ms \\
Xorshift128 & 2048 bits & 0,0304 ms \\
Xorshift128 & 4096 bits & 0,0678 ms \\
\hline
Xorshift128Plus & 40 bits & 0,0006 ms \\
Xorshift128Plus & 56 bits & 0,0006 ms \\
Xorshift128Plus & 80 bits & 0,0010 ms \\
Xorshift128Plus & 128 bits & 0,0011 ms \\
Xorshift128Plus & 168 bits & 0,0014 ms \\
Xorshift128Plus & 224 bits & 0,0018 ms \\
Xorshift128Plus & 256 bits & 0,0017 ms \\
Xorshift128Plus & 512 bits & 0,0035 ms \\
Xorshift128Plus & 1024 bits & 0,0062 ms \\
Xorshift128Plus & 2048 bits & 0,0130 ms \\
Xorshift128Plus & 4096 bits & 0,0290 ms \\
\hline
\end{tabular}
\end{table}

\textit{Cada medição representa a média de 1000 execuções.}

\section{Blum Blum Shub}

\subsection{Descrição}

O algoritmo Blum Blum Shub (BBS) é um gerador de números pseudoaleatórios criptograficamente seguro proposto em 1986 por Lenore Blum, Manuel Blum e Michael Shub \cite{blum1986simple}. Diferentemente de algoritmos como o Xorshift, o BBS foi projetado com foco primário em segurança criptográfica, sendo fundamentado em problemas matemáticos considerados computacionalmente difíceis.

A base matemática do BBS está na teoria dos resíduos quadráticos \cite{menezes1996handbook} e na dificuldade do problema de fatoração de inteiros. O algoritmo funciona da seguinte forma \cite{blum1986simple}:

\begin{enumerate}
    \item Escolhem-se dois números primos grandes $p$ e $q$, cada um congruente a 3 módulo 4 (ou seja, $p \equiv q \equiv 3 \pmod{4}$)
    \item Calcula-se $n = p \times q$, que é chamado de módulo de Blum
    \item Seleciona-se um valor inicial (semente) $s$, tal que $s$ seja co-primo com $n$ (ou seja, $\gcd(s,n) = 1$)
    \item Gera-se a sequência de estados internos através da recorrência: $x_1 = s^2 \bmod n$, $x_2 = x_1^2 \bmod n$, $x_3 = x_2^2 \bmod n$, ...
    \item Para cada estado $x_i$, extrai-se um ou mais bits de saída, tipicamente o bit menos significativo de cada $x_i$
\end{enumerate}

A segurança do BBS deriva da dificuldade computacional de calcular raízes quadradas modulares sem conhecer os fatores de $n$ \cite{sidorenko2005concrete}. Sem o conhecimento de $p$ e $q$, prever os próximos números na sequência é equivalente a resolver o problema de resíduo quadrático, que é considerado computacionalmente intratável para valores suficientemente grandes de $n$ \cite{koblitz2015riddle, blum1986simple}.

Principais características do BBS:
\begin{itemize}
    \item É criptograficamente seguro sob a suposição de que o problema da fatoração de inteiros é difícil \cite{sidorenko2005concrete}
    \item Possui uma prova matemática formal de segurança \cite{blum1986simple}
    \item É significativamente mais lento que PRNGs não criptográficos como Xorshift \cite{vassilev2016entropy}
    \item Requer operações aritméticas de precisão arbitrária para módulos grandes \cite{menezes1996handbook}
    \item Possui período potencialmente muito longo, dependendo da escolha de $p$, $q$ e $s$ \cite{menezes1996handbook}
\end{itemize}

O BBS é principalmente utilizado em aplicações criptográficas onde a previsibilidade representa um risco de segurança \cite{vassilev2016entropy}, embora sua lentidão relativa o torne menos adequado para aplicações que demandam alta performance ou grandes volumes de números aleatórios \cite{koblitz2015riddle}.

\subsection{Implementação}

Para este estudo, implementamos o algoritmo Blum Blum Shub conforme descrito na literatura original, com suporte para geração de números de diferentes tamanhos (em bits).

A implementação completa do algoritmo Blum Blum Shub pode ser encontrada no Apêndice \ref{apx:bbs-impl}. O código foi desenvolvido em Python, utilizando a biblioteca SymPy para verificação de primalidade e cálculo de MDC, garantindo a correta geração de primos congruentes a 3 módulo 4.

As principais características de nossa implementação incluem:
\begin{itemize}
    \item Geração automática de primos adequados (congruentes a 3 módulo 4)
    \item Validação apropriada de sementes
    \item Métodos para geração de bits individuais e sequências de bits
    \item Conversão de bits gerados para inteiros em intervalos específicos
\end{itemize}

Implementamos também um código de teste que avalia o desempenho do algoritmo na geração de números pseudoaleatórios de diferentes tamanhos (40 a 4096 bits), disponível no mesmo apêndice.

\subsection{Experimento}

\begin{table}[H]
\centering
\caption{Tempo médio para geração de números pseudoaleatórios usando o algoritmo Blum Blum Shub}
\label{tab:bbs-performance}
\begin{tabular}{|l|c|c|}
\hline
\textbf{Algoritmo} & \textbf{Tamanho do Número} & \textbf{Tempo para gerar (ms)} \\
\hline
Blum Blum Shub & 40 bits & 0,1566 ms \\
Blum Blum Shub & 56 bits & 0,2212 ms \\
Blum Blum Shub & 80 bits & 0,3244 ms \\
Blum Blum Shub & 128 bits & 0,5477 ms \\
Blum Blum Shub & 168 bits & 0,6986 ms \\
Blum Blum Shub & 224 bits & 0,9175 ms \\
Blum Blum Shub & 256 bits & 1,0078 ms \\
Blum Blum Shub & 512 bits & 2,0929 ms \\
Blum Blum Shub & 1024 bits & 4,1772 ms \\
Blum Blum Shub & 2048 bits & 8,6273 ms \\
Blum Blum Shub & 4096 bits & 17,3932 ms \\
\hline
\end{tabular}
\end{table}

\textit{Cada medição representa a média de 10 execuções.}

\section{Análise}

Foi interessante perceber a grande diferença, que era esperada, entre o resultado de geração de números pseudo-aleatórios ao utilizar os algoritmos de Xorshift e de Blum Blum Shub. Em média, o algoritmo Blum Blum Shub é aproximadamente 244 vezes mais lento que o Xorshift128 (o mais lento entre os Xorshifts). Entendemos que essa lentidão é decorrente da diferença de custo computacional das operações realizadas em cada um dos algoritmos, as operações do Xorshift são muito mais simples que as do Blum Blum Shub.

Também podemos avaliar essa diferença entre os algoritmos ao analisar a complexidade de cada um. O Xorshift possui uma complexidade temporal de $O(1)$, pois independentemente do tamanho do número gerado, ele realiza apenas um número fixo de operações bit a bit (XOR e deslocamentos). Essas operações são extremamente eficientes em hardware moderno.

Por outro lado, o Blum Blum Shub possui uma complexidade temporal de $O(k \log n)$, onde $k$ é o número de bits a serem gerados e $n$ é o módulo utilizado no algoritmo. Cada bit gerado requer uma operação de quadrado modular, que é computacionalmente cara, especialmente para números grandes. Além disso, o BBS depende de operações aritméticas de precisão arbitrária, que são intrinsecamente mais lentas que as operações bit a bit utilizadas pelo Xorshift.

Esta diferença de complexidade explica claramente a discrepância de desempenho observada nos experimentos, onde o BBS se torna progressivamente mais lento à medida que o tamanho dos números aumenta.

    
    \chapter{Números Primos}\label{chap:numeros-primos}

\section{Teste de Primalidade de Fermat}

\subsection{Descrição}

O Teste de Primalidade de Fermat é um método probabilístico para determinar se um número é provavelmente primo. Baseado no Pequeno Teorema de Fermat, este teste foi desenvolvido no século XVII e é um dos testes de primalidade mais antigos e fundamentais \cite{cohen1993course}.

O Pequeno Teorema de Fermat, que é a base matemática do teste, estabelece que:

Se $p$ é um número primo e $a$ é um inteiro não divisível por $p$, então:

\begin{equation}
a^{p-1} \equiv 1 \pmod{p}
\end{equation}

Ou seja, quando elevamos qualquer número $a$ à potência $p-1$ e dividimos o resultado por $p$, o resto será sempre 1, desde que $a$ e $p$ sejam coprimos (não compartilhem fatores comuns além de 1).

O teste funciona da seguinte forma \cite{cohen1993course}:

\begin{enumerate}
    \item Para verificar se um número $n$ é primo, escolhemos aleatoriamente um valor de $a$ tal que $1 < a < n-1$.
    \item Calculamos $a^{n-1} \bmod n$.
    \item Se o resultado não for 1, então $n$ é definitivamente composto (não primo).
    \item Se o resultado for 1, então $n$ é provavelmente primo, e $a$ é chamado de ``testemunha de Fermat'' para a provável primalidade de $n$.
\end{enumerate}

Para aumentar a confiança no resultado, o teste pode ser repetido com diferentes valores de $a$. Cada teste bem-sucedido aumenta a probabilidade de que $n$ seja realmente primo, embora nunca forneça uma prova definitiva de primalidade.

Uma característica importante deste teste é que ele nunca produz falsos negativos - se o teste indica que um número é composto, então ele é definitivamente composto. No entanto, existem números compostos especiais, chamados ``pseudoprimos de Fermat'' \cite{pomerance1980pseudoprimes, ribenboim1995new}, que podem enganar o teste para bases específicas. Ainda mais problemáticos são os números de Carmichael, que são números compostos que passam no teste de Fermat para todas as bases $a$ que são coprimas com $n$ \cite{alford1994infinitely, pinch1993carmichael}.

Apesar desta limitação, o Teste de Fermat é valorizado por sua simplicidade e eficiência computacional, especialmente para verificações iniciais rápidas \cite{ribenboim1995new}. Ele frequentemente é usado como um pré-teste antes de aplicar métodos mais robustos como Miller-Rabin ou outros testes determinísticos mais complexos.

O algoritmo do teste pode ser descrito da seguinte forma \cite{cohen1993course}:

\textbf{Entradas}: $n$ (número a ser testado), $k$ (número de iterações do teste)\\
\textbf{Saída}: ``composto'' ou ``provavelmente primo''

\begin{enumerate}
    \item Repita $k$ vezes:
    \begin{enumerate}
        \item Escolha um número aleatório $a$ no intervalo $[2, n-2]$
        \item Se $\gcd(a, n) \neq 1$, retorne ``composto''
        \item Se $a^{n-1} \bmod n \neq 1$, retorne ``composto''
    \end{enumerate}
    \item Se após $k$ iterações nenhum valor de $a$ demonstrou que $n$ é composto, retorne ``provavelmente primo''
\end{enumerate}

A complexidade computacional do teste é $O(k \log^2 n \log \log n)$, onde $k$ é o número de testes realizados e $n$ é o valor a ser verificado \cite{cormen2009introduction}. Esta eficiência torna o teste de Fermat particularmente útil para números muito grandes, apesar de suas limitações inerentes \cite{cormen2009introduction}.

\subsection{Implementação}

A implementação do Teste de Primalidade de Fermat realizada neste trabalho segue fielmente o algoritmo descrito na seção anterior. Utilizamos a linguagem Python devido à sua facilidade de manipulação de números grandes, essencial para testes de primalidade eficientes.

Nossa implementação inclui recursos para geração de números aleatórios através do algoritmo Xorshift (discutido no Capítulo \ref{chap:numeros-aleatorios}), exponenciação modular rápida e verificação de coprimalidade.

O código completo do Teste de Primalidade de Fermat será apresentado no Apêndice \ref{apx:fermat-impl}, juntamente com a implementação dos testes experimentais. A implementação foi projetada para máxima clareza e eficiência, permitindo a verificação de números de até 4096 bits.

\subsection{Experimento}

\subsubsection{Experimento 1 - Tentativas limitadas por tentativas}

\textbf{Data/Hora}: 2025-04-25 09:52:30

\paragraph{Resultados}

\begin{table}[H]
\centering
\caption{Resultados do Teste de Primalidade de Fermat (Experimento 1)}
\label{tab:fermat-exp1}
\begin{tabular}{|l|c|c|c|}
\hline
\textbf{Algoritmo} & \textbf{Tamanho do Número} & \textbf{Tentativas} & \textbf{Tempo para gerar} \\
\hline
Fermat Primality Test & 40 bits & 4 & 0,56 ms \\
Fermat Primality Test & 56 bits & 1 & 0,17 ms \\
Fermat Primality Test & 80 bits & 8 & 0,43 ms \\
Fermat Primality Test & 128 bits & 57 & 2,81 ms \\
Fermat Primality Test & 168 bits & 13 & 1,96 ms \\
Fermat Primality Test & 224 bits & 10 & 4,48 ms \\
Fermat Primality Test & 256 bits & 154 & 31,63 ms \\
Fermat Primality Test & 512 bits & 23 & 40,07 ms \\
Fermat Primality Test & 1024 bits & 374 & 2206,93 ms \\
Fermat Primality Test & 2048 bits & 100 & 3741,89 ms (falha) \\
Fermat Primality Test & 4096 bits & 100 & 29044,89 ms (falha) \\
\hline
\end{tabular}
\end{table}

\paragraph{Observações}
\begin{itemize}
    \item Taxa de sucesso: 9/11 (81,8\%)
    \item Tempo médio: 254,34 ms
    \item Tempo mínimo: 0,17 ms
    \item Tempo máximo: 2206,93 ms
    \item Tentativas médias: 71,56
    \item Tentativas mínimas: 1
    \item Tentativas máximas: 374
\end{itemize}

\paragraph{Relação entre tamanho e esforço}
\begin{itemize}
    \item À medida que o tamanho em bits aumenta, nota-se:
    \begin{itemize}
        \item Para 40 bits: 4 tentativas, 0,56 ms
        \item Para 1024 bits: 374 tentativas, 2206,93 ms
        \item Aumento de tentativas: 93,50x
        \item Aumento de tempo: 3940,95x
    \end{itemize}
\end{itemize}

\subsubsection{Experimento 2 - Tentativas limitadas por tempo}

\textbf{Data/Hora}: 2025-04-25 11:28:49

\paragraph{Resultados}

\begin{table}[H]
\centering
\caption{Resultados do Teste de Primalidade de Fermat (Experimento 2)}
\label{tab:fermat-exp2}
\begin{tabular}{|l|c|c|c|}
\hline
\textbf{Algoritmo} & \textbf{Tamanho do Número} & \textbf{Tentativas} & \textbf{Tempo para gerar} \\
\hline
Fermat Primality Test & 40 bits & 3 & 0,83 ms \\
Fermat Primality Test & 56 bits & 46 & 0,74 ms \\
Fermat Primality Test & 80 bits & 37 & 0,92 ms \\
Fermat Primality Test & 128 bits & 5 & 0,73 ms \\
Fermat Primality Test & 168 bits & 57 & 8,89 ms \\
Fermat Primality Test & 224 bits & 63 & 10,11 ms \\
Fermat Primality Test & 256 bits & 35 & 11,42 ms \\
Fermat Primality Test & 512 bits & 145 & 166,37 ms \\
Fermat Primality Test & 1024 bits & 128 & 792,42 ms \\
Fermat Primality Test & 2048 bits & 145 & 5343,09 ms \\
Fermat Primality Test & 4096 bits & 768 & 211568,14 ms \\
\hline
\end{tabular}
\end{table}

\paragraph{Observações}
\begin{itemize}
    \item Taxa de sucesso: 11/11 (100,0\%)
    \item Tempo médio: 19809,42 ms
    \item Tempo mínimo: 0,73 ms
    \item Tempo máximo: 211568,14 ms
    \item Tentativas médias: 130,18
    \item Tentativas mínimas: 3
    \item Tentativas máximas: 768
\end{itemize}

\paragraph{Relação entre tamanho e esforço}
\begin{itemize}
    \item À medida que o tamanho em bits aumenta, nota-se:
    \begin{itemize}
        \item Para 40 bits: 3 tentativas, 0,83 ms
        \item Para 4096 bits: 768 tentativas, 211568,14 ms
        \item Aumento de tentativas: 256,00x
        \item Aumento de tempo: 254901,37x
    \end{itemize}
\end{itemize}

\section{Teste de Primalidade de Miller-Rabin}

\subsection{Descrição}

O Teste de Primalidade de Miller-Rabin é um algoritmo probabilístico amplamente utilizado para determinar se um número é provavelmente primo. Desenvolvido por Gary Miller \cite{miller1976riemann} e Michael Rabin \cite{rabin1980probabilistic} na década de 1970, este teste é uma evolução do Teste de Fermat, oferecendo maior confiabilidade e menor probabilidade de falsos positivos.

A base matemática do teste Miller-Rabin está na teoria dos números e em propriedades específicas de números primos \cite{miller1976riemann, shoup2009computational}. Especificamente, o teste explora o fato de que, para números primos $p$, a equação $x^2 \equiv 1 \pmod{p}$ tem exatamente duas soluções: $x \equiv 1 \pmod{p}$ e $x \equiv -1 \pmod{p}$. Para números compostos, podem existir outras soluções, conhecidas como ``raízes quadradas não-triviais de 1''.

O algoritmo funciona da seguinte forma \cite{miller1976riemann}:

\begin{enumerate}
    \item Para um número ímpar $n > 3$ a ser testado, expressamos $n-1$ como $2^s \cdot d$, onde $d$ é ímpar.
    \item Escolhemos uma base $a$ aleatória tal que $2 \leq a \leq n-2$.
    \item Calculamos $x_0 = a^d \bmod n$.
    \item Se $x_0 = 1$ ou $x_0 = n-1$, então $a$ é uma testemunha da provável primalidade de $n$.
    \item Caso contrário, calculamos a sequência $x_1 = x_0^2 \bmod n$, $x_2 = x_1^2 \bmod n$, \ldots, $x_{s-1} = x_{s-2}^2 \bmod n$.
    \item Se em algum momento encontrarmos $x_i = n-1$, então $a$ é uma testemunha da provável primalidade.
    \item Se chegarmos ao fim da sequência sem encontrar $n-1$, então $a$ é uma testemunha da composição de $n$, provando definitivamente que $n$ é composto.
\end{enumerate}

O teste é repetido $k$ vezes com diferentes bases $a$ aleatórias. Se todas as bases testadas forem testemunhas da provável primalidade, declaramos $n$ como ``provavelmente primo''. A probabilidade de um número composto ser incorretamente classificado como primo é no máximo $4^{-k}$ \cite{rabin1980probabilistic, yan2009primality}, o que significa que cada iteração adicional reduz a probabilidade de erro por um fator de 4.

Uma característica fundamental do Miller-Rabin é que, diferentemente do Teste de Fermat, não existem números compostos que passem no teste para todas as bases possíveis. Para um número composto $n$, pelo menos 3/4 de todas as possíveis bases $2 \leq a \leq n-2$ revelarão que $n$ é composto \cite{crandall2005prime, yan2009primality}. Esta propriedade elimina o problema dos números de Carmichael que afeta o teste de Fermat.

Além disso, existem versões determinísticas do teste Miller-Rabin para números de até um certo tamanho, usando um conjunto específico e limitado de bases \cite{crandall2005prime, menezes1996handbook_miller}. Por exemplo, para números de 64 bits, testar apenas as bases $\{2, 3, 5, 7, 11, 13, 17, 19, 23, 29, 31, 37\}$ é suficiente para determinar com certeza absoluta se o número é primo.

A complexidade computacional do teste Miller-Rabin é $O(k \log^3 n)$, onde $k$ é o número de iterações e $n$ é o número sendo testado \cite{crandall2005prime}. Esta eficiência, combinada com sua forte garantia probabilística, torna o Miller-Rabin o teste de primalidade mais utilizado em aplicações criptográficas modernas, incluindo geração de chaves para RSA, DSA e outros sistemas criptográficos \cite{rabin1980probabilistic, menezes1996handbook_miller}.

O pseudocódigo do algoritmo pode ser descrito como \cite{shoup2009computational}:

\begin{verbatim}
Função Miller-Rabin(n, k):
    Se n = 2 ou n = 3, retorne "primo"
    Se n <= 1 ou n é par, retorne "composto"
    
    Escreva n-1 como 2^s * d, onde d é ímpar
    
    Para i de 1 até k:
        a ← número aleatório entre 2 e n-2
        x ← a^d mod n
        
        Se x = 1 ou x = n-1, continue com próxima iteração
        
        Para r de 1 até s-1:
            x ← x² mod n
            Se x = n-1, continue com próxima iteração de i
            Se x = 1, retorne "composto"
        
        Retorne "composto"
    
    Retorne "provavelmente primo"
\end{verbatim}

A principal vantagem do Miller-Rabin sobre outros testes de primalidade é seu equilíbrio entre eficiência computacional e confiabilidade estatística \cite{shoup2009computational, yan2009primality}, tornando-o ideal para sistemas criptográficos que dependem da geração rápida de números primos grandes \cite{menezes1996handbook_miller}.

\subsection{Implementação}

Nossa implementação do Teste de Primalidade de Miller-Rabin segue estritamente o algoritmo descrito na seção anterior. O código foi desenvolvido em Python, beneficiando-se de suas facilidades para manipulação de inteiros grandes.

A implementação inclui otimizações para a exponenciação modular rápida e decomposição eficiente de $n-1$ como $2^s \cdot d$. Para a geração de bases aleatórias, utilizamos o algoritmo Xorshift apresentado no Capítulo \ref{chap:numeros-aleatorios}, garantindo uma boa distribuição estatística das bases testadas.

O código completo do Teste de Primalidade de Miller-Rabin será apresentado no Apêndice \ref{apx:miller-rabin-impl}, junto com os scripts de teste utilizados nos experimentos. Implementamos também uma versão determinística para números de até 64 bits, utilizando um conjunto fixo de bases que garante resultado exato nessa faixa.

\subsection{Experimento}

\textbf{Data/Hora}: 2025-04-25 10:47:04

\paragraph{Resultados}

\begin{table}[H]
\centering
\caption{Resultados do Teste de Primalidade de Miller-Rabin}
\label{tab:miller-rabin}
\begin{tabular}{|l|c|c|c|}
\hline
\textbf{Algoritmo} & \textbf{Tamanho do Número} & \textbf{Tentativas} & \textbf{Tempo para gerar} \\
\hline
Miller-Rabin & 40 bits & 1 & 0,03 ms \\
Miller-Rabin & 56 bits & 70 & 0,56 ms \\
Miller-Rabin & 80 bits & 4 & 0,56 ms \\
Miller-Rabin & 128 bits & 15 & 1,67 ms \\
Miller-Rabin & 168 bits & 129 & 8,60 ms \\
Miller-Rabin & 224 bits & 150 & 19,00 ms \\
Miller-Rabin & 256 bits & 46 & 11,76 ms \\
Miller-Rabin & 512 bits & 128 & 158,37 ms \\
Miller-Rabin & 1024 bits & 844 & 4530,48 ms \\
Miller-Rabin & 2048 bits & 297 & 11650,23 ms \\
Miller-Rabin & 4096 bits & 2356 & 633614,98 ms \\
\hline
\end{tabular}
\end{table}

\paragraph{Observações}
\begin{itemize}
    \item Taxa de sucesso: 11/11 (100,0\%)
    \item Tempo médio: 59090,57 ms
    \item Tempo mínimo: 0,03 ms
    \item Tempo máximo: 633614,98 ms
    \item Tentativas médias: 367,27
    \item Tentativas mínimas: 1
    \item Tentativas máximas: 2356
\end{itemize}

\paragraph{Relação entre tamanho e esforço}
\begin{itemize}
    \item À medida que o tamanho em bits aumenta, nota-se:
    \begin{itemize}
        \item Para 40 bits: 1 tentativa, 0,03 ms
        \item Para 4096 bits: 2356 tentativas, 633614,98 ms
        \item Aumento de tentativas: 2356,00x
        \item Aumento de tempo: 19981758,38x
    \end{itemize}
\end{itemize}

\section{Análise}

Ao comparar o Teste de Primalidade de Fermat e o Teste de Miller-Rabin, podemos observar características importantes em termos de eficiência, precisão e aplicabilidade prática.

Foi decidido implementar o teste de Fermat como complemento ao Miller-Rabin devido à relação evolutiva entre eles. O teste de Fermat constitui a base histórica e conceitual para o teste de Miller-Rabin, permitindo uma análise comparativa que evidencia por que o segundo tornou-se o padrão em aplicações criptográficas modernas.

A diferença fundamental entre os dois algoritmos está na forma como lidam com casos especiais. O teste de Fermat sofre de uma limitação crítica: os números de Carmichael (como 561, 1105, 1729) \cite{alford1994infinitely, pinch1993carmichael}, que são números compostos que passam no teste para todas as bases coprimas, resultando em falsos positivos. Este problema impossibilita seu uso isolado em aplicações de segurança.

Miller-Rabin, por outro lado, refina o teste ao adicionar verificações para raízes quadradas não-triviais de 1, eliminando o problema dos números de Carmichael \cite{yan2009primality}. Para qualquer número composto $n$, pelo menos 3/4 de todas as possíveis bases revelarão sua composição \cite{crandall2005prime}, garantindo uma probabilidade de erro que decresce exponencialmente com o número de iterações ($4^{-k}$) \cite{rabin1980probabilistic}.

Analisando os experimentos realizados, observamos padrões importantes. Ambos os algoritmos apresentam aumento significativo no tempo de execução conforme o tamanho do número cresce. Para números de 4096 bits, o Miller-Rabin apresentou um aumento de aproximadamente 20 milhões de vezes no tempo em relação a números de 40 bits, enquanto o Fermat apresentou um aumento de cerca de 255 mil vezes.

Em termos de complexidade, o teste de Fermat tem complexidade $O(k \times \log^2 n \log \log n)$ \cite{cormen2009introduction}, onde $k$ é o número de iterações e $n$ o número testado. Já o Miller-Rabin apresenta complexidade $O(k \times \log^3 n)$ \cite{crandall2005prime}. A complexidade adicional do Miller-Rabin justifica-se pela verificação mais rigorosa que realiza, executando múltiplas verificações de quadrados modulares ao longo do processo.

Quanto ao número de tentativas, o Miller-Rabin geralmente necessitou mais tentativas para encontrar números primos (média de 367,27 tentativas contra 130,18 do Fermat no segundo experimento). Para 4096 bits, Miller-Rabin necessitou 2356 tentativas contra 768 do Fermat.

Um aspecto importante observado em ambos os algoritmos é a variabilidade do tempo de execução mesmo para números do mesmo tamanho. Isso ocorre porque o tempo depende não apenas do tamanho do número, mas também de suas propriedades matemáticas específicas. A natureza probabilística dos testes significa que alguns números são confirmados como primos mais rapidamente que outros. Além disso, a aleatoriedade na escolha das bases testemunhas influencia significativamente o desempenho. Esta variabilidade reforça a necessidade de múltiplas execuções para obter resultados estatisticamente significativos, como foi feito nos experimentos.

Embora o teste de Fermat seja computacionalmente mais eficiente para números muito grandes (como demonstrado no tempo médio de 19809,42 ms contra 59090,57 ms do Miller-Rabin), sua suscetibilidade a falsos positivos o torna inadequado para aplicações criptográficas críticas, onde a certeza da primalidade é essencial.

Miller-Rabin, apesar do maior custo computacional, oferece garantias probabilísticas muito mais robustas \cite{menezes1996handbook_miller}, tornando-o preferível para geração de chaves em criptografia assimétrica (RSA, DSA), sistemas de assinatura digital e protocolos que dependem da dificuldade da fatoração de números compostos \cite{rabin1980probabilistic}.

Por fim, observamos que ambos os algoritmos conseguiram gerar com sucesso números primos de até 4096 bits, tamanho suficiente para aplicações criptográficas modernas, embora com tempos significativamente diferentes.

Esta análise comparativa demonstra que, apesar da eficiência computacional do teste de Fermat, o teste de Miller-Rabin representa um avanço fundamental na verificação de primalidade \cite{menezes1996handbook_miller, yan2009primality}, equilibrando de forma mais adequada eficiência e confiabilidade para aplicações de segurança computacional \cite{shoup2009computational}.


    
    \chapter{Conclusão}

Este trabalho propôs-se a explorar, implementar e analisar experimentalmente algoritmos para geração de números pseudoaleatórios (PRNGs) e testes de primalidade, buscando compreender os compromissos entre desempenho, segurança e confiabilidade em contextos criptográficos. A seleção específica dos algoritmos Xorshift e Blum Blum Shub para geração de números aleatórios, e dos testes de Fermat e Miller-Rabin para verificação de primalidade, permitiu examinar tanto abordagens otimizadas para eficiência quanto aquelas projetadas com foco principal em segurança.

Os resultados experimentais demonstraram contrastes significativos entre os algoritmos estudados. Na geração de números pseudoaleatórios, observamos que o Xorshift supera drasticamente o Blum Blum Shub em termos de desempenho, sendo aproximadamente 244 vezes mais rápido em média. Esta discrepância é explicada pela diferença fundamental em suas complexidades computacionais: o Xorshift opera com complexidade O(1), realizando apenas operações bit a bit independentemente do tamanho do número gerado, enquanto o BBS apresenta complexidade O(k log n), exigindo custosas operações de quadrado modular para cada bit gerado. No entanto, esta vantagem de desempenho do Xorshift vem com o custo de menor segurança criptográfica, já que o algoritmo é determinístico e previsível quando seu estado interno é conhecido. Em contrapartida, o BBS oferece segurança criptográfica formalmente provada, baseada na dificuldade do problema de fatoração de inteiros, tornando-o adequado para aplicações onde a imprevisibilidade é crítica, apesar de seu desempenho inferior.

De forma similar, a análise dos testes de primalidade revelou um padrão de compromisso entre eficiência e confiabilidade. O teste de Fermat demonstrou ser computacionalmente mais eficiente para números muito grandes, apresentando tempo médio de 19.809,42 ms contra 59.090,57 ms do Miller-Rabin nos experimentos realizados. Contudo, sua suscetibilidade a falsos positivos, particularmente com os números de Carmichael, compromete sua aplicabilidade em sistemas criptográficos onde a certeza da primalidade é essencial. Por outro lado, o teste de Miller-Rabin, apesar de exigir maior esforço computacional, oferece garantias probabilísticas robustas, com probabilidade de erro máxima de $4^{-k}$, onde k é o número de iterações. Esta característica, aliada à ausência de vulnerabilidades a casos especiais como os números de Carmichael, torna o Miller-Rabin o padrão preferido para aplicações criptográficas modernas.

Os experimentos também evidenciaram como o aumento do tamanho dos números afeta drasticamente o desempenho dos algoritmos. Para o Miller-Rabin, observamos um aumento de aproximadamente 20 milhões de vezes no tempo de processamento ao passar de números de 40 bits para 4096 bits, enquanto o teste de Fermat apresentou um aumento de cerca de 255 mil vezes. Esta escalabilidade é particularmente relevante no contexto da criptografia moderna, onde números primos de 2048 a 4096 bits são frequentemente necessários.

Em síntese, este trabalho demonstra empiricamente como a escolha de algoritmos para geração de números pseudoaleatórios e verificação de primalidade envolve necessariamente um equilíbrio entre eficiência computacional e robustez criptográfica. Para aplicações onde o desempenho é prioritário e a segurança não é crítica, algoritmos como o Xorshift e o teste de Fermat podem ser adequados. No entanto, para sistemas criptográficos que exigem garantias de segurança, o custo computacional adicional do Blum Blum Shub e do teste de Miller-Rabin representa um investimento necessário.

Como direcionamento para trabalhos futuros, seria valioso explorar otimizações específicas para o Blum Blum Shub que possam reduzir sua desvantagem de desempenho sem comprometer sua segurança, bem como investigar implementações paralelas do teste de Miller-Rabin que possam acelerar a verificação de primalidade para números muito grandes. Adicionalmente, a análise de outros algoritmos de geração de números pseudoaleatórios criptograficamente seguros, como ChaCha20 ou HMAC-DRBG, poderia complementar este estudo comparativo, oferecendo perspectivas adicionais sobre o espectro de opções disponíveis para sistemas de segurança computacional.

    % ELEMENTOS PÓS-TEXTUAIS
    \postextual
    
    % Apêndices
    % \appendix

\chapter{Apêndices}

Os apêndices deste trabalho contêm as implementações detalhadas dos algoritmos discutidos nos capítulos anteriores. Eles estão organizados da seguinte forma:

\begin{itemize}
    \item \textbf{Apêndice A: Implementações dos Geradores de Números Pseudoaleatórios}
    \begin{itemize}
        \item Seção \ref{apx:xorshift-impl}: Implementação do algoritmo Xorshift e suas variantes (32 bits, 64 bits, 128 bits e 128+ bits)
        \item Seção \ref{apx:bbs-impl}: Implementação do algoritmo Blum Blum Shub
    \end{itemize}
    
    \item \textbf{Apêndice B: Implementações dos Testes de Primalidade}
    \begin{itemize}
        \item Seção \ref{apx:fermat-impl}: Implementação do Teste de Primalidade de Fermat
        \item Seção \ref{apx:miller-rabin-impl}: Implementação do Teste de Primalidade de Miller-Rabin
    \end{itemize}
\end{itemize}

Cada seção dos apêndices contém tanto a implementação do algoritmo em si quanto o código utilizado para os experimentos, permitindo a reprodução de todos os resultados apresentados nos capítulos principais. Os códigos estão documentados com comentários que explicam cada etapa do processo.

As implementações foram desenvolvidas em Python, escolhido pela sua facilidade de manipulação de números grandes e clareza sintática, características essenciais para algoritmos criptográficos e testes de primalidade. Todos os códigos foram projetados visando o equilíbrio entre legibilidade e eficiência, com otimizações específicas para operações críticas como exponenciação modular.

Para executar os experimentos, basta utilizar os arquivos de teste correspondentes a cada implementação, que já estão configurados com os parâmetros necessários para reproduzir os resultados apresentados neste trabalho.

\section{Xorshift}\label{apx:xorshift-impl}

\subsection{Implementação do Algoritmo}

\begin{verbatim}
"""
Implementação do algoritmo Xorshift PRNG
Baseado no trabalho original de George Marsaglia (2003)

Inclui implementações para versões de 32 bits, 64 bits e 128 bits.
"""

class Xorshift32:
    """
    Implementação do Xorshift de 32 bits com período de 2^32-1.
    """
    def __init__(self, seed=123456789):
        """
        Inicializa o gerador com uma semente não-nula.
        
        Args:
            seed (int): Valor de semente. Deve ser não-zero.
        """
        if seed == 0:
            seed = 123456789  # Evita estado zero
        self.state = seed & 0xFFFFFFFF  # Garante valor de 32 bits
    
    def next(self):
        """
        Gera o próximo número pseudoaleatório.
        
        Returns:
            int: Um número pseudoaleatório de 32 bits.
        """
        # Copia o estado atual para a variável x
        x = self.state
        # Aplica o primeiro deslocamento: desloca x 13 bits à esquerda, 
        # faz XOR com o valor original e mantém apenas os 32 bits inferiores
        x ^= (x << 13) & 0xFFFFFFFF
        # Aplica o segundo deslocamento: desloca x 17 bits à direita,
        # faz XOR com o resultado anterior e mantém apenas os 32 bits inferiores
        x ^= (x >> 17) & 0xFFFFFFFF
        # Aplica o terceiro deslocamento: desloca x 5 bits à esquerda,
        # faz XOR com o resultado anterior e mantém apenas os 32 bits inferiores
        x ^= (x << 5) & 0xFFFFFFFF
        # Atualiza o estado interno do gerador com o novo valor
        # garantindo que apenas os 32 bits inferiores sejam mantidos
        self.state = x & 0xFFFFFFFF
        
        # Retorna o novo estado como o número pseudoaleatório gerado
        return self.state
    
    def random(self):
        """
        Retorna um número de ponto flutuante no intervalo [0.0, 1.0).
        
        Returns:
            float: Um número entre 0.0 e 1.0.
        """
        return self.next() / 0x100000000


class Xorshift64:
    """
    Implementação do Xorshift de 64 bits com período de 2^64-1.
    """
    def __init__(self, seed=88172645463325252):
        """
        Inicializa o gerador com uma semente não-nula.
        
        Args:
            seed (int): Valor de semente. Deve ser não-zero.
        """
        if seed == 0:
            seed = 88172645463325252  # Evita estado zero
        self.state = seed & 0xFFFFFFFFFFFFFFFF  # Garante valor de 64 bits
    
    def next(self):
        """
        Gera o próximo número pseudoaleatório.
        
        Returns:
            int: Um número pseudoaleatório de 64 bits.
        """
        # Copia o estado atual para a variável x
        x = self.state
        # Aplica XOR entre x e (x deslocado 13 bits à esquerda), mantendo apenas 64 bits
        x ^= (x << 13) & 0xFFFFFFFFFFFFFFFF
        # Aplica XOR entre x e (x deslocado 7 bits à direita), mantendo apenas 64 bits
        x ^= (x >> 7) & 0xFFFFFFFFFFFFFFFF
        # Aplica XOR entre x e (x deslocado 17 bits à esquerda), mantendo apenas 64 bits
        x ^= (x << 17) & 0xFFFFFFFFFFFFFFFF
        # Atualiza o estado interno com o novo valor, garantindo que tenha apenas 64 bits
        self.state = x & 0xFFFFFFFFFFFFFFFF
        # Retorna o novo estado como o número pseudoaleatório gerado
        return self.state
    
    def random(self):
        """
        Retorna um número de ponto flutuante no intervalo [0.0, 1.0).
        
        Returns:
            float: Um número entre 0.0 e 1.0.
        """
        return self.next() / 0x10000000000000000


class Xorshift128:
    """
    Implementação do Xorshift de 128 bits com período de 2^128-1.
    """
    def __init__(self, seed=None):
        """
        Inicializa o gerador com uma semente ou valores padrão.
        
        Args:
            seed (list, optional): Lista de 4 valores inteiros para inicializar o estado.
        """
        if seed is None:
            # Valores iniciais não-nulos por padrão
            self.state = [123456789, 362436069, 521288629, 88675123]
        else:
            # Verifica se todos os valores são zero
            if all(s == 0 for s in seed):
                self.state = [123456789, 362436069, 521288629, 88675123]
            else:
                self.state = [s & 0xFFFFFFFF for s in seed[:4]]
    
    def next(self):
        """
        Gera o próximo número pseudoaleatório.
        
        Returns:
            int: Um número pseudoaleatório de 32 bits.
        """
        # Armazena o último valor do estado (índice 3) em t
        t = self.state[3]
        # Armazena o primeiro valor do estado (índice 0) em s
        s = self.state[0]
        
        # Desloca os valores do estado: o valor do índice 2 vai para o índice 3
        self.state[3] = self.state[2]
        # Desloca os valores do estado: o valor do índice 1 vai para o índice 2
        self.state[2] = self.state[1]
        # Desloca os valores do estado: o valor do índice 0 (s) vai para o índice 1
        self.state[1] = s
        
        # Aplica a primeira transformação XOR: desloca t 11 bits à esquerda e faz XOR com t
        # A máscara 0xFFFFFFFF garante que o resultado tenha 32 bits
        t ^= (t << 11) & 0xFFFFFFFF
        # Aplica a segunda transformação XOR: desloca t 8 bits à direita e faz XOR com t
        t ^= (t >> 8) & 0xFFFFFFFF
        # Aplica a terceira transformação XOR: faz XOR entre t e o resultado de (s XOR (s deslocado 19 bits à direita))
        t ^= (s ^ (s >> 19)) & 0xFFFFFFFF
        
        # Atualiza o primeiro valor do estado (índice 0) com o novo valor de t
        # A máscara 0xFFFFFFFF garante que o resultado tenha 32 bits
        self.state[0] = t & 0xFFFFFFFF
        # Retorna o valor gerado, garantindo que tenha 32 bits
        return t & 0xFFFFFFFF
    
    def random(self):
        """
        Retorna um número de ponto flutuante no intervalo [0.0, 1.0).
        
        Returns:
            float: Um número entre 0.0 e 1.0.
        """
        return self.next() / 0x100000000


class Xorshift128Plus:
    """
    Implementação do Xorshift128+ - uma variante que utiliza adição para
    melhorar a qualidade estatística (período 2^128-1).
    """
    def __init__(self, seed=None):
        """
        Inicializa o gerador com uma semente ou valores padrão.
        
        Args:
            seed (list, optional): Lista de 2 valores inteiros de 64 bits para inicializar o estado.
        """
        if seed is None:
            # Valores iniciais não-nulos por padrão
            self.state = [1234567890123456789, 9876543210987654321]
        else:
            # Verifica se todos os valores são zero
            if all(s == 0 for s in seed):
                self.state = [1234567890123456789, 9876543210987654321]
            else:
                self.state = [s & 0xFFFFFFFFFFFFFFFF for s in seed[:2]]
    
    def next(self):
        """
        Gera o próximo número pseudoaleatório.
        
        Returns:
            int: Um número pseudoaleatório de 64 bits.
        """
        # Armazena os dois valores do estado atual em variáveis temporárias
        s0 = self.state[0]  # Primeiro valor do estado (64 bits)
        s1 = self.state[1]  # Segundo valor do estado (64 bits)
        
        # Atualiza o primeiro valor do estado com o valor de s1
        # Isso faz parte da rotação do estado para a próxima iteração
        self.state[0] = s1
        
        # Aplica a primeira transformação XOR em s1:
        # Desloca s1 23 bits à esquerda, faz XOR com s1 original
        # A máscara 0xFFFFFFFFFFFFFFFF garante que o resultado tenha 64 bits
        s1 ^= (s1 << 23) & 0xFFFFFFFFFFFFFFFF
        
        # Calcula o novo valor para o segundo elemento do estado:
        # Combina s1 transformado com s0 original e mais duas operações de deslocamento e XOR
        # - s1 ^ s0: XOR entre s1 transformado e s0 original
        # - ^ (s1 >> 18): XOR com s1 deslocado 18 bits à direita
        # - ^ (s0 >> 5): XOR com s0 deslocado 5 bits à direita
        # A máscara final garante que o resultado tenha 64 bits
        self.state[1] = (s1 ^ s0 ^ (s1 >> 18) ^ (s0 >> 5)) & 0xFFFFFFFFFFFFFFFF
        
        # Retorna a soma dos dois valores do estado como o número pseudoaleatório
        # Esta adição é o que diferencia o Xorshift128+ do Xorshift128 padrão,
        # melhorando a qualidade estatística dos números gerados
        # A máscara garante que o resultado tenha 64 bits
        return (self.state[0] + self.state[1]) & 0xFFFFFFFFFFFFFFFF
    
    def random(self):
        """
        Retorna um número de ponto flutuante no intervalo [0.0, 1.0).
        
        Returns:
            float: Um número entre 0.0 e 1.0.
        """
        return self.next() / 0x10000000000000000
\end{verbatim}

\subsection{Implementação do Teste}

\begin{verbatim}
"""
Experimento para avaliação de desempenho de geradores de números pseudoaleatórios.

Este script avalia o tempo necessário para gerar números pseudoaleatórios de diferentes
tamanhos (40-4096 bits) usando diferentes implementações do algoritmo Xorshift.

O experimento gera múltiplos números para cada tamanho e calcula o tempo médio.
"""

import time
import sys
from main import Xorshift32, Xorshift64, Xorshift128, Xorshift128Plus

# Tamanhos de bits a serem testados
BIT_SIZES = [40, 56, 80, 128, 168, 224, 256, 512, 1024, 2048, 4096]

# Número de amostras para calcular o tempo médio
NUM_SAMPLES = 1000


def generate_random_bits(generator, num_bits):
    """
    Gera um número pseudoaleatório com o número especificado de bits.
    
    Args:
        generator: Objeto gerador Xorshift
        num_bits: Número de bits desejado
    
    Returns:
        int: Número pseudoaleatório com o tamanho especificado
    """
    result = 0
    bits_generated = 0
    
    # Determina o número de bits por chamada com base no tipo de gerador
    if isinstance(generator, Xorshift32):
        bits_per_call = 32
    elif isinstance(generator, Xorshift64) or isinstance(generator, Xorshift128Plus):
        bits_per_call = 64
    elif isinstance(generator, Xorshift128):
        bits_per_call = 32
    else:
        raise ValueError("Gerador não reconhecido")
    
    # Gera bits até atingir o tamanho desejado
    while bits_generated < num_bits:
        # Gera um novo número
        new_bits = generator.next()
        
        # Adiciona os novos bits ao resultado (shifts e OR)
        result = (result << bits_per_call) | new_bits
        bits_generated += bits_per_call
    
    # Ajusta para o tamanho exato solicitado (remove bits extras)
    if bits_generated > num_bits:
        extra_bits = bits_generated - num_bits
        result = result >> extra_bits
        
    # Garante que o número tenha exatamente o número de bits solicitado
    # Seta o bit mais significativo para 1
    result |= (1 << (num_bits - 1))
    
    return result


def measure_generation_time(generator_class, bit_size, seed=42):
    """
    Mede o tempo médio para gerar números pseudoaleatórios de determinado tamanho.
    
    Args:
        generator_class: Classe do gerador a ser usada
        bit_size: Tamanho do número em bits
        seed: Semente para inicializar o gerador
    
    Returns:
        float: Tempo médio em milissegundos
    """
    # Inicializa o gerador apropriadamente dependendo da classe
    if generator_class == Xorshift32 or generator_class == Xorshift64:
        generator = generator_class(seed=seed)
    elif generator_class == Xorshift128:
        generator = generator_class(seed=[seed, seed+1, seed+2, seed+3])
    elif generator_class == Xorshift128Plus:
        generator = generator_class(seed=[seed, seed+1])
    
    # Mede o tempo total para gerar NUM_SAMPLES números
    start_time = time.time()
    
    for _ in range(NUM_SAMPLES):
        generate_random_bits(generator, bit_size)
    
    end_time = time.time()
    
    # Calcula o tempo médio em milissegundos
    avg_time_ms = ((end_time - start_time) / NUM_SAMPLES) * 1000
    
    return avg_time_ms


def run_experiment():
    """
    Executa o experimento completo e exibe os resultados em formato de tabela.
    """
    generators = [
        ("Xorshift32", Xorshift32),
        ("Xorshift64", Xorshift64),
        ("Xorshift128", Xorshift128),
        ("Xorshift128Plus", Xorshift128Plus)
    ]
    
    # Imprime o cabeçalho da tabela
    print("\n" + "=" * 80)
    print(f"{'Algoritmo':<20} | {'Tamanho do Número':<20} | {'Tempo para gerar (ms)':<20}")
    print("-" * 80)
    
    # Executa o experimento para cada combinação de gerador e tamanho
    for gen_name, gen_class in generators:
        for bit_size in BIT_SIZES:
            # Mede o tempo de geração
            time_ms = measure_generation_time(gen_class, bit_size)
            
            # Formata e imprime o resultado
            print(f"{gen_name:<20} | {bit_size} bits{' ':<14} | {time_ms:.4f} ms")
        
        # Linha separadora entre geradores
        print("-" * 80)
    
    print("=" * 80)
    print(f"Experimento concluído. Cada medição representa a média de {NUM_SAMPLES} execuções.")


def verify_random_number_quality(show_numbers=False):
    """
    Verifica a qualidade dos números gerados por cada algoritmo.
    Esta função é útil para validar se os números gerados têm o tamanho especificado.
    
    Args:
        show_numbers: Se True, exibe os números gerados para inspeção visual
    """
    print("\nVerificação da qualidade dos números gerados:")
    print("-" * 80)
    
    generators = [
        ("Xorshift32", Xorshift32(seed=42)),
        ("Xorshift64", Xorshift64(seed=42)),
        ("Xorshift128", Xorshift128(seed=[42, 43, 44, 45])),
        ("Xorshift128Plus", Xorshift128Plus(seed=[42, 43]))
    ]
    
    # Seleciona alguns tamanhos para testar
    test_sizes = [40, 128, 256, 1024]
    
    for gen_name, generator in generators:
        print(f"\nGerador: {gen_name}")
        
        for bit_size in test_sizes:
            number = generate_random_bits(generator, bit_size)
            
            # Verifica se o número tem o tamanho correto (em bits)
            binary = bin(number)[2:]  # Remove o prefixo '0b'
            actual_bits = len(binary)
            
            print(f"  Tamanho solicitado: {bit_size} bits")
            print(f"  Tamanho real: {actual_bits} bits")
            
            if show_numbers:
                # Limita a exibição para evitar números muito grandes
                if bit_size <= 128:
                    print(f"  Número binário: {binary}")
                else:
                    print(f"  Número binário: {binary[:64]}...{binary[-64:]} (parte inicial e final)")
            
            print("-" * 40)


if __name__ == "__main__":
    # Verifica se o usuário quer executar o experimento com verificação de qualidade
    verify_quality = len(sys.argv) > 1 and sys.argv[1] == "--verify"
    
    if verify_quality:
        verify_random_number_quality(show_numbers=True)
    else:
        print("\nIniciando experimento de geração de números pseudoaleatórios...")
        print(f"Gerando {NUM_SAMPLES} amostras para cada combinação algoritmo-tamanho")
        run_experiment()
        print("\nDica: Execute com a flag --verify para testar a qualidade dos números gerados.")
\end{verbatim}

\section{Blum Blum Shub}\label{apx:bbs-impl}

\subsection{Implementação do Algoritmo}

\begin{verbatim}
"""
Implementação do algoritmo Blum Blum Shub (BBS) para geração de números pseudoaleatórios.

BBS é um gerador de números pseudoaleatórios criptograficamente seguro proposto em 1986
por Lenore Blum, Manuel Blum e Michael Shub. Sua segurança baseia-se na dificuldade
do problema de fatoração de inteiros.
"""

import random
import math
import time
from typing import Tuple, Optional


def is_prime(n: int, k: int = 40) -> bool:
    """
    Verifica se um número é provavelmente primo usando o teste de Miller-Rabin.
    
    Args:
        n: O número a ser testado
        k: Número de iterações para o teste (maior k = maior confiabilidade)
        
    Returns:
        bool: True se n for provavelmente primo, False caso contrário
    """
    # Casos básicos
    if n <= 1:
        return False
    if n <= 3:
        return True
    if n % 2 == 0:
        return False
    
    # Expressando n-1 como 2^r * d
    r, d = 0, n - 1
    while d % 2 == 0:
        r += 1
        d //= 2
    
    # Teste de Miller-Rabin
    for _ in range(k):
        a = random.randint(2, n - 2)
        x = pow(a, d, n)
        if x == 1 or x == n - 1:
            continue
        for _ in range(r - 1):
            x = pow(x, 2, n)
            if x == n - 1:
                break
        else:
            return False
    return True


def generate_prime(bits: int) -> int:
    """
    Gera um número primo com o número especificado de bits.
    
    Args:
        bits: Número de bits do primo a ser gerado
        
    Returns:
        int: Um número primo com o número especificado de bits
    """
    while True:
        # Gera um número aleatório com o número especificado de bits
        p = random.randint(2**(bits-1), 2**bits - 1)
        
        # Garante que o número é ímpar
        if p % 2 == 0:
            p += 1
        
        # Verifica se é primo usando o teste de Miller-Rabin
        if is_prime(p):
            return p


def generate_blum_integer(bits: int) -> Tuple[int, int, int]:
    """
    Gera um inteiro de Blum, que é o produto de dois primos grandes, ambos
    congruentes a 3 mod 4.
    
    Args:
        bits: Número de bits para cada primo (o inteiro de Blum terá aproximadamente 2*bits)
        
    Returns:
        Tuple[int, int, int]: (n, p, q) onde n = p * q é o inteiro de Blum
    """
    # Gera o primeiro primo p tal que p \equiv 3 (mod 4)
    while True:
        p = generate_prime(bits)
        if p % 4 == 3:
            break
    
    # Gera o segundo primo q tal que q \equiv 3 (mod 4)
    while True:
        q = generate_prime(bits)
        if q % 4 == 3 and p != q:  # Garante que p e q são diferentes
            break
    
    # Retorna o inteiro de Blum n = p * q
    return p * q, p, q


def gcd(a: int, b: int) -> int:
    """
    Calcula o Máximo Divisor Comum (MDC) entre dois números.
    
    Args:
        a: Primeiro número
        b: Segundo número
        
    Returns:
        int: O MDC entre a e b
    """
    while b:
        a, b = b, a % b
    return a


class BlumBlumShub:
    """
    Implementação do gerador de números pseudoaleatórios Blum Blum Shub.
    """
    
    def __init__(self, seed: Optional[int] = None, n: Optional[int] = None, 
                 p: Optional[int] = None, q: Optional[int] = None, bits: int = 512):
        """
        Inicializa o gerador BBS.
        
        Args:
            seed: Semente inicial para o gerador (opcional)
            n: O inteiro de Blum a ser usado (opcional)
            p: Primeiro primo utilizado na geração de n (opcional)
            q: Segundo primo utilizado na geração de n (opcional)
            bits: Número de bits para cada um dos primos (se n não for fornecido)
        """
        # Se n não foi fornecido, gera um novo inteiro de Blum
        if n is None:
            self.n, self.p, self.q = generate_blum_integer(bits)
        else:
            # Usa os valores fornecidos
            self.n = n
            self.p = p
            self.q = q
        
        # Se a semente não foi fornecida, gera uma aleatória
        if seed is None:
            # Gera uma semente aleatória coprima com n
            while True:
                seed = random.randint(2, self.n - 1)
                if gcd(seed, self.n) == 1:
                    break
        else:
            # Verifica se a semente fornecida é válida
            if seed <= 1 or seed >= self.n:
                raise ValueError("A semente deve estar no intervalo [2, n-1]")
            if gcd(seed, self.n) != 1:
                raise ValueError("A semente deve ser coprima com n")
        
        # Inicializa o estado com x_0 = s^2 mod n
        self.state = (seed * seed) % self.n
    
    def next_bit(self) -> int:
        """
        Gera o próximo bit pseudoaleatório.
        
        Returns:
            int: 0 ou 1 (o bit menos significativo do novo estado)
        """
        # Calcula x_{i+1} = x_i² mod n
        self.state = (self.state * self.state) % self.n
        
        # Retorna o bit menos significativo
        return self.state % 2
    
    def next_byte(self) -> int:
        """
        Gera o próximo byte (8 bits) pseudoaleatório.
        
        Returns:
            int: Um número entre 0 e 255
        """
        byte = 0
        # Gera 8 bits para formar um byte
        for i in range(8):
            bit = self.next_bit()
            # Constrói o byte bit a bit (do mais significativo para o menos)
            byte = (byte << 1) | bit
        return byte
    
    def random_bits(self, num_bits: int) -> int:
        """
        Gera um número com a quantidade especificada de bits.
        
        Args:
            num_bits: Número de bits a serem gerados
            
        Returns:
            int: Um número pseudoaleatório com o tamanho especificado
        """
        result = 0
        # Gera os bits um a um
        for _ in range(num_bits):
            bit = self.next_bit()
            # Constrói o número bit a bit
            result = (result << 1) | bit
        
        # Garante que o bit mais significativo seja 1 (para garantir exatamente num_bits bits)
        result |= (1 << (num_bits - 1))
        
        return result
    
    def random(self) -> float:
        """
        Gera um número de ponto flutuante pseudoaleatório no intervalo [0.0, 1.0).
        
        Returns:
            float: Um número entre 0.0 e 1.0
        """
        # Usa 53 bits para garantir precisão suficiente para um float
        return self.random_bits(53) / (1 << 53)
\end{verbatim}

\subsection{Implementação do Teste}

\begin{verbatim}
"""
Experimento para avaliar o desempenho do gerador de números pseudoaleatórios Blum Blum Shub.

Este script mede o tempo necessário para gerar números pseudoaleatórios de diferentes
tamanhos (40-4096 bits) usando o algoritmo Blum Blum Shub (BBS).
"""

import time
import sys
from main import BlumBlumShub

# Tamanhos de bits a serem testados
BIT_SIZES = [40, 56, 80, 128, 168, 224, 256, 512, 1024, 2048, 4096]

# Número de amostras para calcular o tempo médio
NUM_SAMPLES = 10  # Menos amostras para BBS porque é muito mais lento


def measure_generation_time(bit_size):
    """
    Mede o tempo médio para gerar números pseudoaleatórios de determinado tamanho.
    
    Args:
        bit_size: Tamanho do número em bits
    
    Returns:
        float: Tempo médio em milissegundos
    """
    # Inicializa o gerador BBS com um módulo de tamanho adequado
    # O tamanho do módulo deve ser pelo menos duas vezes maior que o do número a ser gerado
    module_bits = max(512, bit_size * 2)
    
    # Medimos também o tempo de inicialização
    init_start_time = time.time()
    bbs = BlumBlumShub(bits=module_bits // 2)
    init_time = (time.time() - init_start_time) * 1000  # em ms
    
    # Mede o tempo total para gerar NUM_SAMPLES números
    total_time = 0
    
    for _ in range(NUM_SAMPLES):
        start_time = time.time()
        bbs.random_bits(bit_size)
        end_time = time.time()
        total_time += (end_time - start_time) * 1000  # em ms
    
    # Calcula o tempo médio em milissegundos
    avg_time_ms = total_time / NUM_SAMPLES
    
    return avg_time_ms, init_time


def run_experiment():
    """
    Executa o experimento completo e exibe os resultados em formato de tabela.
    """
    print("\n" + "=" * 80)
    print("Experimento de geração de números pseudoaleatórios com Blum Blum Shub (BBS)")
    print("=" * 80)
    print(f"{'Tamanho do Número':<20} | {'Tempo de Init (ms)':<20} | {'Tempo de Geração (ms)':<25}")
    print("-" * 80)
    
    # Executa o experimento para cada tamanho
    for bit_size in BIT_SIZES:
        try:
            # Mede o tempo de geração
            gen_time_ms, init_time_ms = measure_generation_time(bit_size)
            
            # Formata e imprime o resultado
            print(f"{bit_size} bits{' ':<14} | {init_time_ms:.4f} ms{' ':<10} | {gen_time_ms:.4f} ms")
            
            # Salva o resultado parcial para não perder progresso
            with open("bbs_results.txt", "a") as f:
                f.write(f"{bit_size},{init_time_ms:.4f},{gen_time_ms:.4f}\n")
                
        except Exception as e:
            print(f"{bit_size} bits{' ':<14} | ERRO: {str(e)}")
    
    print("=" * 80)
    print(f"Experimento concluído. Cada medição representa a média de {NUM_SAMPLES} execuções.")


def verify_random_number_quality():
    """
    Verifica a qualidade dos números gerados pelo algoritmo BBS.
    Esta função é útil para validar se os números gerados têm o tamanho especificado.
    """
    print("\nVerificação da qualidade dos números gerados pelo Blum Blum Shub:")
    print("-" * 80)
    
    # Cria um gerador BBS
    bbs = BlumBlumShub(bits=512)
    
    # Seleciona alguns tamanhos para testar
    test_sizes = [40, 128, 256]
    
    for bit_size in test_sizes:
        print(f"\nGerando número de {bit_size} bits...")
        number = bbs.random_bits(bit_size)
        
        # Verifica se o número tem o tamanho correto
        binary = bin(number)[2:]  # Remove o prefixo '0b'
        actual_bits = len(binary)
        
        print(f"  Tamanho solicitado: {bit_size} bits")
        print(f"  Tamanho real: {actual_bits} bits")
        print(f"  Número binário: {binary}")
        print(f"  Número decimal: {number}")
        
        # Verificação simples da distribuição de bits
        ones = binary.count('1')
        zeros = binary.count('0')
        print(f"  Distribuição: {ones} bits '1' ({ones/actual_bits:.2%}), {zeros} bits '0' ({zeros/actual_bits:.2%})")
        
        print("-" * 40)


if __name__ == "__main__":
    # Verifica se o usuário quer executar o experimento ou testar a qualidade
    verify_quality = len(sys.argv) > 1 and sys.argv[1] == "--verify"
    
    if verify_quality:
        verify_random_number_quality()
    else:
        print("\nIniciando experimento com Blum Blum Shub...")
        print(f"Gerando {NUM_SAMPLES} amostras para cada tamanho de bits")
        print("Nota: Este algoritmo é significativamente mais lento que outros PRNGs.")
        print("      O experimento pode levar vários minutos ou até horas para completar.")
        run_experiment()
        print("\nDica: Execute com a flag --verify para testar a qualidade dos números gerados.")
\end{verbatim}
 
\section{Teste de Primalidade de Fermat}\label{apx:fermat-impl}

\subsection{Implementação do Algoritmo}

\begin{verbatim}
#!/usr/bin/env python3
"""
Implementação do Teste de Primalidade de Fermat

Este algoritmo implementa o Teste de Primalidade de Fermat, um teste probabilístico
que verifica se um número é provavelmente primo baseado no Pequeno Teorema de Fermat.
"""

import random
import time
from typing import Tuple


def gcd(a: int, b: int) -> int:
    """
    Calcula o Máximo Divisor Comum (MDC) entre dois números usando o algoritmo de Euclides.
    
    Args:
        a: Primeiro número
        b: Segundo número
        
    Returns:
        O MDC entre a e b
    """
    while b:
        a, b = b, a % b
    return a


def power_mod(base: int, exponent: int, modulus: int) -> int:
    """
    Calcula (base^exponent) % modulus de forma eficiente usando exponenciação modular rápida.
    
    Args:
        base: A base
        exponent: O expoente
        modulus: O módulo
        
    Returns:
        (base^exponent) % modulus
    """
    result = 1
    base = base % modulus
    
    while exponent > 0:
        # Se o bit menos significativo é 1, multiplique pelo atual resultado
        if exponent & 1:
            result = (result * base) % modulus
            
        # Expoente é reduzido pela metade
        exponent >>= 1
        
        # Base é elevada ao quadrado
        base = (base * base) % modulus
        
    return result


def fermat_primality_test(n: int, k: int = 5) -> Tuple[bool, list]:
    """
    Executa o teste de primalidade de Fermat em um número dado.
    
    Args:
        n: O número a ser testado
        k: Número de iterações (maior k = maior confiabilidade)
        
    Returns:
        (Resultado, Testemunhas):
            - Resultado: True se n for provavelmente primo, False se for definitivamente composto
            - Testemunhas: Lista contendo as testemunhas de Fermat (se composto) ou as bases testadas (se primo)
    """
    # Casos especiais
    if n <= 1:
        return False, []
    if n <= 3:
        return True, []
    if n % 2 == 0:
        return False, [2]  # 2 é uma testemunha para números pares > 2
    
    witnesses = []
    
    # Executar k testes
    for _ in range(k):
        # Escolher base aleatória entre 2 e n-2
        a = random.randint(2, n-2)
        
        # Verificar se a e n são coprimos
        if gcd(a, n) != 1:
            return False, [a]  # Encontramos um fator de n
        
        # Verificar a congruência de Fermat
        if power_mod(a, n-1, n) != 1:
            return False, [a]  # a é uma testemunha de Fermat para a composição de n
        
        witnesses.append(a)
    
    # Se chegamos aqui, n passou em todos os k testes
    return True, witnesses
\end{verbatim}

\subsection{Implementação do Teste}

\begin{verbatim}
#!/usr/bin/env python3
"""
Experimento de Geração de Números Primos com o Teste de Primalidade de Fermat

Este script realiza um experimento para gerar números primos de diferentes tamanhos
(40 a 4096 bits) e mede o tempo necessário para encontrar cada primo.
"""

import random
import time
import sys
from typing import Tuple, List, Dict
import os
import json
from datetime import datetime

# Importa as funções do código principal
sys.path.append(os.path.dirname(os.path.abspath(__file__)))
from main import fermat_primality_test, power_mod, gcd


def generate_random_bits(bits: int) -> int:
    """
    Gera um número aleatório com o número especificado de bits.
    
    Args:
        bits: Número de bits do número a ser gerado
        
    Returns:
        Um número aleatório com o número especificado de bits
    """
    # Gera um número entre 2^(bits-1) e 2^bits - 1
    return random.randint(2 ** (bits - 1), 2 ** bits - 1)


def generate_random_odd(bits: int) -> int:
    """
    Gera um número ímpar aleatório com o número especificado de bits.
    
    Args:
        bits: Número de bits do número a ser gerado
        
    Returns:
        Um número ímpar aleatório com o número especificado de bits
    """
    num = generate_random_bits(bits)
    # Garante que o número é ímpar
    if num % 2 == 0:
        num += 1
    return num


def find_prime(bits: int, timeout_seconds: float = 300, k: int = 10) -> Tuple[int, int, float]:
    """
    Encontra um número primo com o número especificado de bits usando o Teste de Fermat,
    com um limite de tempo para evitar que o programa fique preso.
    
    Args:
        bits: Número de bits do primo a ser encontrado
        timeout_seconds: Tempo máximo em segundos para tentar
        k: Número de iterações para o teste de Fermat
        
    Returns:
        Tupla contendo (número primo, número de tentativas, tempo em ms)
    """
    start_time = time.time()
    end_time = start_time + timeout_seconds
    attempts = 0
    
    # Geramos um número ímpar aleatório inicial
    num = generate_random_odd(bits)
    
    print(f"Buscando primo de {bits} bits...", end="", flush=True)
    
    while time.time() < end_time:
        attempts += 1
        
        if attempts % 10 == 0:
            print(".", end="", flush=True)
        
        # Teste de Fermat
        is_probable_prime, _ = fermat_primality_test(num, k)
        
        if is_probable_prime:
            elapsed_ms = (time.time() - start_time) * 1000
            print(f" Encontrado após {attempts} tentativas!")
            return num, attempts, elapsed_ms
        
        # Se não for primo, tente o próximo número ímpar
        num += 2
    
    # Se chegamos aqui, não encontramos um primo dentro do tempo limite
    elapsed_ms = (time.time() - start_time) * 1000
    print(f" Tempo esgotado após {attempts} tentativas.")
    return 0, attempts, elapsed_ms


def run_experiment() -> Dict:
    """
    Executa o experimento completo para todos os tamanhos de bits especificados.
    
    Returns:
        Um dicionário com os resultados do experimento
    """
    # Tamanhos em bits para testar
    bit_sizes = [40, 56, 80, 128, 168, 224, 256, 512, 1024, 2048, 4096]
    
    # Configurações de timeout por tamanho
    timeouts = {
        40: 60,     # 1 minuto
        56: 60,     # 1 minuto
        80: 60,     # 1 minuto
        128: 120,   # 2 minutos
        168: 180,   # 3 minutos
        224: 240,   # 4 minutos
        256: 300,   # 5 minutos
        512: 600,   # 10 minutos
        1024: 900,  # 15 minutos
        2048: 1200, # 20 minutos
        4096: 1800, # 30 minutos
    }
    
    # Parâmetros do teste de Fermat
    k_small = 10    # Para tamanhos até 256 bits
    k_large = 5     # Para tamanhos maiores
    
    results = {
        "timestamp": datetime.now().strftime("%Y-%m-%d %H:%M:%S"),
        "algorithm": "Fermat Primality Test",
        "results": []
    }
    
    print("\nIniciando experimento de geração de números primos usando o Teste de Fermat")
    print("=" * 80)
    print(f"{'Tamanho (bits)':<15} {'Tentativas':<12} {'Tempo (ms)':<15} {'Status'}")
    print("-" * 80)
    
    for bits in bit_sizes:
        # Ajusta parâmetros com base no tamanho
        timeout = timeouts.get(bits, 300)  # Default de 5 minutos se não especificado
        k = k_small if bits <= 256 else k_large
        
        try:
            prime, attempts, elapsed_ms = find_prime(bits, timeout, k)
            
            status = "Sucesso" if prime > 0 else "Timeout"
            
            print(f"{bits:<15} {attempts:<12} {elapsed_ms:.2f} ms      {status}")
            
            result_entry = {
                "bits": bits,
                "prime": prime,
                "attempts": attempts,
                "time_ms": round(elapsed_ms, 2),
                "status": status
            }
            results["results"].append(result_entry)
            
            # Salvar resultado parcial para não perder tudo se houver falhas
            with open("prime_generation_results.json", "w") as f:
                json.dump(results, f, indent=2)
                
        except Exception as e:
            print(f"{bits:<15} {'Erro':<15} {'-':<12} {'-':<15} {str(e)}")
            results["results"].append({
                "bits": bits,
                "prime": 0,
                "attempts": 0,
                "time_ms": 0,
                "status": f"Erro: {str(e)}"
            })
    
    return results


def format_results_as_table(results: Dict) -> str:
    """
    Formata os resultados como uma tabela Markdown.
    
    Args:
        results: Os resultados do experimento
        
    Returns:
        Uma string contendo a tabela em formato Markdown
    """
    table = "| Algoritmo | Tamanho do Número | Tentativas | Tempo para gerar |\n"
    table += "|-----------|------------------|------------|-------------------|\n"
    
    for result in results["results"]:
        algorithm = results["algorithm"]
        bits = f"{result['bits']} bits"
        
        if result["status"] == "Sucesso":
            attempts_str = str(result["attempts"])
            time_str = f"{result['time_ms']} ms"
        else:
            attempts_str = str(result["attempts"])
            if result["status"] == "Timeout":
                time_str = f"{result['time_ms']} ms (timeout)"
            else:
                time_str = f"{result['time_ms']} ms (erro)"
        
        table += f"| {algorithm} | {bits} | {attempts_str} | {time_str} |\n"
    
    return table


def generate_report(results: Dict) -> None:
    """
    Gera um relatório completo do experimento.
    
    Args:
        results: Os resultados do experimento
    """
    # Cria o relatório
    report = "# Experimento de Geração de Números Primos com o Teste de Fermat\n\n"
    report += f"Data/Hora: {results['timestamp']}\n\n"
    
    # Adiciona a tabela
    report += "## Resultados\n\n"
    report += format_results_as_table(results)
    report += "\n\n"
    
    # Análise e observações
    report += "## Observações\n\n"
    
    # Analisa taxa de sucesso
    successful = sum(1 for r in results["results"] if r["status"] == "Sucesso")
    total = len(results["results"])
    success_rate = (successful / total) * 100 if total > 0 else 0
    
    report += f"- Taxa de sucesso: {successful}/{total} ({success_rate:.1f}%)\n"
    
    # Analisa tempos e tentativas
    if successful > 0:
        times = [r["time_ms"] for r in results["results"] if r["status"] == "Sucesso"]
        attempts = [r["attempts"] for r in results["results"] if r["status"] == "Sucesso"]
        
        avg_time = sum(times) / len(times)
        max_time = max(times)
        min_time = min(times)
        
        avg_attempts = sum(attempts) / len(attempts)
        max_attempts = max(attempts)
        min_attempts = min(attempts)
        
        report += f"- Tempo médio: {avg_time:.2f} ms\n"
        report += f"- Tempo mínimo: {min_time:.2f} ms\n"
        report += f"- Tempo máximo: {max_time:.2f} ms\n"
        report += f"- Tentativas médias: {avg_attempts:.2f}\n"
        report += f"- Tentativas mínimas: {min_attempts}\n"
        report += f"- Tentativas máximas: {max_attempts}\n\n"
        
        # Análise da relação entre tamanho de bits e número de tentativas
        sizes = [r["bits"] for r in results["results"] if r["status"] == "Sucesso"]
        if len(sizes) > 1:
            report += "### Relação entre tamanho e esforço\n\n"
            report += "- À medida que o tamanho em bits aumenta, nota-se:\n"
            
            # Ordenar resultados por tamanho de bits para comparação
            sorted_results = sorted([r for r in results["results"] if r["status"] == "Sucesso"], 
                                    key=lambda x: x["bits"])
            
            if len(sorted_results) >= 2:
                smallest = sorted_results[0]
                largest = sorted_results[-1]
                report += f"  - Para {smallest['bits']} bits: {smallest['attempts']} tentativas, {smallest['time_ms']:.2f} ms\n"
                report += f"  - Para {largest['bits']} bits: {largest['attempts']} tentativas, {largest['time_ms']:.2f} ms\n"
                
                attempt_increase = largest['attempts'] / smallest['attempts'] if smallest['attempts'] > 0 else 0
                time_increase = largest['time_ms'] / smallest['time_ms'] if smallest['time_ms'] > 0 else 0
                
                report += f"  - Aumento de tentativas: {attempt_increase:.2f}x\n"
                report += f"  - Aumento de tempo: {time_increase:.2f}x\n\n"
\end{verbatim}

\section{Teste de Primalidade de Miller-Rabin}\label{apx:miller-rabin-impl}

\subsection{Implementação do Algoritmo}

\begin{verbatim}
#!/usr/bin/env python3
"""
Implementação do Teste de Primalidade de Miller-Rabin

Este módulo implementa o algoritmo de teste de primalidade Miller-Rabin,
um algoritmo probabilístico eficiente para determinar se um número é provavelmente primo.

Autor: João Pedro Schmidt Cordeiro
Data: Abril 2024
"""

import random
import time
from typing import Tuple, List


def decompose(n: int) -> Tuple[int, int]:
    """
    Decompõe n-1 como (2^s) * d, onde d é ímpar.
    
    Args:
        n: O número a ser decomposto (n-1)
    
    Returns:
        Uma tupla (s, d) onde s e d satisfazem n-1 = (2^s) * d e d é ímpar
    """
    n_minus_1 = n - 1
    s = 0
    d = n_minus_1
    
    # Enquanto d for par, divide por 2 e incrementa s
    while d % 2 == 0:
        d //= 2
        s += 1
    
    return s, d


def miller_rabin_round(n: int, a: int) -> bool:
    """
    Executa uma rodada do teste de Miller-Rabin com a base 'a'.
    
    Args:
        n: O número a ser testado para primalidade
        a: A base para o teste (2 <= a <= n-2)
    
    Returns:
        True se n passa no teste para a base a, False caso contrário
    """
    if n == a:
        return True
    
    if n % a == 0:
        return False
    
    # Decompõe n-1 como 2^s * d
    s, d = decompose(n)
    
    # Calcula x = a^d mod n
    x = pow(a, d, n)
    
    # Se x == 1 ou x == n-1, n passa no teste para essa base
    if x == 1 or x == n - 1:
        return True
    
    # Calcula x = x^2 mod n para s-1 iterações
    for _ in range(s - 1):
        x = pow(x, 2, n)
        # Se x == n-1, n passa no teste para essa base
        if x == n - 1:
            return True
        # Se x == 1, encontramos uma raiz quadrada não-trivial de 1, então n é composto
        if x == 1:
            return False
    
    # Se chegamos aqui, n é composto para esta base
    return False


def is_prime_miller_rabin(n: int, k: int = 40) -> bool:
    """
    Determina se n é provavelmente primo usando o teste de Miller-Rabin.
    
    Args:
        n: O número a ser testado
        k: O número de rodadas/bases a serem testadas (padrão: 40)
           Quanto maior o valor de k, menor a probabilidade de erro
    
    Returns:
        True se n é provavelmente primo, False se n é definitivamente composto
    """
    # Casos triviais
    if n <= 1:
        return False
    if n <= 3:
        return True
    if n % 2 == 0:
        return False
    
    # Para números pequenos, podemos usar um conjunto fixo de bases
    # que garantem determinismo até certo limite
    # Recomendação de Menezes, van Oorschot e Vanstone (1996)
    if n < 1_373_653:
        # Para n < 1,373,653, é suficiente testar as bases 2, 3
        bases_deterministic = [2, 3]
    elif n < 9_080_191:
        # Para n < 9,080,191, é suficiente testar as bases 31, 73
        bases_deterministic = [31, 73]
    elif n < 25_326_001:
        # Para n < 25,326,001, é suficiente testar as bases 2, 3, 5
        bases_deterministic = [2, 3, 5]
    elif n < 3_215_031_751:
        # Para n < 3,215,031,751, é suficiente testar as bases 2, 3, 5, 7
        bases_deterministic = [2, 3, 5, 7]
    elif n < 4_759_123_141:
        # Para n < 4,759,123,141, é suficiente testar as bases 2, 7, 61
        bases_deterministic = [2, 7, 61]
    elif n < 1_122_004_669_633:
        # Para n < 1,122,004,669,633, é suficiente testar as bases 2, 13, 23, 1662803
        bases_deterministic = [2, 13, 23, 1662803]
    elif n < 2_152_302_898_747:
        # Para n < 2,152,302,898,747, é suficiente testar as bases 2, 3, 5, 7, 11
        bases_deterministic = [2, 3, 5, 7, 11]
    elif n < 3_474_749_660_383:
        # Para n < 3,474,749,660,383, é suficiente testar as bases 2, 3, 5, 7, 11, 13
        bases_deterministic = [2, 3, 5, 7, 11, 13]
    elif n < 341_550_071_728_321:
        # Para n < 341,550,071,728,321, é suficiente testar as bases 2, 3, 5, 7, 11, 13, 17
        bases_deterministic = [2, 3, 5, 7, 11, 13, 17]
    elif n < 2**64:
        # Para n < 2^64, é suficiente testar as bases 2, 3, 5, 7, 11, 13, 17, 19, 23, 29, 31, 37
        # Conforme provado por Pomerance, Selfridge e Wagstaff e ampliado por Jaeschke
        bases_deterministic = [2, 3, 5, 7, 11, 13, 17, 19, 23, 29, 31, 37]
    else:
        # Para números maiores, usamos bases aleatórias
        for _ in range(k):
            a = random.randint(2, n - 2)
            if not miller_rabin_round(n, a):
                return False
        return True
    
    # Teste com as bases determinísticas selecionadas
    for a in bases_deterministic:
        if a >= n:
            break
        if not miller_rabin_round(n, a):
            return False
    
    # Se todas as k bases passaram no teste, n é provavelmente primo
    return True


def generate_probable_prime(bits: int, k: int = 40) -> int:
    """
    Gera um número provavelmente primo com o número especificado de bits.
    
    Args:
        bits: Número de bits do primo a ser gerado
        k: Número de rodadas no teste de Miller-Rabin (padrão: 40)
    
    Returns:
        Um número provavelmente primo com o número especificado de bits
    """
    while True:
        # Gera um número aleatório com o número especificado de bits
        # Garante que o número é ímpar e tem exatamente o número de bits solicitado
        n = random.getrandbits(bits)
        n |= (1 << (bits - 1)) | 1  # Garante que o bit mais significativo é 1 e que é ímpar
        
        if is_prime_miller_rabin(n, k):
            return n
\end{verbatim}

\subsection{Implementação do Teste}

\begin{verbatim}
#!/usr/bin/env python3
"""
Experimento de Geração de Números Primos usando o Teste de Miller-Rabin

Este script realiza experimentos de geração de números primos de diferentes tamanhos
usando o algoritmo de Miller-Rabin. Ele gera uma tabela com o tempo necessário 
para encontrar um número primo para cada tamanho especificado.

Autor: João Pedro Schmidt Cordeiro
Data: Abril 2024
"""

import sys
import time
import random
from typing import Dict, Tuple, List
from datetime import datetime

# Importa as funções do arquivo principal
from main import is_prime_miller_rabin, generate_probable_prime


def find_prime_with_timeout(bits: int, timeout_seconds: float = 300) -> Tuple[int, float, int, bool]:
    """
    Tenta encontrar um número primo com o número especificado de bits,
    com um limite de tempo para evitar que o programa fique preso.
    
    Args:
        bits: Número de bits do primo a ser gerado
        timeout_seconds: Tempo máximo em segundos para tentar
        
    Returns:
        Uma tupla (primo, tempo_em_ms, tentativas, sucesso)
    """
    print(f"Testando geração de primo de {bits} bits...", end="", flush=True)
    
    start_time = time.time()
    end_time = start_time + timeout_seconds
    attempts = 0
    prime = None
    
    while time.time() < end_time:
        attempts += 1
        if attempts % 10 == 0:
            print(".", end="", flush=True)
        
        # Gera um número ímpar aleatório com o número exato de bits
        n = random.getrandbits(bits)
        n |= (1 << (bits - 1)) | 1  # Garante MSB=1 e número ímpar
        
        if is_prime_miller_rabin(n):
            prime = n
            break
    
    duration = (time.time() - start_time) * 1000  # Converte para milissegundos
    success = prime is not None
    
    if success:
        print(f" Encontrado em {attempts} tentativas e {duration:.2f} ms")
    else:
        print(f" Falha após {attempts} tentativas e {timeout_seconds*1000:.2f} ms")
    
    return prime, duration, attempts, success


def format_number_compact(n: int) -> str:
    """
    Formata um número grande de forma compacta, mostrando apenas o início e o fim.
    
    Args:
        n: O número a ser formatado
        
    Returns:
        Uma string com o número formatado
    """
    if n is None:
        return "N/A"
    
    str_n = str(n)
    if len(str_n) <= 20:
        return str_n
    
    return f"{str_n[:10]}...{str_n[-10:]}"


def run_prime_experiment() -> Dict:
    """
    Executa o experimento de geração de números primos.
    
    Returns:
        Um dicionário com os resultados do experimento
    """
    bit_sizes = [40, 56, 80, 128, 168, 224, 256, 512, 1024, 2048, 4096]
    results = {
        "algorithm": "Miller-Rabin",
        "timestamp": datetime.now().strftime("%Y-%m-%d %H:%M:%S"),
        "results": []
    }
    
    print(f"Experimento iniciado em: {results['timestamp']}")
    print("=" * 80)
    
    # Tempo limite aumenta com o tamanho do número
    for bits in bit_sizes:
        # Define um timeout adequado baseado no tamanho
        if bits <= 128:
            timeout = 60  # 1 minuto para números pequenos
        elif bits <= 512:
            timeout = 300  # 5 minutos para números médios
        elif bits <= 1024:
            timeout = 600  # 10 minutos para números grandes
        else:
            timeout = 1200  # 20 minutos para números muito grandes
        
        prime, duration, attempts, success = find_prime_with_timeout(bits, timeout)
        
        result_entry = {
            "bits": bits,
            "prime": prime,
            "time_ms": duration,
            "attempts": attempts,
            "status": "Sucesso" if success else "Falha"
        }
        
        results["results"].append(result_entry)
    
    print("=" * 80)
    print(f"Experimento concluído em: {datetime.now().strftime('%Y-%m-%d %H:%M:%S')}")
    
    return results


def format_results_as_table(results: Dict) -> str:
    """
    Formata os resultados como uma tabela Markdown.
    
    Args:
        results: Os resultados do experimento
        
    Returns:
        Uma string contendo a tabela em formato Markdown
    """
    table = "| Algoritmo | Tamanho do Número | Tentativas | Tempo para gerar |\n"
    table += "|-----------|-------------------|------------|-------------------|\n"
    
    for result in results["results"]:
        algorithm = results["algorithm"]
        bits = f"{result['bits']} bits"
        
        if result["status"] == "Sucesso":
            attempts_str = str(result["attempts"])
            time_str = f"{result['time_ms']:.2f} ms"
        else:
            attempts_str = str(result["attempts"])
            time_str = "Timeout"
        
        table += f"| {algorithm} | {bits} | {attempts_str} | {time_str} |\n"
    
    return table


def generate_report(results: Dict) -> None:
    """
    Gera um relatório completo do experimento.
    
    Args:
        results: Os resultados do experimento
    """
    # Cria o relatório
    report = "# Experimento de Geração de Números Primos com o Teste de Miller-Rabin\n\n"
    report += f"Data/Hora: {results['timestamp']}\n\n"
    
    # Adiciona a tabela
    report += "## Resultados\n\n"
    report += format_results_as_table(results)
    report += "\n\n"
    
    # Análise e observações
    report += "## Observações\n\n"
    
    # Analisa taxa de sucesso
    successful = sum(1 for r in results["results"] if r["status"] == "Sucesso")
    total = len(results["results"])
    success_rate = (successful / total) * 100 if total > 0 else 0
    
    report += f"- Taxa de sucesso: {successful}/{total} ({success_rate:.1f}%)\n"
    
    # Analisa tempos e tentativas
    if successful > 0:
        times = [r["time_ms"] for r in results["results"] if r["status"] == "Sucesso"]
        attempts = [r["attempts"] for r in results["results"] if r["status"] == "Sucesso"]
        
        avg_time = sum(times) / len(times)
        max_time = max(times)
        min_time = min(times)
        
        avg_attempts = sum(attempts) / len(attempts)
        max_attempts = max(attempts)
        min_attempts = min(attempts)
        
        report += f"- Tempo médio: {avg_time:.2f} ms\n"
        report += f"- Tempo mínimo: {min_time:.2f} ms\n"
        report += f"- Tempo máximo: {max_time:.2f} ms\n"
        report += f"- Tentativas médias: {avg_attempts:.2f}\n"
        report += f"- Tentativas mínimas: {min_attempts}\n"
        report += f"- Tentativas máximas: {max_attempts}\n\n"
        
        # Análise da relação entre tamanho de bits e número de tentativas
        sizes = [r["bits"] for r in results["results"] if r["status"] == "Sucesso"]
        if len(sizes) > 1:
            report += "### Relação entre tamanho e esforço\n\n"
            report += "- À medida que o tamanho em bits aumenta, nota-se:\n"
            
            # Ordenar resultados por tamanho de bits para comparação
            sorted_results = sorted([r for r in results["results"] if r["status"] == "Sucesso"], 
                                    key=lambda x: x["bits"])
            
            if len(sorted_results) >= 2:
                smallest = sorted_results[0]
                largest = sorted_results[-1]
                report += f"  - Para {smallest['bits']} bits: {smallest['attempts']} tentativas, {smallest['time_ms']:.2f} ms\n"
                report += f"  - Para {largest['bits']} bits: {largest['attempts']} tentativas, {largest['time_ms']:.2f} ms\n"
                
                attempt_increase = largest['attempts'] / smallest['attempts'] if smallest['attempts'] > 0 else 0
                time_increase = largest['time_ms'] / smallest['time_ms'] if smallest['time_ms'] > 0 else 0
                
                report += f"  - Aumento de tentativas: {attempt_increase:.2f}x\n"
                report += f"  - Aumento de tempo: {time_increase:.2f}x\n\n"
    
    # Analisa desempenho por tamanho
    report += "### Dificuldades encontradas\n\n"
    
    failed_bits = [r["bits"] for r in results["results"] if r["status"] == "Falha"]
    if failed_bits:
        report += f"- Não foi possível gerar números primos para os seguintes tamanhos: {', '.join(str(b) for b in failed_bits)} bits\n"
    
    # Salva o relatório
    filename = "miller_rabin_report.md"
    with open(filename, "w") as f:
        f.write(report)
    
    print(f"\nRelatório completo gerado com sucesso: {filename}")
\end{verbatim}


    \chapter{Repositório do Código-Fonte}
\label{apx:code-repository}

O código-fonte completo para todos os algoritmos, experimentos e análises apresentados neste trabalho está disponível no seguinte repositório GitHub:

\begin{center}
    \url{https://github.com/username/repository/commit/commit-hash}
\end{center}

Este repositório contém a implementação dos geradores de números pseudoaleatórios (Xorshift e Blum Blum Shub) e dos testes de primalidade (Fermat e Miller-Rabin) como descritos nesta tese, juntamente com o código utilizado para realizar todas as medições e análises de desempenho.


    
    % Referências bibliográficas
    \printbibliography[title=\lang{REFERENCES}{REFERÊNCIAS}]

\end{document}