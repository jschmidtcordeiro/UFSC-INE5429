\chapter{Conclusão}

Este trabalho propôs-se a explorar, implementar e analisar experimentalmente algoritmos para geração de números pseudoaleatórios (PRNGs) e testes de primalidade, buscando compreender os compromissos entre desempenho, segurança e confiabilidade em contextos criptográficos. A seleção específica dos algoritmos Xorshift e Blum Blum Shub para geração de números aleatórios, e dos testes de Fermat e Miller-Rabin para verificação de primalidade, permitiu examinar tanto abordagens otimizadas para eficiência quanto aquelas projetadas com foco principal em segurança.

Os resultados experimentais demonstraram contrastes significativos entre os algoritmos estudados. Na geração de números pseudoaleatórios, observamos que o Xorshift supera drasticamente o Blum Blum Shub em termos de desempenho, sendo aproximadamente 244 vezes mais rápido em média. Esta discrepância é explicada pela diferença fundamental em suas complexidades computacionais: o Xorshift opera com complexidade O(1), realizando apenas operações bit a bit independentemente do tamanho do número gerado, enquanto o BBS apresenta complexidade O(k log n), exigindo custosas operações de quadrado modular para cada bit gerado. No entanto, esta vantagem de desempenho do Xorshift vem com o custo de menor segurança criptográfica, já que o algoritmo é determinístico e previsível quando seu estado interno é conhecido. Em contrapartida, o BBS oferece segurança criptográfica formalmente provada, baseada na dificuldade do problema de fatoração de inteiros, tornando-o adequado para aplicações onde a imprevisibilidade é crítica, apesar de seu desempenho inferior.

De forma similar, a análise dos testes de primalidade revelou um padrão de compromisso entre eficiência e confiabilidade. O teste de Fermat demonstrou ser computacionalmente mais eficiente para números muito grandes, apresentando tempo médio de 19.809,42 ms contra 59.090,57 ms do Miller-Rabin nos experimentos realizados. Contudo, sua suscetibilidade a falsos positivos, particularmente com os números de Carmichael, compromete sua aplicabilidade em sistemas criptográficos onde a certeza da primalidade é essencial. Por outro lado, o teste de Miller-Rabin, apesar de exigir maior esforço computacional, oferece garantias probabilísticas robustas, com probabilidade de erro máxima de $4^{-k}$, onde k é o número de iterações. Esta característica, aliada à ausência de vulnerabilidades a casos especiais como os números de Carmichael, torna o Miller-Rabin o padrão preferido para aplicações criptográficas modernas.

Os experimentos também evidenciaram como o aumento do tamanho dos números afeta drasticamente o desempenho dos algoritmos. Para o Miller-Rabin, observamos um aumento de aproximadamente 20 milhões de vezes no tempo de processamento ao passar de números de 40 bits para 4096 bits, enquanto o teste de Fermat apresentou um aumento de cerca de 255 mil vezes. Esta escalabilidade é particularmente relevante no contexto da criptografia moderna, onde números primos de 2048 a 4096 bits são frequentemente necessários.

Em síntese, este trabalho demonstra empiricamente como a escolha de algoritmos para geração de números pseudoaleatórios e verificação de primalidade envolve necessariamente um equilíbrio entre eficiência computacional e robustez criptográfica. Para aplicações onde o desempenho é prioritário e a segurança não é crítica, algoritmos como o Xorshift e o teste de Fermat podem ser adequados. No entanto, para sistemas criptográficos que exigem garantias de segurança, o custo computacional adicional do Blum Blum Shub e do teste de Miller-Rabin representa um investimento necessário.

Como direcionamento para trabalhos futuros, seria valioso explorar otimizações específicas para o Blum Blum Shub que possam reduzir sua desvantagem de desempenho sem comprometer sua segurança, bem como investigar implementações paralelas do teste de Miller-Rabin que possam acelerar a verificação de primalidade para números muito grandes. Adicionalmente, a análise de outros algoritmos de geração de números pseudoaleatórios criptograficamente seguros, como ChaCha20 ou HMAC-DRBG, poderia complementar este estudo comparativo, oferecendo perspectivas adicionais sobre o espectro de opções disponíveis para sistemas de segurança computacional.