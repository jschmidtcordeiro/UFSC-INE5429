\chapter{Números Aleatórios}\label{chap:numeros-aleatorios}

\section{Xorshift}

\subsection{Descrição}

O algoritmo Xorshift é uma classe de geradores de números pseudoaleatórios (PRNGs) desenvolvida pelo matemático George Marsaglia e publicada em 2003 no Journal of Statistical Software \cite{marsaglia2003xorshift}. O Xorshift pertence à família dos registradores de deslocamento com retroalimentação linear (LFSRs) \cite{brent2004note}, mas foi projetado para permitir implementações particularmente eficientes em software.

O princípio fundamental do Xorshift é a geração de números pseudoaleatórios através de uma sequência de operações bit a bit simples: o XOR (ou exclusivo) de um número com uma versão deslocada de si mesmo. Este processo é repetido algumas vezes em cada ciclo de geração.

A fórmula básica do algoritmo pode ser expressa da seguinte forma \cite{marsaglia2003xorshift}:
\begin{enumerate}
    \item Inicia-se com um valor de estado inicial não-nulo $x$
    \item Aplica-se $x \oplus= x \ll a$ (desloca $x$ à esquerda por $a$ bits e faz XOR com o valor original)
    \item Aplica-se $x \oplus= x \gg b$ (desloca $x$ à direita por $b$ bits e faz XOR com o valor resultante)
    \item Aplica-se $x \oplus= x \ll c$ (desloca $x$ à esquerda por $c$ bits e faz XOR com o valor resultante)
    \item O resultado é o novo valor de $x$, que também se torna o estado para a próxima iteração
\end{enumerate}

Os parâmetros $a$, $b$ e $c$ devem ser escolhidos cuidadosamente para garantir que o algoritmo atinja seu período máximo $(2^n-1)$ \cite{marsaglia2003xorshift, brent2004note}, onde $n$ é o tamanho do estado em bits. Existem versões de 32 bits, 64 bits e 128 bits, cada uma com períodos diferentes.

Uma característica notável do Xorshift é sua extrema eficiência em implementações de software modernas, exigindo apenas algumas operações de bits por número gerado \cite{marsaglia2003xorshift}. Isso o torna um dos PRNGs mais rápidos disponíveis, requerindo código mínimo e pouco estado interno.

Apesar de sua simplicidade, as versões básicas do Xorshift não passam em todos os testes estatísticos de aleatoriedade sem refinamentos adicionais \cite{panneton2005xorshift, vigna2016experimental}. Para melhorar a qualidade estatística, variantes como Xorshift*, Xorshift+ e Xoroshiro foram desenvolvidas \cite{vigna2016experimental, blackman2018scrambled}, adicionando operações não-lineares (como multiplicação ou adição) ao resultado final.

Na prática, o algoritmo Xorshift oferece um excelente equilíbrio entre velocidade, simplicidade de implementação e qualidade estatística suficiente para muitas aplicações \cite{vigna2016experimental, blackman2018scrambled}, embora não seja adequado para uso criptográfico devido à sua natureza determinística e previsível quando o estado interno é conhecido \cite{panneton2005xorshift}.

\subsection{Implementação}

Para este estudo, implementamos quatro variantes do algoritmo Xorshift: Xorshift32, Xorshift64, Xorshift128 e Xorshift128Plus. Cada implementação utiliza diferentes tamanhos de estado interno e parâmetros de deslocamento recomendados na literatura.

A implementação completa do algoritmo Xorshift pode ser encontrada no Apêndice \ref{apx:xorshift-impl}, incluindo todas as variantes mencionadas. O código está escrito em Python e foi projetado para máxima clareza, mantendo a eficiência característica do algoritmo original.

As principais características de nossa implementação incluem:
\begin{itemize}
    \item Suporte para diferentes tamanhos de estado (32, 64 e 128 bits)
    \item Manipulação adequada de sementes iniciais
    \item Garantia de período completo através de parâmetros de deslocamento otimizados
    \item Métodos auxiliares para geração de números de ponto flutuante no intervalo [0,1)
\end{itemize}

Adicionalmente, criamos um código de teste que avalia o desempenho de cada variante na geração de números pseudoaleatórios de diferentes tamanhos (40 a 4096 bits). Este código também está disponível no Apêndice \ref{apx:xorshift-impl}.

\subsection{Experimento}

\begin{table}[H]
\centering
\caption{Desempenho do algoritmo Xorshift para geração de números pseudoaleatórios}
\label{tab:xorshift-performance}
\begin{tabular}{|l|c|c|}
\hline
\textbf{Algoritmo} & \textbf{Tamanho do Número} & \textbf{Tempo para gerar (ms)} \\
\hline
Xorshift32 & 40 bits & 0,0011 ms \\
Xorshift32 & 56 bits & 0,0009 ms \\
Xorshift32 & 80 bits & 0,0013 ms \\
Xorshift32 & 128 bits & 0,0016 ms \\
Xorshift32 & 168 bits & 0,0024 ms \\
Xorshift32 & 224 bits & 0,0027 ms \\
Xorshift32 & 256 bits & 0,0031 ms \\
Xorshift32 & 512 bits & 0,0059 ms \\
Xorshift32 & 1024 bits & 0,0124 ms \\
Xorshift32 & 2048 bits & 0,0250 ms \\
Xorshift32 & 4096 bits & 0,0547 ms \\
\hline
Xorshift64 & 40 bits & 0,0005 ms \\
Xorshift64 & 56 bits & 0,0005 ms \\
Xorshift64 & 80 bits & 0,0008 ms \\
Xorshift64 & 128 bits & 0,0008 ms \\
Xorshift64 & 168 bits & 0,0011 ms \\
Xorshift64 & 224 bits & 0,0015 ms \\
Xorshift64 & 256 bits & 0,0015 ms \\
Xorshift64 & 512 bits & 0,0029 ms \\
Xorshift64 & 1024 bits & 0,0058 ms \\
Xorshift64 & 2048 bits & 0,0115 ms \\
Xorshift64 & 4096 bits & 0,0261 ms \\
\hline
Xorshift128 & 40 bits & 0,0011 ms \\
Xorshift128 & 56 bits & 0,0012 ms \\
Xorshift128 & 80 bits & 0,0016 ms \\
Xorshift128 & 128 bits & 0,0019 ms \\
Xorshift128 & 168 bits & 0,0028 ms \\
Xorshift128 & 224 bits & 0,0033 ms \\
Xorshift128 & 256 bits & 0,0040 ms \\
Xorshift128 & 512 bits & 0,0074 ms \\
Xorshift128 & 1024 bits & 0,0156 ms \\
Xorshift128 & 2048 bits & 0,0304 ms \\
Xorshift128 & 4096 bits & 0,0678 ms \\
\hline
Xorshift128Plus & 40 bits & 0,0006 ms \\
Xorshift128Plus & 56 bits & 0,0006 ms \\
Xorshift128Plus & 80 bits & 0,0010 ms \\
Xorshift128Plus & 128 bits & 0,0011 ms \\
Xorshift128Plus & 168 bits & 0,0014 ms \\
Xorshift128Plus & 224 bits & 0,0018 ms \\
Xorshift128Plus & 256 bits & 0,0017 ms \\
Xorshift128Plus & 512 bits & 0,0035 ms \\
Xorshift128Plus & 1024 bits & 0,0062 ms \\
Xorshift128Plus & 2048 bits & 0,0130 ms \\
Xorshift128Plus & 4096 bits & 0,0290 ms \\
\hline
\end{tabular}
\end{table}

\textit{Cada medição representa a média de 1000 execuções.}

\section{Blum Blum Shub}

\subsection{Descrição}

O algoritmo Blum Blum Shub (BBS) é um gerador de números pseudoaleatórios criptograficamente seguro proposto em 1986 por Lenore Blum, Manuel Blum e Michael Shub \cite{blum1986simple}. Diferentemente de algoritmos como o Xorshift, o BBS foi projetado com foco primário em segurança criptográfica, sendo fundamentado em problemas matemáticos considerados computacionalmente difíceis.

A base matemática do BBS está na teoria dos resíduos quadráticos \cite{menezes1996handbook} e na dificuldade do problema de fatoração de inteiros. O algoritmo funciona da seguinte forma \cite{blum1986simple}:

\begin{enumerate}
    \item Escolhem-se dois números primos grandes $p$ e $q$, cada um congruente a 3 módulo 4 (ou seja, $p \equiv q \equiv 3 \pmod{4}$)
    \item Calcula-se $n = p \times q$, que é chamado de módulo de Blum
    \item Seleciona-se um valor inicial (semente) $s$, tal que $s$ seja co-primo com $n$ (ou seja, $\gcd(s,n) = 1$)
    \item Gera-se a sequência de estados internos através da recorrência: $x_1 = s^2 \bmod n$, $x_2 = x_1^2 \bmod n$, $x_3 = x_2^2 \bmod n$, ...
    \item Para cada estado $x_i$, extrai-se um ou mais bits de saída, tipicamente o bit menos significativo de cada $x_i$
\end{enumerate}

A segurança do BBS deriva da dificuldade computacional de calcular raízes quadradas modulares sem conhecer os fatores de $n$ \cite{sidorenko2005concrete}. Sem o conhecimento de $p$ e $q$, prever os próximos números na sequência é equivalente a resolver o problema de resíduo quadrático, que é considerado computacionalmente intratável para valores suficientemente grandes de $n$ \cite{koblitz2015riddle, blum1986simple}.

Principais características do BBS:
\begin{itemize}
    \item É criptograficamente seguro sob a suposição de que o problema da fatoração de inteiros é difícil \cite{sidorenko2005concrete}
    \item Possui uma prova matemática formal de segurança \cite{blum1986simple}
    \item É significativamente mais lento que PRNGs não criptográficos como Xorshift \cite{vassilev2016entropy}
    \item Requer operações aritméticas de precisão arbitrária para módulos grandes \cite{menezes1996handbook}
    \item Possui período potencialmente muito longo, dependendo da escolha de $p$, $q$ e $s$ \cite{menezes1996handbook}
\end{itemize}

O BBS é principalmente utilizado em aplicações criptográficas onde a previsibilidade representa um risco de segurança \cite{vassilev2016entropy}, embora sua lentidão relativa o torne menos adequado para aplicações que demandam alta performance ou grandes volumes de números aleatórios \cite{koblitz2015riddle}.

\subsection{Implementação}

Para este estudo, implementamos o algoritmo Blum Blum Shub conforme descrito na literatura original, com suporte para geração de números de diferentes tamanhos (em bits).

A implementação completa do algoritmo Blum Blum Shub pode ser encontrada no Apêndice \ref{apx:bbs-impl}. O código foi desenvolvido em Python, utilizando a biblioteca SymPy para verificação de primalidade e cálculo de MDC, garantindo a correta geração de primos congruentes a 3 módulo 4.

As principais características de nossa implementação incluem:
\begin{itemize}
    \item Geração automática de primos adequados (congruentes a 3 módulo 4)
    \item Validação apropriada de sementes
    \item Métodos para geração de bits individuais e sequências de bits
    \item Conversão de bits gerados para inteiros em intervalos específicos
\end{itemize}

Implementamos também um código de teste que avalia o desempenho do algoritmo na geração de números pseudoaleatórios de diferentes tamanhos (40 a 4096 bits), disponível no mesmo apêndice.

\subsection{Experimento}

\begin{table}[H]
\centering
\caption{Tempo médio para geração de números pseudoaleatórios usando o algoritmo Blum Blum Shub}
\label{tab:bbs-performance}
\begin{tabular}{|l|c|c|}
\hline
\textbf{Algoritmo} & \textbf{Tamanho do Número} & \textbf{Tempo para gerar (ms)} \\
\hline
Blum Blum Shub & 40 bits & 0,1566 ms \\
Blum Blum Shub & 56 bits & 0,2212 ms \\
Blum Blum Shub & 80 bits & 0,3244 ms \\
Blum Blum Shub & 128 bits & 0,5477 ms \\
Blum Blum Shub & 168 bits & 0,6986 ms \\
Blum Blum Shub & 224 bits & 0,9175 ms \\
Blum Blum Shub & 256 bits & 1,0078 ms \\
Blum Blum Shub & 512 bits & 2,0929 ms \\
Blum Blum Shub & 1024 bits & 4,1772 ms \\
Blum Blum Shub & 2048 bits & 8,6273 ms \\
Blum Blum Shub & 4096 bits & 17,3932 ms \\
\hline
\end{tabular}
\end{table}

\textit{Cada medição representa a média de 10 execuções.}

\section{Análise}

Foi interessante perceber a grande diferença, que era esperada, entre o resultado de geração de números pseudo-aleatórios ao utilizar os algoritmos de Xorshift e de Blum Blum Shub. Em média, o algoritmo Blum Blum Shub é aproximadamente 244 vezes mais lento que o Xorshift128 (o mais lento entre os Xorshifts). Entendemos que essa lentidão é decorrente da diferença de custo computacional das operações realizadas em cada um dos algoritmos, as operações do Xorshift são muito mais simples que as do Blum Blum Shub.

Também podemos avaliar essa diferença entre os algoritmos ao analisar a complexidade de cada um. O Xorshift possui uma complexidade temporal de $O(1)$, pois independentemente do tamanho do número gerado, ele realiza apenas um número fixo de operações bit a bit (XOR e deslocamentos). Essas operações são extremamente eficientes em hardware moderno.

Por outro lado, o Blum Blum Shub possui uma complexidade temporal de $O(k \log n)$, onde $k$ é o número de bits a serem gerados e $n$ é o módulo utilizado no algoritmo. Cada bit gerado requer uma operação de quadrado modular, que é computacionalmente cara, especialmente para números grandes. Além disso, o BBS depende de operações aritméticas de precisão arbitrária, que são intrinsecamente mais lentas que as operações bit a bit utilizadas pelo Xorshift.

Esta diferença de complexidade explica claramente a discrepância de desempenho observada nos experimentos, onde o BBS se torna progressivamente mais lento à medida que o tamanho dos números aumenta.
