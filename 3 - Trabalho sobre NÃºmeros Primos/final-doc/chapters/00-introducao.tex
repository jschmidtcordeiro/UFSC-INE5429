\chapter{Introdução}

% Contextualização
A segurança computacional moderna depende fortemente de conceitos matemáticos fundamentais, entre os quais se destacam a geração de números pseudo-aleatórios e a identificação de números primos. Números primos, especialmente os de grande magnitude (como os de 2048 bits exigidos para certas aplicações criptográficas no Brasil), são a espinha dorsal de muitos algoritmos criptográficos de chave pública, como o RSA~\cite{menezes1996handbook_miller}. A geração eficiente e confiável desses grandes primos, no entanto, não é uma tarefa trivial~\cite{crandall2005prime}. O processo geralmente envolve duas etapas críticas: primeiro, a geração de um número candidato, que deve possuir boas propriedades de aleatoriedade para evitar previsibilidade~\cite{menezes1996handbook}; segundo, a aplicação de testes rigorosos para verificar a primalidade desse candidato com alta probabilidade~\cite{yan2009primality}. A qualidade dos números pseudo-aleatórios gerados e a eficácia dos testes de primalidade aplicados são, portanto, cruciais para a robustez dos sistemas de segurança que deles dependem.

% Problematização e Objetivo
Diante da variedade de algoritmos disponíveis tanto para a geração de números pseudo-aleatórios (PRNGs) quanto para testes de primalidade, surge a questão de como selecionar as técnicas mais adequadas para diferentes necessidades, considerando os compromissos inerentes entre desempenho (velocidade), segurança (imprevisibilidade e robustez criptográfica) e confiabilidade (probabilidade de erro em testes probabilísticos)~\cite{vassilev2016entropy}. Este trabalho tem como objetivo principal explorar, implementar e analisar experimentalmente um conjunto selecionado de algoritmos para ambas as tarefas. Especificamente, investigaremos os PRNGs \textit{Xorshift}~\cite{marsaglia2003xorshift} e \textit{Blum Blum Shub}~\cite{blum1986simple}, comparando sua eficiência na geração de números de diferentes ordens de grandeza. Adicionalmente, implementaremos e avaliaremos os testes de primalidade de \textit{Fermat}~\cite{cohen1993course} e \textit{Miller-Rabin}~\cite{rabin1980probabilistic,miller1976riemann}, analisando seu desempenho e confiabilidade na identificação de números primos grandes. Buscamos, através desta análise comparativa, compreender as características de desempenho, as complexidades computacionais e as implicações práticas de cada algoritmo estudado.

% Estrutura do Trabalho
Este documento está estruturado da seguinte forma: O Capítulo \ref{chap:numeros-aleatorios} aborda a geração de números pseudo-aleatórios, detalhando os algoritmos Xorshift e Blum Blum Shub, apresentando suas implementações, os resultados experimentais obtidos e uma análise comparativa de seu desempenho e complexidade. O Capítulo \ref{chap:numeros-primos} foca nos testes de primalidade, descrevendo os testes de Fermat e Miller-Rabin, suas implementações, os experimentos realizados para gerar números primos de diversos tamanhos e uma análise comparativa de sua eficiência e limitações. Seguem-se as conclusões do trabalho e as referências bibliográficas consultadas. O código fonte das implementações encontra-se disponível nos apêndices.