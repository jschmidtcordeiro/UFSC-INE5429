A análise do sistema de clusters sob a perspectiva dos princípios éticos da ACM revela diversos pontos de alinhamento e também áreas que merecem atenção especial. O sistema demonstra conformidade com vários imperativos morais gerais estabelecidos pela ACM, particularmente no que diz respeito à proteção da privacidade e ao compromisso com a confiabilidade dos serviços.

A implementação de controles de acesso e mecanismos de criptografia reflete o imperativo de "evitar causar danos", protegendo ativamente os dados dos usuários contra acessos não autorizados. O sistema de registro de alterações e a utilização de IaC demonstram o compromisso com a honestidade e confiabilidade, permitindo rastreabilidade e transparência nas operações realizadas.

No âmbito das responsabilidades profissionais específicas, o sistema incorpora práticas que promovem a qualidade e competência profissional. A estrutura de monitoramento de recursos e os mecanismos de detecção de uso indevido evidenciam o compromisso com o acesso responsável aos recursos computacionais. Como destacam \citeauthor{adams1999users}, "o acesso responsável aos recursos computacionais não é apenas uma questão técnica, mas uma responsabilidade ética que requer equilíbrio entre segurança e usabilidade, garantindo que os usuários possam realizar suas tarefas de forma eficiente e segura" \cite{adams1999users}. Além disso, a documentação das políticas de segurança e os acordos de confidencialidade demonstram respeito aos contratos e à legislação vigente.

Contudo, existem aspectos que merecem atenção para maior alinhamento com os princípios da ACM. A existência de um usuário admin com amplos poderes, embora justificada por razões operacionais, pode conflitar com o princípio de minimização de riscos. Da mesma forma, a impossibilidade de criptografar todos os tipos de dados representa um desafio para a completa conformidade com o princípio de proteção da privacidade.