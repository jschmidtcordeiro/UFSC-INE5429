No contexto do sistema de clusters analisado, a responsabilidade social e profissional assume um papel fundamental, especialmente considerando a natureza sensível dos dados gerenciados e o impacto direto na continuidade dos serviços prestados aos clientes. Como profissionais de TI responsáveis por este sistema, temos a obrigação ética de garantir não apenas a segurança técnica, mas também a proteção dos direitos e da privacidade dos usuários cujos dados são processados em nossa infraestrutura. Temos quase 500 aplicações que dependem deste sistema; caso ele falhe, muitas dessas aplicações também falharão, e isso impactará a vida das pessoas e organizações que confiam no sistema.

A implementação de mecanismos de criptografia para proteção dos dados sensíveis demonstra uma preocupação ativa com a privacidade dos usuários. No entanto, a existência de dados não criptografados e o acesso privilegiado dos administradores do cluster representam uma responsabilidade social significativa. É crucial reconhecer que cada decisão técnica tomada em relação ao acesso e proteção desses dados tem implicações diretas nos usuários da nossa organização. Desse modo, os profissionais precisam ter o devido treinamento ético para estar em concordância com os valores morais da sociedade, e um modo de transmitir esses conhecimentos dentro da organização pode ser através da difusão da cultura da empresa.

A responsabilidade profissional também se manifesta na gestão da disponibilidade e continuidade das aplicações. A adoção de Infrastructure as Code (IaC) e o registro detalhado de alterações demonstram um compromisso com a transparência e accountability, o que colabora para deixar mais claro aos demais membros da equipe e da organização o que está sendo feito. Contudo, a existência de alguns usuários admin com amplos poderes evidencia um dilema ético entre praticidade operacional e segurança, exigindo uma avaliação constante do equilíbrio entre eficiência e responsabilidade. É interessante ressaltar que, mesmo existindo alguns usuários admin, as informações dos clientes são criptografadas, podendo somente serem identificadas pelo próprio cliente final.

Como profissionais de TI, entendo que nossa responsabilidade se estende além da mera execução técnica, abrangendo a necessidade de educar e conscientizar todos os envolvidos sobre as implicações éticas de suas ações. Os acordos de confidencialidade e políticas de segurança implementados são exemplos práticos dessa responsabilidade, estabelecendo não apenas barreiras legais, mas também promovendo uma cultura de responsabilidade compartilhada e consciência ética. Como destaca \citeauthor{mason1986ethical}, "a responsabilidade ética dos profissionais de tecnologia da informação transcende a mera conformidade com regras e regulamentos, exigindo uma compreensão profunda das implicações morais de suas decisões e ações" \cite{mason1986ethical}.