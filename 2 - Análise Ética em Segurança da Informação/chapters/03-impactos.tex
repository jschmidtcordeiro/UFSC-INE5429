As falhas de segurança no sistema de clusters podem gerar impactos éticos significativos, com consequências que transcendem o âmbito técnico e afetam diretamente a vida das pessoas e organizações envolvidas. A violação de privacidade representa um dos impactos mais graves, considerando que o sistema processa dados sensíveis dos clientes. Uma falha que resulte em vazamento de dados pode comprometer informações pessoais, financeiras ou comerciais, gerando prejuízos materiais e morais para os afetados.

O abuso de dados sensíveis por parte de administradores ou desenvolvedores do sistema configura uma violação ética grave, pois representa uma quebra de confiança e pode resultar em discriminação ou manipulação indevida. Por exemplo, o acesso não autorizado a dados de clientes pode ser utilizado para vantagem competitiva ou para prejudicar determinados grupos ou indivíduos. Este tipo de violação não apenas fere os princípios éticos da ACM, mas também pode configurar crimes digitais conforme a \cite{lei12737}.

A interrupção dos serviços devido a falhas de segurança também apresenta implicações éticas significativas. A indisponibilidade do sistema pode afetar a continuidade dos negócios dos clientes, gerando prejuízos financeiros e danos à reputação. Como a maioria dos nossos produtos está relacionada à gestão pecuária, esses danos podem resultar em consequências graves para um grande número de animais. Neste contexto, a responsabilidade legal se estende além da esfera civil, podendo envolver questões trabalhistas e regulatórias, especialmente considerando a \cite{lgpd} e a legislação de proteção animal, conforme estabelecido na \cite{lei9605} e na \cite{lei14064}, que tratam especificamente da proteção e bem-estar animal.

A replicação não autorizada da infraestrutura por adversários internos representa outro aspecto ético crítico. Além das implicações legais relacionadas à propriedade intelectual, tal ação pode resultar em concorrência desleal e comprometer a sustentabilidade do negócio. A responsabilidade profissional neste caso se estende à necessidade de implementar controles adequados e promover uma cultura organizacional que valorize a ética e a integridade.