%https://github.com/SublimeText/LaTeXTools/issues/1439
%!TEX output_directory=build

% Allows you to write your thesis both in English and Portuguese
% https://tex.stackexchange.com/questions/5076/is-it-possible-to-keep-my-translation-together-with-original-text
\newif\ifenglish\englishfalse
\newif\ifadvisor\advisorfalse

% Uncomment the line `\englishtrue` to set the document default language to English
% \englishtrue
\advisortrue

% https://tex.stackexchange.com/questions/131002/how-to-expand-ifthenelse-so-that-it-can-be-used-in-parshape
\newcommand{\lang}[2]{\ifenglish#1\else#2\fi}
\newcommand{\advisor}[2]{\ifadvisor#1\else#2\fi}

% https://tex.stackexchange.com/questions/385895/how-to-make-passoptionstopackage-add-the-option-as-the-last
% https://tex.stackexchange.com/questions/484400/changing-the-cleveref-package-language-conjunction-definition
% https://tex.stackexchange.com/questions/516058/why-isnt-my-biblatex-language-changing-when-passing-the-language-on-my-document
\ifenglish
    \PassOptionsToPackage{brazil,main=english,spanish,french}{babel}
\else
    \PassOptionsToPackage{main=brazil,english,spanish,french}{babel}
\fi

% Simple alias for English and Portuguese words
% https://tex.stackexchange.com/questions/513019/argument-of-bbltempd-has-an-extra
\newcommand{\brazilword}[1]{\protect\foreignlanguage{brazil}{#1}}
\newcommand{\englishword}[1]{\protect\foreignlanguage{english}{#1}}

% Allow you to write `Evandro's house` in latex as `Evandro\s house` instead of `Evandro\textquotesingle{}s house`
% https://tex.stackexchange.com/questions/31091/space-after-latex-commands
\newcommand{\s}[0]{\textquotesingle{}s\xspace}
\newcommand{\q}[0]{\textquotesingle{}\xspace}

% Uncomment the following line if you want to use other biblatex settings
% \PassOptionsToPackage{style=numeric,repeatfields=true,backend=biber,backref=true,citecounter=true}{biblatex}
\documentclass[
\lang{english}{brazilian,brazil}, % https://tex.stackexchange.com/questions/484400/changing-the-cleveref-package-language-conjunction-definition
12pt, % Padrão UFSC para versão final
a4paper, % Padrão UFSC para versão final
oneside, % Impressão nos dois lados da folha
chapter=TITLE, % Título de capítulos em caixa alta
section=TITLE, % Título de seções em caixa alta
]{setup/ufscthesisx}

% Utilize o arquivo aftertext/references.bib para incluir sua bibliografia.
% http://tug.ctan.org/tex-archive/macros/latex/contrib/cleveref/cleveref.pdf
\addbibresource{aftertext/references.bib}

% https://www.overleaf.com/learn/latex/Inserting_Images
\graphicspath{{pictures/}}

% ======================= DADOS DO AUTOR E DO TRABALHO =======================
% Preencha com seus dados
\autor{\brazilword{João Pedro Schmidt Cordeiro}}
\titulo{\lang{Work Title}{Título do trabalho}}

% Se houver subtítulo, descomente a linha abaixo
% \subtitulo{\lang{Subtitle}{Subtítulo}}

% Siglas para grau de formação Dr./Dra., Me./Ma, Bel. Bela. (inglês: PhD., MSc., Bs.)
\orientador[\lang{Supervisor}{Orientador(a)}]{\brazilword{Nome do Orientador(a)}, \lang{Phd.}{Dr.}}

% Se houver coorientador, descomente a linha abaixo
% \coorientador[\lang{Co-supervisor}{Coorientador(a)}]{\brazilword{Nome do coorientador(a)}, \lang{Phd.}{Dr.}}

% Preencher com o nome do Coordenador de TCCs/Teses do seu curso
\coordenador[\lang{Coordinator}{Coordenador(a)}]{\brazilword{Nome do Coordenador(a)}, \lang{Phd.}{Dr.}}

% Local da sua defesa
\local{\brazilword{Florianópolis, Santa Catarina} -- \lang{Brazil}{Brasil}}

% Ano da sua defesa
\ano{2025}
\biblioteca{\lang{University Library}{Biblioteca Universitária}}

% Sigla da sua instituição
\instituicaosigla{UFSC}
\instituicao{\brazilword{Universidade Federal de Santa Catarina}}

% Preencha com Tese, Dissertação, Monografia ou Trabalho de Conclusão de Curso, Bachelor's Thesis, etc
\tipotrabalho{\lang{Bachelor\s Thesis}{Trabalho de Conclusão de Curso}}

% Se houver Área de Concentração, descomente a linha abaixo
% \area{\lang{Information Security}{Segurança da Informação}}

% Preencha com Doutor, Bacharel ou Mestrando
\formacao{\lang
    {Bachelor of Science degree in Computer Science}
    {Bacharel em Ciências da Computação}%
}
\programa{\lang
    {Undergraduate Program in Computer Science}
    {Programa de Graduação em Ciências da Computação}%
}

% Preencha com Departamento de XXXXXX, Centro de XXXXXX
\centro{\lang
    {INE -- Department of Informatics and Statistics, CTC -- Technological Center}
    {INE -- Departamento de Informática e Estatística, CTC -- Centro Tecnológico}%
}

% Preencha com Campus XXXXXX     ou     Centro de XXXXXX
\campus{\brazilword{Campus Reitor João David Ferreira Lima}}

% Data da sua defesa
\data{\lang{30 of march of}{30 de março de} 2025}

% O preambulo deve conter tipo do trabalho, objetivo, nome da instituição e a área de concentração.
\preambulo{\lang%
    {%
        \imprimirtipotrabalho~submitted to the \imprimirprograma~of
        \imprimirinstituicao~for degree acquirement in \imprimirformacao.%
    }{%
        \imprimirtipotrabalho~submetido ao \imprimirprograma~da
        \imprimirinstituicao~para a obtenção do Grau de \imprimirformacao.%
    }%
}

% Allows you to use ~= instead of `\hyp{}`
% https://tex.stackexchange.com/questions/488008/how-to-create-an-alternative-to-shortcut-or-hyp
% https://tex.stackexchange.com/questions/405718/depending-on-babel-language-setting-i-get-biblatex-error-argument-of-language
% https://tex.stackexchange.com/questions/340661/argument-of-languageactivearg-has-an-extra-i-use-includegraphics-and-r
\useshorthands{~}\defineshorthand{~=}{\hyp{}}

% ======================= PALAVRAS-CHAVE =======================
% Adicione suas palavras-chave para o documento
\palavraschaveufsc{palavraschaveingles}   {Information Security}
\palavraschaveufsc{palavraschaveportugues}{Segurança da Informação}

\palavraschaveufsc{palavraschaveingles}   {Cryptography}
\palavraschaveufsc{palavraschaveportugues}{Criptografia}

\palavraschaveufsc{palavraschaveingles}   {Ethics}
\palavraschaveufsc{palavraschaveportugues}{Ética}

\hypersetup
{
    pdfsubject={Document Abstract},
    pdfcreator={LaTeX with abnTeX2 for UFSC},
    pdftitle={\imprimirtitulo},
    pdfauthor={\imprimirautor},
    pdfkeywords={\lang{\palavraschaveinglessemitem}{\palavraschaveportuguessemitem}},
}

% Altere o arquivo 'settings.tex' para incluir customizações de aparência da sua tese
%----------------------------------------------------------------------------------------
%   Thesis Tweaks and Utilities
%----------------------------------------------------------------------------------------

% Add the lipsum package for generating dummy text
\usepackage{lipsum}

% Add the float package to use [H] placement specifier for tables and figures
\usepackage{float}

\makeatletter


% Uncomment this if you are debugging pages' badness Underfull & Overflow
% https://tex.stackexchange.com/questions/115908/geometry-showframe-landscape
% https://tex.stackexchange.com/questions/387077/what-is-the-difference-between-usepackageshowframe-and-usepackageshowframe
% https://tex.stackexchange.com/questions/387257/how-to-do-the-memoir-headings-fix-but-not-have-my-text-going-over-the-page-botto
% https://tex.stackexchange.com/questions/14508/print-page-margins-of-a-document
% \usepackage[showframe,pass]{geometry}

% To use the font Times New Roman, instead of the default LaTeX font
% more up-to-date than '\usepackage{mathptmx}'
% \usepackage{newtxtext}
% \usepackage{newtxmath}

% https://tex.stackexchange.com/questions/182569/how-to-manually-set-where-a-word-is-split
\hyphenation{Ge-la-im}
\hyphenation{Cis-la-ghi}

% Add missing translations for Portuguese
% https://tex.stackexchange.com/questions/8564/what-is-the-right-way-to-redefine-macros-defined-by-babel
\@ifpackageloaded{babel}{\@ifpackagewith{babel}{brazil}{\addto\captionsbrazil{%
  \renewcommand{\mytextpreliminarylistname}{Breve Sumário}
}}{}}{}
\@ifundefined{advisor}{\newcommand{\advisor}[2]{#1}}{}

% Selects a sans serif font family
% \renewcommand{\sfdefault}{cmss}

% Selects a monospaced (“typewriter”) font family
% \renewcommand{\ttdefault}{cmtt}

% Spacing between lines and paragraphs
% https://tex.stackexchange.com/questions/70212/ifpackageloaded-question
\@ifclassloaded{memoir}
{
  % New custom chapter style VZ14, see other chapters styles in:
  % http://repositorios.cpai.unb.br/ctan/info/latex-samples/MemoirChapStyles/MemoirChapStyles.pdf
  \newcommand\thickhrulefill{\leavevmode \leaders \hrule height 1ex \hfill \kern \z@}
  \makechapterstyle{VZ14} { %
    % \thispagestyle{empty}
    \setlength\beforechapskip{50pt}
    \setlength\midchapskip{20pt}
    \setlength\afterchapskip{20pt}
    \renewcommand\chapternamenum{}
    \renewcommand\printchaptername{}
    \renewcommand\chapnamefont{\Huge\scshape}
    \renewcommand\printchapternum {%
      \chapnamefont\null\thickhrulefill\quad
      \@chapapp\space\thechapter\quad\thickhrulefill
    }
    \renewcommand\printchapternonum {%
      \par\thickhrulefill\par\vskip\midchapskip
      \hrule\vskip\midchapskip
    }
    \renewcommand\chaptitlefont{\huge\scshape\centering}
    \renewcommand\afterchapternum {%
      \par\nobreak\vskip\midchapskip\hrule\vskip\midchapskip
    }
    \renewcommand\afterchaptertitle {%
      \par\vskip\midchapskip\hrule\nobreak\vskip\afterchapskip
    }
  }

  % Apply the style `VZ14` just created
  % \chapterstyle{VZ14}

  % http://mirrors.ibiblio.org/CTAN/macros/latex/contrib/memoir/memman.pdf
  \setlength\beforechapskip{0pt}
  \setlength\midchapskip{15pt}
  \setlength\afterchapskip{15pt}

  % Memoir: Warnings "The material used in the headers is too large" w/ accented titles
  % https://tex.stackexchange.com/questions/387293/how-to-change-the-page-layout-with-memoir
  \setheadfoot{30.0pt}{\footskip}
  \checkandfixthelayout
}{}

% Controlling the spacing between one paragraph and another
% Default value for UFSC 0.0cm
\setlength{\parskip}{\advisor{0.0cm}{0.2cm}}

% Paragraph size is given by
% Default value for UFSC 1.5cm
% \setlength{\parindent}{1.3cm}

% https://tex.stackexchange.com/questions/148647/how-to-remove-space-before-enumerate
% https://tex.stackexchange.com/questions/433543/behaviour-of-enumitem-setlist
\advisor{}{
    \setlist*[enumerate]{label=\arabic*,}
    \setlist*[enumerateoptional]{label=\arabic*,}

    % https://tex.stackexchange.com/questions/24454/space-after-float-with-h
    % https://tex.stackexchange.com/questions/23313/how-can-i-reduce-padding-after-figure
    \AtBeginEnvironment{figure}{
      \setlength{\intextsep}{5pt} % Vertical space above & below [h] floats
      % \setlength{\textfloatsep}{10pt} % Vertical space below (above) [t] ([b]) floats
      % \setlength{\abovecaptionskip}{10pt}
      % \setlength{\belowcaptionskip}{5pt}
    }

    % Patch the `abntex2` citacao environment removing the extra space from its top
    % https://tex.stackexchange.com/questions/300340/topsep-itemsep-partopsep-and-parsep-what-does-each-of-them-mean-and-wha
    \xpatchcmd{\citacao}
    {\list{}}
    {\list{}{\topsep=0pt}}
    {}
    {\FAILEDPATCHINGCITACAO}
}


% Color settings across the document
\@ifpackageloaded{xcolor}
{
  % RGB colors in absolute values from 0 to 255 by using `RGB` tag
  \definecolor{darkblue}{RGB}{26,13,178}

  % Colors names definitions as RGB colors in percentage notation by using `rgb` tag
  \definecolor{mygreen}{rgb}{0,0.6,0}
  \definecolor{mygray}{rgb}{0.5,0.5,0.5}
  \definecolor{mymauve}{rgb}{0.58,0,0.82}
  \definecolor{figcolor}{rgb}{1,0.4,0}
  \definecolor{tabcolor}{rgb}{1,0.4,0}
  \definecolor{eqncolor}{rgb}{1,0.4,0}
  \definecolor{linkcolor}{rgb}{1,0.4,0}
  \definecolor{citecolor}{rgb}{1,0.4,0}
  \definecolor{seccolor}{rgb}{0,0,1}
  \definecolor{abscolor}{rgb}{0,0,1}
  \definecolor{titlecolor}{rgb}{0,0,1}
  \definecolor{biocolor}{rgb}{0,0,1}
  \definecolor{blue}{RGB}{41,5,195}

  % PDF Hyperlinks settings
  \@ifpackageloaded{hyperref}
  {
    \hypersetup
    {
      colorlinks=true,     % false: boxed links; true: colored links
      linkcolor=darkblue,  % color of internal links
      citecolor=darkblue, % color of links to bibliography
      filecolor=black,     % color of file links
      urlcolor=\advisor{black}{darkgreen},
      bookmarksdepth=4,
      pdfencoding=auto,%
      psdextra,
    }
  }
}{}


% Filtering and Mapping Bibliographies
% \DeclareFieldFormat{url}{Disponível~em:\addspace\url{#1}}

% https://tex.stackexchange.com/questions/517526/how-to-make-biblatex-url-links-generated-with-brackets-around-it-url-correctly
\DeclareFieldFormat{url}{\bibstring{urlfrom}\addcolon\space\textless\url{#1}\textgreater}
\DefineBibliographyStrings{brazil}{urlfrom = {Disponível em}}
\DefineBibliographyStrings{english}{urlfrom = {Available from}}

% https://tex.stackexchange.com/questions/391695/is-possible-to-remove-the-link-color-of-the-comma-on-the-citation-link
% \DeclareFieldFormat{citehyperref}{#1}

% % https://tex.stackexchange.com/questions/203764/reduce-font-size-of-bibliography-overfull-bibliography
% \newcommand{\bibliographyfontsize}{\fontsize{10.0pt}{10.5pt}\selectfont}
% \renewcommand*{\bibfont}{\bibliographyfontsize}

% Uncomment this to insert the abstract into your bibliography entries when the abstract is available
% https://tex.stackexchange.com/questions/398666/how-to-correctly-insert-and-justify-abstract
\ifadvisor
\else
  \DeclareFieldFormat{abstract}%
  {%
    \par\justifying
    \begin{adjustwidth}{1cm}{}
      \textbf{\bibsentence\bibstring{abstract}:} #1
    \end{adjustwidth}
  }
  \renewbibmacro*{finentry}%
  {%
    \iffieldundef{abstract}
    {\finentry}
    {\finentrypunct
      \printfield{abstract}%
      \renewcommand*{\finentrypunct}{}%
      \finentry
    }
  }

  % Backref package settings, pages with citations in bibliography
  \newcommand{\biblatexcitedntimes}{\autocap{c}ited \arabic{citecounter} times}
  \newcommand{\biblatexcitedonetime}{\autocap{c}ited one time}
  \newcommand{\biblatexcitednotimes}{\autocap{n}o citation in the text}

  \@ifpackageloaded{babel}{\@ifpackagewith{babel}{brazil}{\addto\captionsbrazil{%
    \renewcommand{\biblatexcitedntimes}{\autocap{c}itado \arabic{citecounter} vezes}
    \renewcommand{\biblatexcitedonetime}{\autocap{c}itado uma vez}
    \renewcommand{\biblatexcitednotimes}{\autocap{n}enhuma citação no texto}
  }}{}}{}
  \@ifpackageloaded{biblatex}
  {%
    % https://tex.stackexchange.com/questions/483707/how-to-detect-whether-the-option-citecounter-was-enabled-on-biblatex
    \ifx\blx@citecounter\relax
      \message{Is citecounter defined? NO!^^J}
    \else
      \message{Is citecounter defined? YES!^^J}
      \ifbacktracker
        \message{Is backtracker defined? YES!^^J}
        \renewbibmacro*{pageref}
        {%
          % https://tex.stackexchange.com/questions/516054/how-to-use-a-dot-to-separate-my-new-bibliography-entry
          \renewcommand*{\bibpagerefpunct}{\addperiod\space}%
          \iflistundef{pageref}%
          {\printtext{\biblatexcitednotimes}}
          {%
            \printtext
            {%
              \ifnumgreater{\value{citecounter}}{1}
                {\biblatexcitedntimes}
                {\biblatexcitedonetime}%
            }%
            \setunit{\addspace}%
            \ifnumgreater{\value{pageref}}{1}
              {\bibstring{backrefpages}\ppspace}
              {\bibstring{backrefpage}\ppspace}%
            \printlist[pageref][-\value{listtotal}]{pageref}%
          }%
        }

        \DefineBibliographyStrings{brazil}
        {
          backrefpage  = {na página},
          backrefpages = {nas páginas},
        }

        \DefineBibliographyStrings{english}
        {
          backrefpage  = {on page},
          backrefpages = {on pages},
        }
      \else
        \message{Is backtracker defined? NO!^^J}
      \fi
    \fi
  }{}
\fi


% https://tex.stackexchange.com/questions/516056/why-an-empty-or-not-biblatex-declaresourcemap-is-removing-my-bibliography-acces
% https://github.com/abntex/biblatex-abnt/pull/56/files
\DeclareStyleSourcemap{%% >>>2
  % This maps some fields used in abntex2cite to biblatex fields.
  \maps[datatype=bibtex]{%
    \map{%
      \step[fieldsource=conference-number,fieldtarget=number]%
      \step[fieldsource=conference-year,fieldtarget=eventdate]%
      \step[fieldsource=conference-location,fieldtarget=venue]%
      \step[fieldsource=conference-number,fieldtarget=number]%
      \step[fieldsource=org-short,fieldtarget=shortauthor]%
      \step[fieldsource=urlaccessdate,fieldtarget=urldate]%
      \step[fieldsource=year-presented,fieldtarget=eventyear]%
      \step[fieldsource=furtherresp,fieldtarget=titleaddon]%
      \step[typesource=journalpart,typetarget=supperiodical]%
    }%
    \map[overwrite=false]{%
      \step[fieldsource=reprinted-from, final]%
      \step[fieldset=related, origfieldval]%
    }%
    \map[overwrite=false]{%
      \step[fieldsource=reprinted-text, final]%
      \step[fieldset=relatedtype, fieldvalue={reprintfrom}]%
    }%
    \map{%
      \pertype{patent}% Use the organization as sourcekey for patents
      \step[fieldsource=organization, final]%
      \step[fieldset=sortkey, origfieldval]%
    }%
    \map[overwrite=false]{%
      \pertype{thesis}%
      \pertype{phdthesis}%
      \pertype{mastersthesis}%
      \pertype{monography}%
      \step[fieldset=bookpagination, fieldvalue={sheet}]%
    }%
    % remove fields that are always useless
    \map{
      % \step[fieldset=abstract, null]
      \step[fieldset=pagetotal, null]
    }
    % % remove URLs for types that are primarily printed
    % \map{
    %   \pernottype{software}
    %   \pernottype{online}
    %   \pernottype{report}
    %   \pernottype{techreport}
    %   \pernottype{standard}
    %   \pernottype{manual}
    %   \pernottype{misc}
    %   \step[fieldset=url, null]
    %   \step[fieldset=urldate, null]
    % }
    \map{
      \pertype{inproceedings}
      % remove mostly redundant conference information
      \step[fieldset=venue, null]
      \step[fieldset=eventdate, null]
      \step[fieldset=eventtitle, null]
      % do not show ISBN for proceedings
      \step[fieldset=isbn, null]
      % Citavi bug
      \step[fieldset=volume, null]
    }
  }%
}% <<<2


% https://tex.stackexchange.com/questions/14314/changing-the-font-of-the-numbers-in-the-toc-in-the-memoir-class
\renewcommand{\cftpartfont}{\ABNTEXpartfont\color{black}}
\renewcommand{\cftpartpagefont}{\ABNTEXpartfont\color{black}}

\renewcommand{\cftchapterfont}{\ABNTEXchapterfont\color{black}}
\renewcommand{\cftchapterpagefont}{\ABNTEXchapterfont\color{black}}

\renewcommand{\cftsectionfont}{\ABNTEXsectionfont\color{black}}
\renewcommand{\cftsectionpagefont}{\ABNTEXsectionfont\color{black}}

\renewcommand{\cftsubsectionfont}{\ABNTEXsubsectionfont\color{black}}
\renewcommand{\cftsubsectionpagefont}{\ABNTEXsubsectionfont\color{black}}

\renewcommand{\cftsubsubsectionfont}{\ABNTEXsubsubsectionfont\color{black}}
\renewcommand{\cftsubsubsectionpagefont}{\ABNTEXsubsubsectionfont\color{black}}

\renewcommand{\cftparagraphfont}{\ABNTEXsubsubsubsectionfont\color{black}}
\renewcommand{\cftparagraphpagefont}{\ABNTEXsubsubsubsectionfont\color{black}}

% Memoir has another mechanism for the job: \cftsetindents{‹kind›}{indent}{numwidth}. Here kind is
% chapter, section, or whatever; the indent specifies the 'margin' before the entry starts; and the
% width is of the box into which the number is typeset (so needs to be wide enough for the largest
% number, with the necessary spacing to separate it from what comes after it in the line.
% http://www.tex.ac.uk/FAQ-tocloftwrong.html
% https://tex.stackexchange.com/questions/264668/memoir-indentation-of-unnumbered-sections-in-table-of-contents
% https://tex.stackexchange.com/questions/394227/memoir-toc-indent-the-second-line-by-numberspace
%
% `\cftlastnumwidth` and these `\cftsetindents` are defined by the abntex2 class,
% obeying the `ABNTEXsumario-abnt-6027-2012`. \newlength{\cftlastnumwidth}
% \setlength{\cftlastnumwidth}{\cftsubsubsectionnumwidth}
% \addtolength{\cftlastnumwidth}{-1em}

% http://www.tex.ac.uk/FAQ-tocloftwrong.html
% Use \setlength\cftsectionnumwidth{4em} to override all these values at once
\ifadvisor
\else
  \makechapterstyle{fixedabntex2indentation}
  {%
    \renewcommand{\chapterheadstart}{}
    \setlength{\beforechapskip}{20pt}
    \setlength{\midchapskip}{20pt}
    \setlength{\afterchapskip}{15pt}

    \ifx \chapternamenumlength \undefined
      \newlength{\chapternamenumlength}
    \fi

    % tamanhos de fontes de chapter e part
    \ifthenelse{\equal{\ABNTEXisarticle}{true}}{%
      \setlength\beforechapskip{\baselineskip}%
      \renewcommand{\chaptitlefont}{\ABNTEXsectionfont\ABNTEXsectionfontsize}%
    }{%else
       \setlength{\beforechapskip}{0pt}%
       \renewcommand{\chaptitlefont}{\ABNTEXchapterfont\ABNTEXchapterfontsize}%
    }

    \renewcommand{\chapnumfont}{\chaptitlefont}
    \renewcommand{\parttitlefont}{\ABNTEXpartfont\ABNTEXpartfontsize}
    \renewcommand{\partnumfont}{\ABNTEXpartfont\ABNTEXpartfontsize}
    \renewcommand{\partnamefont}{\ABNTEXpartfont\ABNTEXpartfontsize}

    % tamanhos de fontes de section, subsection, subsubsection e subsubsubsection
    \setsecheadstyle{\ABNTEXsectionfont\ABNTEXsectionfontsize\ABNTEXsectionupperifneeded}
    \setsubsecheadstyle{\ABNTEXsubsectionfont\ABNTEXsubsectionfontsize\ABNTEXsubsectionupperifneeded}
    \setsubsubsecheadstyle{\ABNTEXsubsubsectionfont\ABNTEXsubsubsectionfontsize\ABNTEXsubsubsectionupperifneeded}
    \setsubsubsubsecheadstyle{\ABNTEXsubsubsubsectionfont\ABNTEXsubsubsubsectionfontsize\ABNTEXsubsubsubsectionupperifneeded}

    % Impressão do número do capítulo
    \renewcommand{\chapternamenum}{}

    % Impressão do nome do capítulo
    \renewcommand{\printchaptername}{%
       \chaptitlefont%
       \ifthenelse{\boolean{abntex@apendiceousecao}}{\appendixname}{}%
    }

    % Impressão do título do capítulo
    \def\printchaptertitle##1{%
      \chaptitlefont%
      \ifthenelse{\boolean{abntex@innonumchapter}}{\centering\ABNTEXchapterupperifneeded{##1}}{%
      \ifthenelse{\boolean{abntex@apendiceousecao}}{%
        \centering%
        \settowidth{\chapternamenumlength}{\printchaptername\printchapternum\afterchapternum}%
        \ABNTEXchapterupperifneeded{##1}%
      }{%
        \settowidth{\chapternamenumlength}{\printchaptername\printchapternum\afterchapternum}%
        \parbox[t]{\columnwidth-\chapternamenumlength}{\ABNTEXchapterupperifneeded{##1}}}%
      }%
    }

    % https://tex.stackexchange.com/questions/264668/memoir-indentation-of-unnumbered-sections-in-table-of-contents
    \renewcommand{\tocinnonumchapter}{%
      \addtocontents{toc}{\cftsetindents{chapter}{2.5em}{2em}}%
      \cftinserthook{toc}{A}}

    % Impressão do número do capítulo (no capítulo e não toc)
    \renewcommand{\printchapternum}{%
      \setboolean{abntex@innonumchapter}{false}%
      \chapnumfont%
      ~~\thechapter~%
      \ifthenelse{\boolean{abntex@apendiceousecao}}{%
        \tocinnonumchapter%
        ~\ABNTEXcaptiondelim~~%
      }{}%
    }

    \renewcommand{\ABNTEXcaptiondelim}{~\textendash~}
    \renewcommand{\afterchapternum}{}

    % Impressão do capítulo não numerado
    \renewcommand\printchapternonum{%
      \setboolean{abntex@innonumchapter}{true}%
    }
  }
  \chapterstyle{fixedabntex2indentation}

  \cftsetindents{part}          {0em} {3em}
  \cftsetindents{chapter}       {0em} {3em}
  \cftsetindents{section}       {0em} {4.3em}
  \cftsetindents{subsection}    {0em} {5.2em}
  \cftsetindents{subsubsection} {0em} {5.1em}
  \cftsetindents{paragraph}     {0em} {6.0em}
  \cftsetindents{subparagraph}  {0em} {7.0em}
\fi


\makeatother



\begin{document}
    % FIXME: Comment this after finishing the thesis, so you can start fixing the \flushbottom vs \raggedbottom
    % https://tex.stackexchange.com/questions/65355/flushbottom-vs-raggedbottom
    \raggedbottom

    % https://tex.stackexchange.com/questions/4705/double-space-between-sentences
    \frenchspacing

    % Uncomment this to put a ←← | ← (Go To Top/Go Back) on each section header
    \advisor{}{\addGoToSummary}

    % ELEMENTOS PRÉ-TEXTUAIS
    % Capa
    \pretextual
    % https://tex.stackexchange.com/questions/227711/blank-page-after-titlingpage
    \advisor{}{\AtBeginShipoutNext{\AtBeginShipoutNext{\AtBeginShipoutDiscard}}}
    \imprimircapa

    % ======================= CONTEÚDO DO DOCUMENTO =======================
    \textual
    
    % Incluindo os capítulos do trabalho
    \chapter{Introdução}
    A segurança da informação tradicionalmente tem sido abordada como um conjunto de desafios técnicos: como proteger sistemas contra invasores, como garantir a confidencialidade, integridade e disponibilidade de dados, e como implementar medidas de proteção eficazes. No entanto, à medida que sistemas computacionais se tornam mais integrados à sociedade e assumem papéis mais críticos na mediação de aspectos fundamentais da vida humana, torna-se evidente que a segurança da informação não é apenas um domínio técnico, mas também profundamente ético.

Esta análise busca explorar as dimensões éticas dessa segurança, examinando como princípios morais fundamentais devem guiar as práticas profissionais neste campo. O estudo se concentra especificamente em um sistema de clusters empresariais, composto por quatro clusters (três de desenvolvimento e um de produção), onde aplicações são executadas em containers orquestrados via Kubernetes. Como administrador deste sistema, enfrento diariamente dilemas éticos que vão além das questões técnicas, envolvendo o acesso a dados sensíveis de clientes, a gestão de recursos computacionais e a supervisão de equipes de desenvolvimento.

O objetivo é demonstrar que uma abordagem puramente técnica à segurança da computação é insuficiente; os profissionais da área precisam incorporar considerações éticas em todas as facetas de seu trabalho, desde o desenvolvimento de sistemas até a resposta a incidentes.

A premissa central deste documento é que a ética não é um componente opcional ou secundário da segurança da computação, mas constitui sua própria essência. Como argumenta \citeauthor{spinello2013cyberethics}, "a ética da segurança da informação não é meramente sobre seguir regras ou códigos de conduta, mas sobre cultivar uma sensibilidade moral que permita aos profissionais reconhecer e responder apropriadamente às dimensões éticas de seu trabalho" \cite{spinello2013cyberethics}.

Esta análise está estruturada em quatro seções principais. Primeiramente, examinaremos a responsabilidade social e profissional inerente ao trabalho em segurança da computação. Em seguida, analisaremos como os princípios éticos estabelecidos pela Association for Computing Machinery (ACM) se aplicam especificamente aos desafios de segurança. A terceira seção explorará os impactos éticos das falhas de segurança, destacando como vulnerabilidades e brechas afetam indivíduos e sociedades além das consequências técnicas imediatas. Finalmente, abordaremos as dimensões éticas do gerenciamento de riscos em segurança da computação, analisando como decisões sobre quais riscos aceitar, mitigar ou transferir refletem valores e prioridades éticas.

Através desta análise, espera-se cultivar uma compreensão mais profunda das responsabilidades éticas dos profissionais de segurança da computação e contribuir para o desenvolvimento de práticas que não apenas protejam sistemas e dados, mas também respeitem e promovam valores humanos fundamentais. 
    
    \chapter{Responsabilidade Social e Profissional}
    Os profissionais de segurança da informação carregam uma responsabilidade que extrapola os limites técnicos de sua função. A natureza do seu trabalho, protegendo dados e sistemas críticos, implica uma série de obrigações éticas para com indivíduos, organizações e a sociedade como um todo.

\section{Responsabilidade além do Contrato Formal}

A responsabilidade profissional em segurança da informação transcende os termos contratuais e descrições de cargo. Além das obrigações formalmente estabelecidas com empregadores ou clientes, estes profissionais têm um compromisso ético com todos os indivíduos afetados por seus sistemas, mesmo que indiretamente.

Como destaca \citeauthor{burmeister2016applied}, "a responsabilidade do profissional de TI estende-se para além do círculo imediato de stakeholders até os usuários finais e indivíduos cujos dados são processados pelos sistemas" \cite{burmeister2016applied}. Esta concepção ampliada de responsabilidade é particularmente relevante em segurança da informação, onde falhas podem comprometer a privacidade e autonomia de pessoas que nunca consentiram explicitamente com o processamento de seus dados.

\section{Transparência e Accountability}

A responsabilidade ética em segurança da informação requer transparência sobre vulnerabilidades e riscos. Embora exista tensão entre a divulgação completa de vulnerabilidades e o risco de facilitar ataques, os profissionais têm a obrigação ética de comunicar riscos significativos aos stakeholders que podem ser afetados, permitindo decisões informadas.

O conceito de "divulgação responsável" (\textit{responsible disclosure}) exemplifica esta tensão ética, buscando equilibrar a necessidade de transparência com a proteção contra exploração maliciosa. Como argumenta \citeauthor{schneier2015data}, "a completa supressão de informações sobre vulnerabilidades raramente serve ao interesse público; em vez disso, protocolos éticos de divulgação que permitem a correção antes da exposição pública representam um compromisso responsável" \cite{schneier2015data}.

\section{Advocacia pelos Direitos dos Usuários}

Os profissionais de segurança da informação frequentemente servem como defensores dos direitos e interesses dos usuários dentro de organizações, especialmente quando pressões comerciais ou operacionais podem incentivar práticas que comprometem a privacidade ou segurança.

Esta responsabilidade ética de advocacia é evidenciada pela crescente importância do papel do "privacy champion" ou "embaixador de privacidade" em organizações modernas – profissionais que assumem a responsabilidade de defender considerações de privacidade e segurança durante o desenvolvimento de produtos e políticas. Como observa \citeauthor{cavoukian2009privacy}, a incorporação do conceito de "Privacy by Design" depende criticamente destes defensores internos que elevam considerações éticas ao mesmo patamar de prioridades técnicas e comerciais \cite{cavoukian2009privacy}.

\section{Desenvolvimento Profissional Contínuo}

A rápida evolução das tecnologias e ameaças em segurança da informação cria uma obrigação ética de aprendizado contínuo. Profissionais têm a responsabilidade de se manterem atualizados não apenas sobre aspectos técnicos, mas também sobre desenvolvimentos legais, sociais e éticos relacionados ao seu campo.

Como destaca o Código de Ética da ACM, os profissionais de computação devem "melhorar suas próprias competências e conhecimentos" como parte de suas responsabilidades profissionais \cite{acm2018}. Esta obrigação é particularmente aguda em segurança da informação, onde o desconhecimento de novas vulnerabilidades ou técnicas de ataque pode resultar em falhas significativas na proteção de indivíduos e organizações. 
    
    \chapter{Princípios Éticos da ACM}
    O Código de Ética e Conduta Profissional da Association for Computing Machinery (ACM) representa um conjunto abrangente de princípios éticos para profissionais de computação. Esta seção analisa como estes princípios se aplicam especificamente aos desafios da segurança da informação, oferecendo um framework para tomada de decisões éticas neste domínio.

\section{Princípio 1.1: Contribuir para o Bem-Estar Humano}

O primeiro princípio da ACM enfatiza a obrigação de "contribuir para a sociedade e o bem-estar humano, reconhecendo que todas as pessoas são stakeholders na computação" \cite{acm2018}. Na segurança da informação, este princípio se manifesta no compromisso de proteger não apenas sistemas, mas as pessoas que dependem deles.

Sistemas que implementam segurança sem considerar o bem-estar humano podem criar proteções contraproducentes que impedem funcionalidades legítimas ou impõem fardos desproporcionais aos usuários. Por exemplo, políticas de senhas excessivamente complexas frequentemente levam usuários a adotar comportamentos inseguros, como anotar senhas ou reutilizá-las entre serviços \cite{adams1999users}.

A segurança ética deve buscar o equilíbrio entre proteção e usabilidade, reconhecendo que sistemas seguros que são difíceis de usar podem falhar em sua missão fundamental de proteger pessoas. Como argumenta \citeauthor{cranor2005security}, "a segurança não deve ser implementada às custas da experiência humana, mas como um componente integral dela" \cite{cranor2005security}.

\section{Princípio 1.2: Evitar Danos}

O princípio de "evitar danos" no código da ACM tem aplicação direta em segurança da informação, onde vulnerabilidades podem resultar em consequências severas. Este princípio exige avaliação cuidadosa dos riscos potenciais de segurança e implementação de medidas proporcionais para mitigá-los.

A questão de divulgação de vulnerabilidades ilustra a complexidade deste princípio. Por um lado, a não-divulgação de vulnerabilidades conhecidas pode expor usuários a riscos; por outro, a divulgação imprudente pode facilitar ataques antes que correções estejam disponíveis. Frameworks éticos de "divulgação coordenada" buscam navegar este dilema, permitindo que organizações afetadas corrijam vulnerabilidades antes da divulgação pública \cite{householder2020coordinated}.

Este princípio também implica na obrigação de considerar consequências não intencionais das medidas de segurança. Por exemplo, tecnologias de vigilância implementadas para aumentar a segurança podem inadvertidamente criar riscos à privacidade e liberdade civil quando implementadas sem salvaguardas apropriadas.

\section{Princípio 1.3: Honestidade e Confiabilidade}

A honestidade é fundamental em segurança da informação, particularmente na comunicação sobre riscos e capacidades de segurança. Organizações frequentemente enfrentam a tentação de exagerar suas medidas de segurança ou minimizar a seriedade de incidentes, mas este princípio exige transparência mesmo quando desconfortável.

Como observa \citeauthor{friedman2008trust}, "a confiança em sistemas digitais depende criticamente da percepção de honestidade por parte dos responsáveis por estes sistemas" \cite{friedman2008trust}. Esta honestidade se manifesta em práticas como notificação oportuna após violações de dados, documentação clara sobre práticas de segurança, e comunicação precisa sobre limitações dos sistemas de segurança.

A prática controversa de "security through obscurity" – depender do segredo do funcionamento interno de um sistema para sua segurança – muitas vezes conflita com este princípio, pois pode criar uma falsa sensação de segurança enquanto dificulta avaliações independentes das proteções oferecidas.

\section{Princípio 1.4: Justiça e Não-Discriminação}

O princípio de justiça no Código da ACM enfatiza o tratamento equitativo de todos os indivíduos e a prevenção de práticas discriminatórias. Em segurança da informação, este princípio tem implicações significativas para o desenvolvimento e implementação de medidas de segurança.

Sistemas de segurança devem ser projetados para oferecer proteção equitativa a todos os usuários, independentemente de características como idioma, localização geográfica, capacidade técnica ou recursos financeiros. Como destacam \citeauthor{abebe2020roles}, "a acessibilidade de medidas de segurança deve ser considerada tão importante quanto sua eficácia técnica" \cite{abebe2020roles}.

Algoritmos de detecção de fraude e sistemas de monitoramento de segurança podem inadvertidamente incorporar vieses que resultam em flagging desproporcional de grupos marginalizados. Por exemplo, sistemas que avaliam comportamentos "anormais" como indicativos de atividade maliciosa podem penalizar injustamente padrões de uso que diferem da norma presumida, mas são legítimos.

A implementação ética de segurança da informação requer consciência destes riscos potenciais de discriminação e esforços ativos para mitigá-los através de testes inclusivos, diversidade nas equipes de desenvolvimento e auditorias regulares para identificar e corrigir padrões discriminatórios. 
    
    \chapter{Impactos Éticos das Falhas de Segurança}
    As falhas em sistemas de segurança da informação podem ter implicações éticas profundas e abrangentes, afetando indivíduos, organizações e a sociedade como um todo. A análise desses impactos é fundamental para compreender a importância da segurança não apenas como uma questão técnica, mas como um imperativo ético.

\section{Violação de Privacidade}

A violação de privacidade resultante de falhas de segurança representa uma das consequências éticas mais significativas. Quando dados pessoais são expostos indevidamente, os indivíduos perdem controle sobre informações que podem revelar aspectos íntimos de suas vidas, preferências e comportamentos. Isto pode levar a danos psicológicos, como ansiedade e perda de confiança, além de potenciais prejuízos materiais.

Em contextos específicos, como dados de saúde ou informações sobre orientação sexual, religião ou opiniões políticas, vazamentos podem ter consequências particularmente graves. Por exemplo, a exposição de registros médicos pode levar a discriminação em oportunidades de emprego ou contratação de seguro \cite{bishop2018computer}. A Lei Geral de Proteção de Dados (LGPD) e regulamentações similares internacionalmente reconhecem essa gravidade ao impor obrigações estritas para proteção de dados sensíveis.

\section{Discriminação e Exclusão Social}

Falhas de segurança podem levar a discriminação quando dados expostos são utilizados para tomar decisões injustas sobre indivíduos ou grupos. Por exemplo, se informações sobre histórico médico, situação financeira ou características demográficas são comprometidas, podem ser utilizadas para discriminar em processos de contratação, concessão de crédito ou acesso a serviços essenciais.

Sistemas biométricos comprometidos representam um risco particular, pois envolvem características imutáveis dos indivíduos. Diferentemente de senhas, atributos biométricos não podem ser alterados quando comprometidos, criando vulnerabilidades permanentes para as pessoas afetadas. Estas consequências podem amplificar desigualdades existentes e afetar desproporcionalmente grupos já marginalizados \cite{spinello2013cyberethics}.

\section{Abuso de Dados Sensíveis e Responsabilidades Legais}

O abuso de dados sensíveis após falhas de segurança pode tomar múltiplas formas: desde extorsão e chantagem até fraudes de identidade e manipulação psicológica. Dados comprometidos podem ser utilizados em ataques direcionados (spear phishing), para assumir contas digitais ou para acessar serviços financeiros fraudulentamente em nome das vítimas.

As responsabilidades legais das organizações após violações de dados tornaram-se mais significativas com a implementação de leis como a LGPD no Brasil e o GDPR na Europa. Estas regulamentações estabelecem obrigações de notificação, sanções administrativas substanciais e possibilidade de ações civis por danos. No entanto, como observa \citeauthor{hoepers2007csirt}, a responsabilidade ética transcende a conformidade legal e inclui o compromisso proativo com a segurança como um valor fundamental \cite{hoepers2007csirt}.

O impacto cumulativo de falhas recorrentes de segurança inclui a erosão da confiança pública em sistemas digitais, potencialmente prejudicando a adoção de tecnologias benéficas e a inclusão digital, o que constitui um dano social mais amplo. 
    
    \chapter{Dimensões Éticas do Gerenciamento de Riscos}
    O gerenciamento de riscos em segurança da informação não é um processo puramente técnico ou financeiro, mas fundamentalmente ético. As decisões sobre quais riscos aceitar, mitigar ou transferir refletem valores organizacionais e têm implicações diretas para a privacidade, autonomia e bem-estar dos usuários afetados.

\section{Balanceamento entre Custo-Benefício e Direitos dos Usuários}

Uma das questões éticas centrais no gerenciamento de riscos é o equilíbrio entre considerações econômicas e a proteção dos direitos fundamentais dos usuários. A abordagem puramente utilitarista, que considera apenas custos financeiros versus probabilidade e impacto de incidentes, é eticamente insuficiente quando direitos como privacidade e segurança estão em jogo.

Organizações frequentemente enfrentam decisões onde medidas de segurança adicionais representam custos significativos para proteger contra ameaças de probabilidade relativamente baixa. A perspectiva ética exige que essas decisões considerem o valor intrínseco dos direitos dos usuários, mesmo quando a análise puramente financeira não justificaria investimentos adicionais. Como argumenta \citeauthor{bishop2018computer}, os custos de violações não podem ser medidos apenas em termos monetários, mas devem incluir danos à dignidade, autonomia e bem-estar dos indivíduos afetados \cite{bishop2018computer}.

\section{Transparência e Participação no Processo de Gerenciamento de Riscos}

O gerenciamento ético de riscos requer transparência sobre os riscos assumidos e, quando possível, participação dos stakeholders afetados nas decisões. Usuários têm o direito de compreender os riscos a que seus dados estão expostos e as medidas implementadas para protegê-los, permitindo escolhas informadas sobre o uso de sistemas digitais.

A ISO 27001 e frameworks similares reconhecem a importância da comunicação e consulta com stakeholders como parte integral do processo de gerenciamento de riscos \cite{iso27001}. No entanto, a implementação ética desses frameworks vai além da conformidade técnica, buscando engajamento genuíno com as preocupações dos afetados e consideração séria de suas perspectivas nas decisões sobre riscos.

\section{Responsabilidade Compartilhada e Governança Ética}

O gerenciamento ético de riscos reconhece que a responsabilidade pela segurança da informação não pode ser isolada em departamentos técnicos, mas deve ser compartilhada por toda a organização. A governança de segurança da informação deve incorporar considerações éticas em todos os níveis decisórios, desde políticas estratégicas até operações diárias.

Comitês de ética que incluem diversidade de perspectivas, incluindo representantes de usuários e especialistas em direitos digitais, podem fortalecer a qualidade ética das decisões sobre gerenciamento de riscos. Como destacam \citeauthor{stallings2017cryptography}, a segurança efetiva requer uma cultura organizacional que valorize a proteção de dados como responsabilidade compartilhada, não apenas uma função técnica isolada \cite{stallings2017cryptography}.

Este modelo de governança ética reconhece que, mesmo com recursos limitados, organizações podem priorizar a proteção de dados mais sensíveis e implementar controles compensatórios quando proteções técnicas completas não são viáveis, demonstrando compromisso com princípios éticos mesmo sob restrições práticas. 
    
    \chapter{Conclusão}
    \chapter{Conclusão}

Este trabalho apresentou uma série de experimentos práticos sobre o Pretty Good Privacy (PGP), demonstrando os principais aspectos de seu funcionamento e aplicação. Através da manipulação direta do GnuPG, foi possível compreender em profundidade os mecanismos que sustentam esta tecnologia de criptografia amplamente utilizada.

Os experimentos realizados abrangeram todo o ciclo de vida dos certificados PGP:

\begin{itemize}
    \item A criação de certificados PGP, com a geração de pares de chaves, exportação, backup e publicação em servidores;
    \item A revogação de certificados, essencial quando há comprometimento da chave privada ou mudança de informações do usuário;
    \item O processo de assinatura de chaves de terceiros e a posterior revogação dessas assinaturas, elementos fundamentais para o funcionamento da Web of Trust;
    \item A utilização prática do PGP para criptografar arquivos e assinar documentos, demonstrando sua aplicabilidade cotidiana.
\end{itemize}

Além dos aspectos práticos, foram explorados conceitos teóricos importantes como a estrutura do anel de chaves privadas, a organização do banco de dados de confiabilidade, o propósito das subchaves e os requisitos para criação e manutenção de servidores de chaves PGP.

Os experimentos confirmaram a eficácia do PGP em fornecer:

\begin{enumerate}
    \item \textbf{Confidencialidade}: Através da criptografia, garantindo que apenas o destinatário pretendido possa acessar o conteúdo da mensagem;
    \item \textbf{Integridade}: Assegurando que o conteúdo não seja alterado durante a transmissão;
    \item \textbf{Autenticidade}: Confirmando a identidade do remetente por meio de assinaturas digitais;
    \item \textbf{Não-repúdio}: Impedindo que o signatário negue posteriormente a autoria da mensagem assinada.
\end{enumerate}

Também ficou evidente que, apesar de sua robustez criptográfica, a eficácia do PGP depende criticamente da gestão adequada das chaves, especialmente da proteção das chaves privadas. A possibilidade de revogação de certificados e assinaturas mostrou-se um mecanismo crucial de segurança, permitindo responder a incidentes como o comprometimento de chaves.

Os desafios encontrados durante os experimentos, como a complexidade da interface de linha de comando do GnuPG e a dificuldade em verificar o status de chaves em servidores, refletem alguns dos obstáculos que contribuem para a adoção limitada do PGP entre usuários comuns, apesar de suas vantagens técnicas.

É importante ressaltar que o PGP, mesmo com mais de três décadas desde sua criação, continua sendo uma ferramenta relevante no cenário atual de crescente vigilância digital e ameaças à privacidade. O conhecimento adquirido neste trabalho não apenas fornece habilidades técnicas específicas, mas também contribui para uma compreensão mais ampla dos princípios de criptografia de chave pública e privacidade digital.

Concluímos que, apesar dos desafios de usabilidade, o PGP permanece como uma solução robusta para segurança de comunicações, especialmente em contextos que exigem alto nível de confidencialidade e autenticidade. O conjunto de experimentos realizados fornece uma base sólida para futuras explorações e aplicações práticas desta tecnologia, seja em contextos pessoais, acadêmicos ou profissionais relacionados à segurança da informação.

    % ELEMENTOS PÓS-TEXTUAIS
    \postextual
    
    % Referências bibliográficas
    \printbibliography[title=\lang{REFERENCES}{REFERÊNCIAS}]

\end{document}