\section{Teste de Primalidade de Miller-Rabin}\label{apx:miller-rabin-impl}

\subsection{Implementação do Algoritmo}

\begin{verbatim}
#!/usr/bin/env python3
"""
Implementação do Teste de Primalidade de Miller-Rabin

Este módulo implementa o algoritmo de teste de primalidade Miller-Rabin,
um algoritmo probabilístico eficiente para determinar se um número é provavelmente primo.

Autor: João Pedro Schmidt Cordeiro
Data: Abril 2024
"""

import random
import time
from typing import Tuple, List


def decompose(n: int) -> Tuple[int, int]:
    """
    Decompõe n-1 como (2^s) * d, onde d é ímpar.
    
    Args:
        n: O número a ser decomposto (n-1)
    
    Returns:
        Uma tupla (s, d) onde s e d satisfazem n-1 = (2^s) * d e d é ímpar
    """
    n_minus_1 = n - 1
    s = 0
    d = n_minus_1
    
    # Enquanto d for par, divide por 2 e incrementa s
    while d % 2 == 0:
        d //= 2
        s += 1
    
    return s, d


def miller_rabin_round(n: int, a: int) -> bool:
    """
    Executa uma rodada do teste de Miller-Rabin com a base 'a'.
    
    Args:
        n: O número a ser testado para primalidade
        a: A base para o teste (2 <= a <= n-2)
    
    Returns:
        True se n passa no teste para a base a, False caso contrário
    """
    if n == a:
        return True
    
    if n % a == 0:
        return False
    
    # Decompõe n-1 como 2^s * d
    s, d = decompose(n)
    
    # Calcula x = a^d mod n
    x = pow(a, d, n)
    
    # Se x == 1 ou x == n-1, n passa no teste para essa base
    if x == 1 or x == n - 1:
        return True
    
    # Calcula x = x^2 mod n para s-1 iterações
    for _ in range(s - 1):
        x = pow(x, 2, n)
        # Se x == n-1, n passa no teste para essa base
        if x == n - 1:
            return True
        # Se x == 1, encontramos uma raiz quadrada não-trivial de 1, então n é composto
        if x == 1:
            return False
    
    # Se chegamos aqui, n é composto para esta base
    return False


def is_prime_miller_rabin(n: int, k: int = 40) -> bool:
    """
    Determina se n é provavelmente primo usando o teste de Miller-Rabin.
    
    Args:
        n: O número a ser testado
        k: O número de rodadas/bases a serem testadas (padrão: 40)
           Quanto maior o valor de k, menor a probabilidade de erro
    
    Returns:
        True se n é provavelmente primo, False se n é definitivamente composto
    """
    # Casos triviais
    if n <= 1:
        return False
    if n <= 3:
        return True
    if n % 2 == 0:
        return False
    
    # Para números pequenos, podemos usar um conjunto fixo de bases
    # que garantem determinismo até certo limite
    # Recomendação de Menezes, van Oorschot e Vanstone (1996)
    if n < 1_373_653:
        # Para n < 1,373,653, é suficiente testar as bases 2, 3
        bases_deterministic = [2, 3]
    elif n < 9_080_191:
        # Para n < 9,080,191, é suficiente testar as bases 31, 73
        bases_deterministic = [31, 73]
    elif n < 25_326_001:
        # Para n < 25,326,001, é suficiente testar as bases 2, 3, 5
        bases_deterministic = [2, 3, 5]
    elif n < 3_215_031_751:
        # Para n < 3,215,031,751, é suficiente testar as bases 2, 3, 5, 7
        bases_deterministic = [2, 3, 5, 7]
    elif n < 4_759_123_141:
        # Para n < 4,759,123,141, é suficiente testar as bases 2, 7, 61
        bases_deterministic = [2, 7, 61]
    elif n < 1_122_004_669_633:
        # Para n < 1,122,004,669,633, é suficiente testar as bases 2, 13, 23, 1662803
        bases_deterministic = [2, 13, 23, 1662803]
    elif n < 2_152_302_898_747:
        # Para n < 2,152,302,898,747, é suficiente testar as bases 2, 3, 5, 7, 11
        bases_deterministic = [2, 3, 5, 7, 11]
    elif n < 3_474_749_660_383:
        # Para n < 3,474,749,660,383, é suficiente testar as bases 2, 3, 5, 7, 11, 13
        bases_deterministic = [2, 3, 5, 7, 11, 13]
    elif n < 341_550_071_728_321:
        # Para n < 341,550,071,728,321, é suficiente testar as bases 2, 3, 5, 7, 11, 13, 17
        bases_deterministic = [2, 3, 5, 7, 11, 13, 17]
    elif n < 2**64:
        # Para n < 2^64, é suficiente testar as bases 2, 3, 5, 7, 11, 13, 17, 19, 23, 29, 31, 37
        # Conforme provado por Pomerance, Selfridge e Wagstaff e ampliado por Jaeschke
        bases_deterministic = [2, 3, 5, 7, 11, 13, 17, 19, 23, 29, 31, 37]
    else:
        # Para números maiores, usamos bases aleatórias
        for _ in range(k):
            a = random.randint(2, n - 2)
            if not miller_rabin_round(n, a):
                return False
        return True
    
    # Teste com as bases determinísticas selecionadas
    for a in bases_deterministic:
        if a >= n:
            break
        if not miller_rabin_round(n, a):
            return False
    
    # Se todas as k bases passaram no teste, n é provavelmente primo
    return True


def generate_probable_prime(bits: int, k: int = 40) -> int:
    """
    Gera um número provavelmente primo com o número especificado de bits.
    
    Args:
        bits: Número de bits do primo a ser gerado
        k: Número de rodadas no teste de Miller-Rabin (padrão: 40)
    
    Returns:
        Um número provavelmente primo com o número especificado de bits
    """
    while True:
        # Gera um número aleatório com o número especificado de bits
        # Garante que o número é ímpar e tem exatamente o número de bits solicitado
        n = random.getrandbits(bits)
        n |= (1 << (bits - 1)) | 1  # Garante que o bit mais significativo é 1 e que é ímpar
        
        if is_prime_miller_rabin(n, k):
            return n
\end{verbatim}

\subsection{Implementação do Teste}

\begin{verbatim}
#!/usr/bin/env python3
"""
Experimento de Geração de Números Primos usando o Teste de Miller-Rabin

Este script realiza experimentos de geração de números primos de diferentes tamanhos
usando o algoritmo de Miller-Rabin. Ele gera uma tabela com o tempo necessário 
para encontrar um número primo para cada tamanho especificado.

Autor: João Pedro Schmidt Cordeiro
Data: Abril 2024
"""

import sys
import time
import random
from typing import Dict, Tuple, List
from datetime import datetime

# Importa as funções do arquivo principal
from main import is_prime_miller_rabin, generate_probable_prime


def find_prime_with_timeout(bits: int, timeout_seconds: float = 300) -> Tuple[int, float, int, bool]:
    """
    Tenta encontrar um número primo com o número especificado de bits,
    com um limite de tempo para evitar que o programa fique preso.
    
    Args:
        bits: Número de bits do primo a ser gerado
        timeout_seconds: Tempo máximo em segundos para tentar
        
    Returns:
        Uma tupla (primo, tempo_em_ms, tentativas, sucesso)
    """
    print(f"Testando geração de primo de {bits} bits...", end="", flush=True)
    
    start_time = time.time()
    end_time = start_time + timeout_seconds
    attempts = 0
    prime = None
    
    while time.time() < end_time:
        attempts += 1
        if attempts % 10 == 0:
            print(".", end="", flush=True)
        
        # Gera um número ímpar aleatório com o número exato de bits
        n = random.getrandbits(bits)
        n |= (1 << (bits - 1)) | 1  # Garante MSB=1 e número ímpar
        
        if is_prime_miller_rabin(n):
            prime = n
            break
    
    duration = (time.time() - start_time) * 1000  # Converte para milissegundos
    success = prime is not None
    
    if success:
        print(f" Encontrado em {attempts} tentativas e {duration:.2f} ms")
    else:
        print(f" Falha após {attempts} tentativas e {timeout_seconds*1000:.2f} ms")
    
    return prime, duration, attempts, success


def format_number_compact(n: int) -> str:
    """
    Formata um número grande de forma compacta, mostrando apenas o início e o fim.
    
    Args:
        n: O número a ser formatado
        
    Returns:
        Uma string com o número formatado
    """
    if n is None:
        return "N/A"
    
    str_n = str(n)
    if len(str_n) <= 20:
        return str_n
    
    return f"{str_n[:10]}...{str_n[-10:]}"


def run_prime_experiment() -> Dict:
    """
    Executa o experimento de geração de números primos.
    
    Returns:
        Um dicionário com os resultados do experimento
    """
    bit_sizes = [40, 56, 80, 128, 168, 224, 256, 512, 1024, 2048, 4096]
    results = {
        "algorithm": "Miller-Rabin",
        "timestamp": datetime.now().strftime("%Y-%m-%d %H:%M:%S"),
        "results": []
    }
    
    print(f"Experimento iniciado em: {results['timestamp']}")
    print("=" * 80)
    
    # Tempo limite aumenta com o tamanho do número
    for bits in bit_sizes:
        # Define um timeout adequado baseado no tamanho
        if bits <= 128:
            timeout = 60  # 1 minuto para números pequenos
        elif bits <= 512:
            timeout = 300  # 5 minutos para números médios
        elif bits <= 1024:
            timeout = 600  # 10 minutos para números grandes
        else:
            timeout = 1200  # 20 minutos para números muito grandes
        
        prime, duration, attempts, success = find_prime_with_timeout(bits, timeout)
        
        result_entry = {
            "bits": bits,
            "prime": prime,
            "time_ms": duration,
            "attempts": attempts,
            "status": "Sucesso" if success else "Falha"
        }
        
        results["results"].append(result_entry)
    
    print("=" * 80)
    print(f"Experimento concluído em: {datetime.now().strftime('%Y-%m-%d %H:%M:%S')}")
    
    return results


def format_results_as_table(results: Dict) -> str:
    """
    Formata os resultados como uma tabela Markdown.
    
    Args:
        results: Os resultados do experimento
        
    Returns:
        Uma string contendo a tabela em formato Markdown
    """
    table = "| Algoritmo | Tamanho do Número | Tentativas | Tempo para gerar |\n"
    table += "|-----------|-------------------|------------|-------------------|\n"
    
    for result in results["results"]:
        algorithm = results["algorithm"]
        bits = f"{result['bits']} bits"
        
        if result["status"] == "Sucesso":
            attempts_str = str(result["attempts"])
            time_str = f"{result['time_ms']:.2f} ms"
        else:
            attempts_str = str(result["attempts"])
            time_str = "Timeout"
        
        table += f"| {algorithm} | {bits} | {attempts_str} | {time_str} |\n"
    
    return table


def generate_report(results: Dict) -> None:
    """
    Gera um relatório completo do experimento.
    
    Args:
        results: Os resultados do experimento
    """
    # Cria o relatório
    report = "# Experimento de Geração de Números Primos com o Teste de Miller-Rabin\n\n"
    report += f"Data/Hora: {results['timestamp']}\n\n"
    
    # Adiciona a tabela
    report += "## Resultados\n\n"
    report += format_results_as_table(results)
    report += "\n\n"
    
    # Análise e observações
    report += "## Observações\n\n"
    
    # Analisa taxa de sucesso
    successful = sum(1 for r in results["results"] if r["status"] == "Sucesso")
    total = len(results["results"])
    success_rate = (successful / total) * 100 if total > 0 else 0
    
    report += f"- Taxa de sucesso: {successful}/{total} ({success_rate:.1f}%)\n"
    
    # Analisa tempos e tentativas
    if successful > 0:
        times = [r["time_ms"] for r in results["results"] if r["status"] == "Sucesso"]
        attempts = [r["attempts"] for r in results["results"] if r["status"] == "Sucesso"]
        
        avg_time = sum(times) / len(times)
        max_time = max(times)
        min_time = min(times)
        
        avg_attempts = sum(attempts) / len(attempts)
        max_attempts = max(attempts)
        min_attempts = min(attempts)
        
        report += f"- Tempo médio: {avg_time:.2f} ms\n"
        report += f"- Tempo mínimo: {min_time:.2f} ms\n"
        report += f"- Tempo máximo: {max_time:.2f} ms\n"
        report += f"- Tentativas médias: {avg_attempts:.2f}\n"
        report += f"- Tentativas mínimas: {min_attempts}\n"
        report += f"- Tentativas máximas: {max_attempts}\n\n"
        
        # Análise da relação entre tamanho de bits e número de tentativas
        sizes = [r["bits"] for r in results["results"] if r["status"] == "Sucesso"]
        if len(sizes) > 1:
            report += "### Relação entre tamanho e esforço\n\n"
            report += "- À medida que o tamanho em bits aumenta, nota-se:\n"
            
            # Ordenar resultados por tamanho de bits para comparação
            sorted_results = sorted([r for r in results["results"] if r["status"] == "Sucesso"], 
                                    key=lambda x: x["bits"])
            
            if len(sorted_results) >= 2:
                smallest = sorted_results[0]
                largest = sorted_results[-1]
                report += f"  - Para {smallest['bits']} bits: {smallest['attempts']} tentativas, {smallest['time_ms']:.2f} ms\n"
                report += f"  - Para {largest['bits']} bits: {largest['attempts']} tentativas, {largest['time_ms']:.2f} ms\n"
                
                attempt_increase = largest['attempts'] / smallest['attempts'] if smallest['attempts'] > 0 else 0
                time_increase = largest['time_ms'] / smallest['time_ms'] if smallest['time_ms'] > 0 else 0
                
                report += f"  - Aumento de tentativas: {attempt_increase:.2f}x\n"
                report += f"  - Aumento de tempo: {time_increase:.2f}x\n\n"
    
    # Analisa desempenho por tamanho
    report += "### Dificuldades encontradas\n\n"
    
    failed_bits = [r["bits"] for r in results["results"] if r["status"] == "Falha"]
    if failed_bits:
        report += f"- Não foi possível gerar números primos para os seguintes tamanhos: {', '.join(str(b) for b in failed_bits)} bits\n"
    
    # Salva o relatório
    filename = "miller_rabin_report.md"
    with open(filename, "w") as f:
        f.write(report)
    
    print(f"\nRelatório completo gerado com sucesso: {filename}")
\end{verbatim}
