\section{Xorshift}\label{apx:xorshift-impl}

\subsection{Implementação do Algoritmo}

\begin{verbatim}
"""
Implementação do algoritmo Xorshift PRNG
Baseado no trabalho original de George Marsaglia (2003)

Inclui implementações para versões de 32 bits, 64 bits e 128 bits.
"""

class Xorshift32:
    """
    Implementação do Xorshift de 32 bits com período de 2^32-1.
    """
    def __init__(self, seed=123456789):
        """
        Inicializa o gerador com uma semente não-nula.
        
        Args:
            seed (int): Valor de semente. Deve ser não-zero.
        """
        if seed == 0:
            seed = 123456789  # Evita estado zero
        self.state = seed & 0xFFFFFFFF  # Garante valor de 32 bits
    
    def next(self):
        """
        Gera o próximo número pseudoaleatório.
        
        Returns:
            int: Um número pseudoaleatório de 32 bits.
        """
        # Copia o estado atual para a variável x
        x = self.state
        # Aplica o primeiro deslocamento: desloca x 13 bits à esquerda, 
        # faz XOR com o valor original e mantém apenas os 32 bits inferiores
        x ^= (x << 13) & 0xFFFFFFFF
        # Aplica o segundo deslocamento: desloca x 17 bits à direita,
        # faz XOR com o resultado anterior e mantém apenas os 32 bits inferiores
        x ^= (x >> 17) & 0xFFFFFFFF
        # Aplica o terceiro deslocamento: desloca x 5 bits à esquerda,
        # faz XOR com o resultado anterior e mantém apenas os 32 bits inferiores
        x ^= (x << 5) & 0xFFFFFFFF
        # Atualiza o estado interno do gerador com o novo valor
        # garantindo que apenas os 32 bits inferiores sejam mantidos
        self.state = x & 0xFFFFFFFF
        
        # Retorna o novo estado como o número pseudoaleatório gerado
        return self.state
    
    def random(self):
        """
        Retorna um número de ponto flutuante no intervalo [0.0, 1.0).
        
        Returns:
            float: Um número entre 0.0 e 1.0.
        """
        return self.next() / 0x100000000


class Xorshift64:
    """
    Implementação do Xorshift de 64 bits com período de 2^64-1.
    """
    def __init__(self, seed=88172645463325252):
        """
        Inicializa o gerador com uma semente não-nula.
        
        Args:
            seed (int): Valor de semente. Deve ser não-zero.
        """
        if seed == 0:
            seed = 88172645463325252  # Evita estado zero
        self.state = seed & 0xFFFFFFFFFFFFFFFF  # Garante valor de 64 bits
    
    def next(self):
        """
        Gera o próximo número pseudoaleatório.
        
        Returns:
            int: Um número pseudoaleatório de 64 bits.
        """
        # Copia o estado atual para a variável x
        x = self.state
        # Aplica XOR entre x e (x deslocado 13 bits à esquerda), mantendo apenas 64 bits
        x ^= (x << 13) & 0xFFFFFFFFFFFFFFFF
        # Aplica XOR entre x e (x deslocado 7 bits à direita), mantendo apenas 64 bits
        x ^= (x >> 7) & 0xFFFFFFFFFFFFFFFF
        # Aplica XOR entre x e (x deslocado 17 bits à esquerda), mantendo apenas 64 bits
        x ^= (x << 17) & 0xFFFFFFFFFFFFFFFF
        # Atualiza o estado interno com o novo valor, garantindo que tenha apenas 64 bits
        self.state = x & 0xFFFFFFFFFFFFFFFF
        # Retorna o novo estado como o número pseudoaleatório gerado
        return self.state
    
    def random(self):
        """
        Retorna um número de ponto flutuante no intervalo [0.0, 1.0).
        
        Returns:
            float: Um número entre 0.0 e 1.0.
        """
        return self.next() / 0x10000000000000000


class Xorshift128:
    """
    Implementação do Xorshift de 128 bits com período de 2^128-1.
    """
    def __init__(self, seed=None):
        """
        Inicializa o gerador com uma semente ou valores padrão.
        
        Args:
            seed (list, optional): Lista de 4 valores inteiros para inicializar o estado.
        """
        if seed is None:
            # Valores iniciais não-nulos por padrão
            self.state = [123456789, 362436069, 521288629, 88675123]
        else:
            # Verifica se todos os valores são zero
            if all(s == 0 for s in seed):
                self.state = [123456789, 362436069, 521288629, 88675123]
            else:
                self.state = [s & 0xFFFFFFFF for s in seed[:4]]
    
    def next(self):
        """
        Gera o próximo número pseudoaleatório.
        
        Returns:
            int: Um número pseudoaleatório de 32 bits.
        """
        # Armazena o último valor do estado (índice 3) em t
        t = self.state[3]
        # Armazena o primeiro valor do estado (índice 0) em s
        s = self.state[0]
        
        # Desloca os valores do estado: o valor do índice 2 vai para o índice 3
        self.state[3] = self.state[2]
        # Desloca os valores do estado: o valor do índice 1 vai para o índice 2
        self.state[2] = self.state[1]
        # Desloca os valores do estado: o valor do índice 0 (s) vai para o índice 1
        self.state[1] = s
        
        # Aplica a primeira transformação XOR: desloca t 11 bits à esquerda e faz XOR com t
        # A máscara 0xFFFFFFFF garante que o resultado tenha 32 bits
        t ^= (t << 11) & 0xFFFFFFFF
        # Aplica a segunda transformação XOR: desloca t 8 bits à direita e faz XOR com t
        t ^= (t >> 8) & 0xFFFFFFFF
        # Aplica a terceira transformação XOR: faz XOR entre t e o resultado de (s XOR (s deslocado 19 bits à direita))
        t ^= (s ^ (s >> 19)) & 0xFFFFFFFF
        
        # Atualiza o primeiro valor do estado (índice 0) com o novo valor de t
        # A máscara 0xFFFFFFFF garante que o resultado tenha 32 bits
        self.state[0] = t & 0xFFFFFFFF
        # Retorna o valor gerado, garantindo que tenha 32 bits
        return t & 0xFFFFFFFF
    
    def random(self):
        """
        Retorna um número de ponto flutuante no intervalo [0.0, 1.0).
        
        Returns:
            float: Um número entre 0.0 e 1.0.
        """
        return self.next() / 0x100000000


class Xorshift128Plus:
    """
    Implementação do Xorshift128+ - uma variante que utiliza adição para
    melhorar a qualidade estatística (período 2^128-1).
    """
    def __init__(self, seed=None):
        """
        Inicializa o gerador com uma semente ou valores padrão.
        
        Args:
            seed (list, optional): Lista de 2 valores inteiros de 64 bits para inicializar o estado.
        """
        if seed is None:
            # Valores iniciais não-nulos por padrão
            self.state = [1234567890123456789, 9876543210987654321]
        else:
            # Verifica se todos os valores são zero
            if all(s == 0 for s in seed):
                self.state = [1234567890123456789, 9876543210987654321]
            else:
                self.state = [s & 0xFFFFFFFFFFFFFFFF for s in seed[:2]]
    
    def next(self):
        """
        Gera o próximo número pseudoaleatório.
        
        Returns:
            int: Um número pseudoaleatório de 64 bits.
        """
        # Armazena os dois valores do estado atual em variáveis temporárias
        s0 = self.state[0]  # Primeiro valor do estado (64 bits)
        s1 = self.state[1]  # Segundo valor do estado (64 bits)
        
        # Atualiza o primeiro valor do estado com o valor de s1
        # Isso faz parte da rotação do estado para a próxima iteração
        self.state[0] = s1
        
        # Aplica a primeira transformação XOR em s1:
        # Desloca s1 23 bits à esquerda, faz XOR com s1 original
        # A máscara 0xFFFFFFFFFFFFFFFF garante que o resultado tenha 64 bits
        s1 ^= (s1 << 23) & 0xFFFFFFFFFFFFFFFF
        
        # Calcula o novo valor para o segundo elemento do estado:
        # Combina s1 transformado com s0 original e mais duas operações de deslocamento e XOR
        # - s1 ^ s0: XOR entre s1 transformado e s0 original
        # - ^ (s1 >> 18): XOR com s1 deslocado 18 bits à direita
        # - ^ (s0 >> 5): XOR com s0 deslocado 5 bits à direita
        # A máscara final garante que o resultado tenha 64 bits
        self.state[1] = (s1 ^ s0 ^ (s1 >> 18) ^ (s0 >> 5)) & 0xFFFFFFFFFFFFFFFF
        
        # Retorna a soma dos dois valores do estado como o número pseudoaleatório
        # Esta adição é o que diferencia o Xorshift128+ do Xorshift128 padrão,
        # melhorando a qualidade estatística dos números gerados
        # A máscara garante que o resultado tenha 64 bits
        return (self.state[0] + self.state[1]) & 0xFFFFFFFFFFFFFFFF
    
    def random(self):
        """
        Retorna um número de ponto flutuante no intervalo [0.0, 1.0).
        
        Returns:
            float: Um número entre 0.0 e 1.0.
        """
        return self.next() / 0x10000000000000000
\end{verbatim}

\subsection{Implementação do Teste}

\begin{verbatim}
"""
Experimento para avaliação de desempenho de geradores de números pseudoaleatórios.

Este script avalia o tempo necessário para gerar números pseudoaleatórios de diferentes
tamanhos (40-4096 bits) usando diferentes implementações do algoritmo Xorshift.

O experimento gera múltiplos números para cada tamanho e calcula o tempo médio.
"""

import time
import sys
from main import Xorshift32, Xorshift64, Xorshift128, Xorshift128Plus

# Tamanhos de bits a serem testados
BIT_SIZES = [40, 56, 80, 128, 168, 224, 256, 512, 1024, 2048, 4096]

# Número de amostras para calcular o tempo médio
NUM_SAMPLES = 1000


def generate_random_bits(generator, num_bits):
    """
    Gera um número pseudoaleatório com o número especificado de bits.
    
    Args:
        generator: Objeto gerador Xorshift
        num_bits: Número de bits desejado
    
    Returns:
        int: Número pseudoaleatório com o tamanho especificado
    """
    result = 0
    bits_generated = 0
    
    # Determina o número de bits por chamada com base no tipo de gerador
    if isinstance(generator, Xorshift32):
        bits_per_call = 32
    elif isinstance(generator, Xorshift64) or isinstance(generator, Xorshift128Plus):
        bits_per_call = 64
    elif isinstance(generator, Xorshift128):
        bits_per_call = 32
    else:
        raise ValueError("Gerador não reconhecido")
    
    # Gera bits até atingir o tamanho desejado
    while bits_generated < num_bits:
        # Gera um novo número
        new_bits = generator.next()
        
        # Adiciona os novos bits ao resultado (shifts e OR)
        result = (result << bits_per_call) | new_bits
        bits_generated += bits_per_call
    
    # Ajusta para o tamanho exato solicitado (remove bits extras)
    if bits_generated > num_bits:
        extra_bits = bits_generated - num_bits
        result = result >> extra_bits
        
    # Garante que o número tenha exatamente o número de bits solicitado
    # Seta o bit mais significativo para 1
    result |= (1 << (num_bits - 1))
    
    return result


def measure_generation_time(generator_class, bit_size, seed=42):
    """
    Mede o tempo médio para gerar números pseudoaleatórios de determinado tamanho.
    
    Args:
        generator_class: Classe do gerador a ser usada
        bit_size: Tamanho do número em bits
        seed: Semente para inicializar o gerador
    
    Returns:
        float: Tempo médio em milissegundos
    """
    # Inicializa o gerador apropriadamente dependendo da classe
    if generator_class == Xorshift32 or generator_class == Xorshift64:
        generator = generator_class(seed=seed)
    elif generator_class == Xorshift128:
        generator = generator_class(seed=[seed, seed+1, seed+2, seed+3])
    elif generator_class == Xorshift128Plus:
        generator = generator_class(seed=[seed, seed+1])
    
    # Mede o tempo total para gerar NUM_SAMPLES números
    start_time = time.time()
    
    for _ in range(NUM_SAMPLES):
        generate_random_bits(generator, bit_size)
    
    end_time = time.time()
    
    # Calcula o tempo médio em milissegundos
    avg_time_ms = ((end_time - start_time) / NUM_SAMPLES) * 1000
    
    return avg_time_ms


def run_experiment():
    """
    Executa o experimento completo e exibe os resultados em formato de tabela.
    """
    generators = [
        ("Xorshift32", Xorshift32),
        ("Xorshift64", Xorshift64),
        ("Xorshift128", Xorshift128),
        ("Xorshift128Plus", Xorshift128Plus)
    ]
    
    # Imprime o cabeçalho da tabela
    print("\n" + "=" * 80)
    print(f"{'Algoritmo':<20} | {'Tamanho do Número':<20} | {'Tempo para gerar (ms)':<20}")
    print("-" * 80)
    
    # Executa o experimento para cada combinação de gerador e tamanho
    for gen_name, gen_class in generators:
        for bit_size in BIT_SIZES:
            # Mede o tempo de geração
            time_ms = measure_generation_time(gen_class, bit_size)
            
            # Formata e imprime o resultado
            print(f"{gen_name:<20} | {bit_size} bits{' ':<14} | {time_ms:.4f} ms")
        
        # Linha separadora entre geradores
        print("-" * 80)
    
    print("=" * 80)
    print(f"Experimento concluído. Cada medição representa a média de {NUM_SAMPLES} execuções.")


def verify_random_number_quality(show_numbers=False):
    """
    Verifica a qualidade dos números gerados por cada algoritmo.
    Esta função é útil para validar se os números gerados têm o tamanho especificado.
    
    Args:
        show_numbers: Se True, exibe os números gerados para inspeção visual
    """
    print("\nVerificação da qualidade dos números gerados:")
    print("-" * 80)
    
    generators = [
        ("Xorshift32", Xorshift32(seed=42)),
        ("Xorshift64", Xorshift64(seed=42)),
        ("Xorshift128", Xorshift128(seed=[42, 43, 44, 45])),
        ("Xorshift128Plus", Xorshift128Plus(seed=[42, 43]))
    ]
    
    # Seleciona alguns tamanhos para testar
    test_sizes = [40, 128, 256, 1024]
    
    for gen_name, generator in generators:
        print(f"\nGerador: {gen_name}")
        
        for bit_size in test_sizes:
            number = generate_random_bits(generator, bit_size)
            
            # Verifica se o número tem o tamanho correto (em bits)
            binary = bin(number)[2:]  # Remove o prefixo '0b'
            actual_bits = len(binary)
            
            print(f"  Tamanho solicitado: {bit_size} bits")
            print(f"  Tamanho real: {actual_bits} bits")
            
            if show_numbers:
                # Limita a exibição para evitar números muito grandes
                if bit_size <= 128:
                    print(f"  Número binário: {binary}")
                else:
                    print(f"  Número binário: {binary[:64]}...{binary[-64:]} (parte inicial e final)")
            
            print("-" * 40)


if __name__ == "__main__":
    # Verifica se o usuário quer executar o experimento com verificação de qualidade
    verify_quality = len(sys.argv) > 1 and sys.argv[1] == "--verify"
    
    if verify_quality:
        verify_random_number_quality(show_numbers=True)
    else:
        print("\nIniciando experimento de geração de números pseudoaleatórios...")
        print(f"Gerando {NUM_SAMPLES} amostras para cada combinação algoritmo-tamanho")
        run_experiment()
        print("\nDica: Execute com a flag --verify para testar a qualidade dos números gerados.")
\end{verbatim}
