\section{Blum Blum Shub}\label{apx:bbs-impl}

\subsection{Implementação do Algoritmo}

\begin{verbatim}
"""
Implementação do algoritmo Blum Blum Shub (BBS) para geração de números pseudoaleatórios.

BBS é um gerador de números pseudoaleatórios criptograficamente seguro proposto em 1986
por Lenore Blum, Manuel Blum e Michael Shub. Sua segurança baseia-se na dificuldade
do problema de fatoração de inteiros.
"""

import random
import math
import time
from typing import Tuple, Optional


def is_prime(n: int, k: int = 40) -> bool:
    """
    Verifica se um número é provavelmente primo usando o teste de Miller-Rabin.
    
    Args:
        n: O número a ser testado
        k: Número de iterações para o teste (maior k = maior confiabilidade)
        
    Returns:
        bool: True se n for provavelmente primo, False caso contrário
    """
    # Casos básicos
    if n <= 1:
        return False
    if n <= 3:
        return True
    if n % 2 == 0:
        return False
    
    # Expressando n-1 como 2^r * d
    r, d = 0, n - 1
    while d % 2 == 0:
        r += 1
        d //= 2
    
    # Teste de Miller-Rabin
    for _ in range(k):
        a = random.randint(2, n - 2)
        x = pow(a, d, n)
        if x == 1 or x == n - 1:
            continue
        for _ in range(r - 1):
            x = pow(x, 2, n)
            if x == n - 1:
                break
        else:
            return False
    return True


def generate_prime(bits: int) -> int:
    """
    Gera um número primo com o número especificado de bits.
    
    Args:
        bits: Número de bits do primo a ser gerado
        
    Returns:
        int: Um número primo com o número especificado de bits
    """
    while True:
        # Gera um número aleatório com o número especificado de bits
        p = random.randint(2**(bits-1), 2**bits - 1)
        
        # Garante que o número é ímpar
        if p % 2 == 0:
            p += 1
        
        # Verifica se é primo usando o teste de Miller-Rabin
        if is_prime(p):
            return p


def generate_blum_integer(bits: int) -> Tuple[int, int, int]:
    """
    Gera um inteiro de Blum, que é o produto de dois primos grandes, ambos
    congruentes a 3 mod 4.
    
    Args:
        bits: Número de bits para cada primo (o inteiro de Blum terá aproximadamente 2*bits)
        
    Returns:
        Tuple[int, int, int]: (n, p, q) onde n = p * q é o inteiro de Blum
    """
    # Gera o primeiro primo p tal que p \equiv 3 (mod 4)
    while True:
        p = generate_prime(bits)
        if p % 4 == 3:
            break
    
    # Gera o segundo primo q tal que q \equiv 3 (mod 4)
    while True:
        q = generate_prime(bits)
        if q % 4 == 3 and p != q:  # Garante que p e q são diferentes
            break
    
    # Retorna o inteiro de Blum n = p * q
    return p * q, p, q


def gcd(a: int, b: int) -> int:
    """
    Calcula o Máximo Divisor Comum (MDC) entre dois números.
    
    Args:
        a: Primeiro número
        b: Segundo número
        
    Returns:
        int: O MDC entre a e b
    """
    while b:
        a, b = b, a % b
    return a


class BlumBlumShub:
    """
    Implementação do gerador de números pseudoaleatórios Blum Blum Shub.
    """
    
    def __init__(self, seed: Optional[int] = None, n: Optional[int] = None, 
                 p: Optional[int] = None, q: Optional[int] = None, bits: int = 512):
        """
        Inicializa o gerador BBS.
        
        Args:
            seed: Semente inicial para o gerador (opcional)
            n: O inteiro de Blum a ser usado (opcional)
            p: Primeiro primo utilizado na geração de n (opcional)
            q: Segundo primo utilizado na geração de n (opcional)
            bits: Número de bits para cada um dos primos (se n não for fornecido)
        """
        # Se n não foi fornecido, gera um novo inteiro de Blum
        if n is None:
            self.n, self.p, self.q = generate_blum_integer(bits)
        else:
            # Usa os valores fornecidos
            self.n = n
            self.p = p
            self.q = q
        
        # Se a semente não foi fornecida, gera uma aleatória
        if seed is None:
            # Gera uma semente aleatória coprima com n
            while True:
                seed = random.randint(2, self.n - 1)
                if gcd(seed, self.n) == 1:
                    break
        else:
            # Verifica se a semente fornecida é válida
            if seed <= 1 or seed >= self.n:
                raise ValueError("A semente deve estar no intervalo [2, n-1]")
            if gcd(seed, self.n) != 1:
                raise ValueError("A semente deve ser coprima com n")
        
        # Inicializa o estado com x_0 = s^2 mod n
        self.state = (seed * seed) % self.n
    
    def next_bit(self) -> int:
        """
        Gera o próximo bit pseudoaleatório.
        
        Returns:
            int: 0 ou 1 (o bit menos significativo do novo estado)
        """
        # Calcula x_{i+1} = x_i² mod n
        self.state = (self.state * self.state) % self.n
        
        # Retorna o bit menos significativo
        return self.state % 2
    
    def next_byte(self) -> int:
        """
        Gera o próximo byte (8 bits) pseudoaleatório.
        
        Returns:
            int: Um número entre 0 e 255
        """
        byte = 0
        # Gera 8 bits para formar um byte
        for i in range(8):
            bit = self.next_bit()
            # Constrói o byte bit a bit (do mais significativo para o menos)
            byte = (byte << 1) | bit
        return byte
    
    def random_bits(self, num_bits: int) -> int:
        """
        Gera um número com a quantidade especificada de bits.
        
        Args:
            num_bits: Número de bits a serem gerados
            
        Returns:
            int: Um número pseudoaleatório com o tamanho especificado
        """
        result = 0
        # Gera os bits um a um
        for _ in range(num_bits):
            bit = self.next_bit()
            # Constrói o número bit a bit
            result = (result << 1) | bit
        
        # Garante que o bit mais significativo seja 1 (para garantir exatamente num_bits bits)
        result |= (1 << (num_bits - 1))
        
        return result
    
    def random(self) -> float:
        """
        Gera um número de ponto flutuante pseudoaleatório no intervalo [0.0, 1.0).
        
        Returns:
            float: Um número entre 0.0 e 1.0
        """
        # Usa 53 bits para garantir precisão suficiente para um float
        return self.random_bits(53) / (1 << 53)
\end{verbatim}

\subsection{Implementação do Teste}

\begin{verbatim}
"""
Experimento para avaliar o desempenho do gerador de números pseudoaleatórios Blum Blum Shub.

Este script mede o tempo necessário para gerar números pseudoaleatórios de diferentes
tamanhos (40-4096 bits) usando o algoritmo Blum Blum Shub (BBS).
"""

import time
import sys
from main import BlumBlumShub

# Tamanhos de bits a serem testados
BIT_SIZES = [40, 56, 80, 128, 168, 224, 256, 512, 1024, 2048, 4096]

# Número de amostras para calcular o tempo médio
NUM_SAMPLES = 10  # Menos amostras para BBS porque é muito mais lento


def measure_generation_time(bit_size):
    """
    Mede o tempo médio para gerar números pseudoaleatórios de determinado tamanho.
    
    Args:
        bit_size: Tamanho do número em bits
    
    Returns:
        float: Tempo médio em milissegundos
    """
    # Inicializa o gerador BBS com um módulo de tamanho adequado
    # O tamanho do módulo deve ser pelo menos duas vezes maior que o do número a ser gerado
    module_bits = max(512, bit_size * 2)
    
    # Medimos também o tempo de inicialização
    init_start_time = time.time()
    bbs = BlumBlumShub(bits=module_bits // 2)
    init_time = (time.time() - init_start_time) * 1000  # em ms
    
    # Mede o tempo total para gerar NUM_SAMPLES números
    total_time = 0
    
    for _ in range(NUM_SAMPLES):
        start_time = time.time()
        bbs.random_bits(bit_size)
        end_time = time.time()
        total_time += (end_time - start_time) * 1000  # em ms
    
    # Calcula o tempo médio em milissegundos
    avg_time_ms = total_time / NUM_SAMPLES
    
    return avg_time_ms, init_time


def run_experiment():
    """
    Executa o experimento completo e exibe os resultados em formato de tabela.
    """
    print("\n" + "=" * 80)
    print("Experimento de geração de números pseudoaleatórios com Blum Blum Shub (BBS)")
    print("=" * 80)
    print(f"{'Tamanho do Número':<20} | {'Tempo de Init (ms)':<20} | {'Tempo de Geração (ms)':<25}")
    print("-" * 80)
    
    # Executa o experimento para cada tamanho
    for bit_size in BIT_SIZES:
        try:
            # Mede o tempo de geração
            gen_time_ms, init_time_ms = measure_generation_time(bit_size)
            
            # Formata e imprime o resultado
            print(f"{bit_size} bits{' ':<14} | {init_time_ms:.4f} ms{' ':<10} | {gen_time_ms:.4f} ms")
            
            # Salva o resultado parcial para não perder progresso
            with open("bbs_results.txt", "a") as f:
                f.write(f"{bit_size},{init_time_ms:.4f},{gen_time_ms:.4f}\n")
                
        except Exception as e:
            print(f"{bit_size} bits{' ':<14} | ERRO: {str(e)}")
    
    print("=" * 80)
    print(f"Experimento concluído. Cada medição representa a média de {NUM_SAMPLES} execuções.")


def verify_random_number_quality():
    """
    Verifica a qualidade dos números gerados pelo algoritmo BBS.
    Esta função é útil para validar se os números gerados têm o tamanho especificado.
    """
    print("\nVerificação da qualidade dos números gerados pelo Blum Blum Shub:")
    print("-" * 80)
    
    # Cria um gerador BBS
    bbs = BlumBlumShub(bits=512)
    
    # Seleciona alguns tamanhos para testar
    test_sizes = [40, 128, 256]
    
    for bit_size in test_sizes:
        print(f"\nGerando número de {bit_size} bits...")
        number = bbs.random_bits(bit_size)
        
        # Verifica se o número tem o tamanho correto
        binary = bin(number)[2:]  # Remove o prefixo '0b'
        actual_bits = len(binary)
        
        print(f"  Tamanho solicitado: {bit_size} bits")
        print(f"  Tamanho real: {actual_bits} bits")
        print(f"  Número binário: {binary}")
        print(f"  Número decimal: {number}")
        
        # Verificação simples da distribuição de bits
        ones = binary.count('1')
        zeros = binary.count('0')
        print(f"  Distribuição: {ones} bits '1' ({ones/actual_bits:.2%}), {zeros} bits '0' ({zeros/actual_bits:.2%})")
        
        print("-" * 40)


if __name__ == "__main__":
    # Verifica se o usuário quer executar o experimento ou testar a qualidade
    verify_quality = len(sys.argv) > 1 and sys.argv[1] == "--verify"
    
    if verify_quality:
        verify_random_number_quality()
    else:
        print("\nIniciando experimento com Blum Blum Shub...")
        print(f"Gerando {NUM_SAMPLES} amostras para cada tamanho de bits")
        print("Nota: Este algoritmo é significativamente mais lento que outros PRNGs.")
        print("      O experimento pode levar vários minutos ou até horas para completar.")
        run_experiment()
        print("\nDica: Execute com a flag --verify para testar a qualidade dos números gerados.")
\end{verbatim}
