\appendix

\chapter{Apêndices}

Os apêndices deste trabalho contêm as implementações detalhadas dos algoritmos discutidos nos capítulos anteriores. Eles estão organizados da seguinte forma:

\begin{itemize}
    \item \textbf{Apêndice A: Implementações dos Geradores de Números Pseudoaleatórios}
    \begin{itemize}
        \item Seção \ref{apx:xorshift-impl}: Implementação do algoritmo Xorshift e suas variantes (32 bits, 64 bits, 128 bits e 128+ bits)
        \item Seção \ref{apx:bbs-impl}: Implementação do algoritmo Blum Blum Shub
    \end{itemize}
    
    \item \textbf{Apêndice B: Implementações dos Testes de Primalidade}
    \begin{itemize}
        \item Seção \ref{apx:fermat-impl}: Implementação do Teste de Primalidade de Fermat
        \item Seção \ref{apx:miller-rabin-impl}: Implementação do Teste de Primalidade de Miller-Rabin
    \end{itemize}
\end{itemize}

Cada seção dos apêndices contém tanto a implementação do algoritmo em si quanto o código utilizado para os experimentos, permitindo a reprodução de todos os resultados apresentados nos capítulos principais. Os códigos estão documentados com comentários que explicam cada etapa do processo.

As implementações foram desenvolvidas em Python, escolhido pela sua facilidade de manipulação de números grandes e clareza sintática, características essenciais para algoritmos criptográficos e testes de primalidade. Todos os códigos foram projetados visando o equilíbrio entre legibilidade e eficiência, com otimizações específicas para operações críticas como exponenciação modular.

Para executar os experimentos, basta utilizar os arquivos de teste correspondentes a cada implementação, que já estão configurados com os parâmetros necessários para reproduzir os resultados apresentados neste trabalho.

\section{Xorshift}\label{apx:xorshift-impl}

\subsection{Implementação do Algoritmo}

\begin{verbatim}
"""
Implementação do algoritmo Xorshift PRNG
Baseado no trabalho original de George Marsaglia (2003)

Inclui implementações para versões de 32 bits, 64 bits e 128 bits.
"""

class Xorshift32:
    """
    Implementação do Xorshift de 32 bits com período de 2^32-1.
    """
    def __init__(self, seed=123456789):
        """
        Inicializa o gerador com uma semente não-nula.
        
        Args:
            seed (int): Valor de semente. Deve ser não-zero.
        """
        if seed == 0:
            seed = 123456789  # Evita estado zero
        self.state = seed & 0xFFFFFFFF  # Garante valor de 32 bits
    
    def next(self):
        """
        Gera o próximo número pseudoaleatório.
        
        Returns:
            int: Um número pseudoaleatório de 32 bits.
        """
        # Copia o estado atual para a variável x
        x = self.state
        # Aplica o primeiro deslocamento: desloca x 13 bits à esquerda, 
        # faz XOR com o valor original e mantém apenas os 32 bits inferiores
        x ^= (x << 13) & 0xFFFFFFFF
        # Aplica o segundo deslocamento: desloca x 17 bits à direita,
        # faz XOR com o resultado anterior e mantém apenas os 32 bits inferiores
        x ^= (x >> 17) & 0xFFFFFFFF
        # Aplica o terceiro deslocamento: desloca x 5 bits à esquerda,
        # faz XOR com o resultado anterior e mantém apenas os 32 bits inferiores
        x ^= (x << 5) & 0xFFFFFFFF
        # Atualiza o estado interno do gerador com o novo valor
        # garantindo que apenas os 32 bits inferiores sejam mantidos
        self.state = x & 0xFFFFFFFF
        
        # Retorna o novo estado como o número pseudoaleatório gerado
        return self.state
    
    def random(self):
        """
        Retorna um número de ponto flutuante no intervalo [0.0, 1.0).
        
        Returns:
            float: Um número entre 0.0 e 1.0.
        """
        return self.next() / 0x100000000


class Xorshift64:
    """
    Implementação do Xorshift de 64 bits com período de 2^64-1.
    """
    def __init__(self, seed=88172645463325252):
        """
        Inicializa o gerador com uma semente não-nula.
        
        Args:
            seed (int): Valor de semente. Deve ser não-zero.
        """
        if seed == 0:
            seed = 88172645463325252  # Evita estado zero
        self.state = seed & 0xFFFFFFFFFFFFFFFF  # Garante valor de 64 bits
    
    def next(self):
        """
        Gera o próximo número pseudoaleatório.
        
        Returns:
            int: Um número pseudoaleatório de 64 bits.
        """
        # Copia o estado atual para a variável x
        x = self.state
        # Aplica XOR entre x e (x deslocado 13 bits à esquerda), mantendo apenas 64 bits
        x ^= (x << 13) & 0xFFFFFFFFFFFFFFFF
        # Aplica XOR entre x e (x deslocado 7 bits à direita), mantendo apenas 64 bits
        x ^= (x >> 7) & 0xFFFFFFFFFFFFFFFF
        # Aplica XOR entre x e (x deslocado 17 bits à esquerda), mantendo apenas 64 bits
        x ^= (x << 17) & 0xFFFFFFFFFFFFFFFF
        # Atualiza o estado interno com o novo valor, garantindo que tenha apenas 64 bits
        self.state = x & 0xFFFFFFFFFFFFFFFF
        # Retorna o novo estado como o número pseudoaleatório gerado
        return self.state
    
    def random(self):
        """
        Retorna um número de ponto flutuante no intervalo [0.0, 1.0).
        
        Returns:
            float: Um número entre 0.0 e 1.0.
        """
        return self.next() / 0x10000000000000000


class Xorshift128:
    """
    Implementação do Xorshift de 128 bits com período de 2^128-1.
    """
    def __init__(self, seed=None):
        """
        Inicializa o gerador com uma semente ou valores padrão.
        
        Args:
            seed (list, optional): Lista de 4 valores inteiros para inicializar o estado.
        """
        if seed is None:
            # Valores iniciais não-nulos por padrão
            self.state = [123456789, 362436069, 521288629, 88675123]
        else:
            # Verifica se todos os valores são zero
            if all(s == 0 for s in seed):
                self.state = [123456789, 362436069, 521288629, 88675123]
            else:
                self.state = [s & 0xFFFFFFFF for s in seed[:4]]
    
    def next(self):
        """
        Gera o próximo número pseudoaleatório.
        
        Returns:
            int: Um número pseudoaleatório de 32 bits.
        """
        # Armazena o último valor do estado (índice 3) em t
        t = self.state[3]
        # Armazena o primeiro valor do estado (índice 0) em s
        s = self.state[0]
        
        # Desloca os valores do estado: o valor do índice 2 vai para o índice 3
        self.state[3] = self.state[2]
        # Desloca os valores do estado: o valor do índice 1 vai para o índice 2
        self.state[2] = self.state[1]
        # Desloca os valores do estado: o valor do índice 0 (s) vai para o índice 1
        self.state[1] = s
        
        # Aplica a primeira transformação XOR: desloca t 11 bits à esquerda e faz XOR com t
        # A máscara 0xFFFFFFFF garante que o resultado tenha 32 bits
        t ^= (t << 11) & 0xFFFFFFFF
        # Aplica a segunda transformação XOR: desloca t 8 bits à direita e faz XOR com t
        t ^= (t >> 8) & 0xFFFFFFFF
        # Aplica a terceira transformação XOR: faz XOR entre t e o resultado de (s XOR (s deslocado 19 bits à direita))
        t ^= (s ^ (s >> 19)) & 0xFFFFFFFF
        
        # Atualiza o primeiro valor do estado (índice 0) com o novo valor de t
        # A máscara 0xFFFFFFFF garante que o resultado tenha 32 bits
        self.state[0] = t & 0xFFFFFFFF
        # Retorna o valor gerado, garantindo que tenha 32 bits
        return t & 0xFFFFFFFF
    
    def random(self):
        """
        Retorna um número de ponto flutuante no intervalo [0.0, 1.0).
        
        Returns:
            float: Um número entre 0.0 e 1.0.
        """
        return self.next() / 0x100000000


class Xorshift128Plus:
    """
    Implementação do Xorshift128+ - uma variante que utiliza adição para
    melhorar a qualidade estatística (período 2^128-1).
    """
    def __init__(self, seed=None):
        """
        Inicializa o gerador com uma semente ou valores padrão.
        
        Args:
            seed (list, optional): Lista de 2 valores inteiros de 64 bits para inicializar o estado.
        """
        if seed is None:
            # Valores iniciais não-nulos por padrão
            self.state = [1234567890123456789, 9876543210987654321]
        else:
            # Verifica se todos os valores são zero
            if all(s == 0 for s in seed):
                self.state = [1234567890123456789, 9876543210987654321]
            else:
                self.state = [s & 0xFFFFFFFFFFFFFFFF for s in seed[:2]]
    
    def next(self):
        """
        Gera o próximo número pseudoaleatório.
        
        Returns:
            int: Um número pseudoaleatório de 64 bits.
        """
        # Armazena os dois valores do estado atual em variáveis temporárias
        s0 = self.state[0]  # Primeiro valor do estado (64 bits)
        s1 = self.state[1]  # Segundo valor do estado (64 bits)
        
        # Atualiza o primeiro valor do estado com o valor de s1
        # Isso faz parte da rotação do estado para a próxima iteração
        self.state[0] = s1
        
        # Aplica a primeira transformação XOR em s1:
        # Desloca s1 23 bits à esquerda, faz XOR com s1 original
        # A máscara 0xFFFFFFFFFFFFFFFF garante que o resultado tenha 64 bits
        s1 ^= (s1 << 23) & 0xFFFFFFFFFFFFFFFF
        
        # Calcula o novo valor para o segundo elemento do estado:
        # Combina s1 transformado com s0 original e mais duas operações de deslocamento e XOR
        # - s1 ^ s0: XOR entre s1 transformado e s0 original
        # - ^ (s1 >> 18): XOR com s1 deslocado 18 bits à direita
        # - ^ (s0 >> 5): XOR com s0 deslocado 5 bits à direita
        # A máscara final garante que o resultado tenha 64 bits
        self.state[1] = (s1 ^ s0 ^ (s1 >> 18) ^ (s0 >> 5)) & 0xFFFFFFFFFFFFFFFF
        
        # Retorna a soma dos dois valores do estado como o número pseudoaleatório
        # Esta adição é o que diferencia o Xorshift128+ do Xorshift128 padrão,
        # melhorando a qualidade estatística dos números gerados
        # A máscara garante que o resultado tenha 64 bits
        return (self.state[0] + self.state[1]) & 0xFFFFFFFFFFFFFFFF
    
    def random(self):
        """
        Retorna um número de ponto flutuante no intervalo [0.0, 1.0).
        
        Returns:
            float: Um número entre 0.0 e 1.0.
        """
        return self.next() / 0x10000000000000000
\end{verbatim}

\subsection{Implementação do Teste}

\begin{verbatim}
"""
Experimento para avaliação de desempenho de geradores de números pseudoaleatórios.

Este script avalia o tempo necessário para gerar números pseudoaleatórios de diferentes
tamanhos (40-4096 bits) usando diferentes implementações do algoritmo Xorshift.

O experimento gera múltiplos números para cada tamanho e calcula o tempo médio.
"""

import time
import sys
from main import Xorshift32, Xorshift64, Xorshift128, Xorshift128Plus

# Tamanhos de bits a serem testados
BIT_SIZES = [40, 56, 80, 128, 168, 224, 256, 512, 1024, 2048, 4096]

# Número de amostras para calcular o tempo médio
NUM_SAMPLES = 1000


def generate_random_bits(generator, num_bits):
    """
    Gera um número pseudoaleatório com o número especificado de bits.
    
    Args:
        generator: Objeto gerador Xorshift
        num_bits: Número de bits desejado
    
    Returns:
        int: Número pseudoaleatório com o tamanho especificado
    """
    result = 0
    bits_generated = 0
    
    # Determina o número de bits por chamada com base no tipo de gerador
    if isinstance(generator, Xorshift32):
        bits_per_call = 32
    elif isinstance(generator, Xorshift64) or isinstance(generator, Xorshift128Plus):
        bits_per_call = 64
    elif isinstance(generator, Xorshift128):
        bits_per_call = 32
    else:
        raise ValueError("Gerador não reconhecido")
    
    # Gera bits até atingir o tamanho desejado
    while bits_generated < num_bits:
        # Gera um novo número
        new_bits = generator.next()
        
        # Adiciona os novos bits ao resultado (shifts e OR)
        result = (result << bits_per_call) | new_bits
        bits_generated += bits_per_call
    
    # Ajusta para o tamanho exato solicitado (remove bits extras)
    if bits_generated > num_bits:
        extra_bits = bits_generated - num_bits
        result = result >> extra_bits
        
    # Garante que o número tenha exatamente o número de bits solicitado
    # Seta o bit mais significativo para 1
    result |= (1 << (num_bits - 1))
    
    return result


def measure_generation_time(generator_class, bit_size, seed=42):
    """
    Mede o tempo médio para gerar números pseudoaleatórios de determinado tamanho.
    
    Args:
        generator_class: Classe do gerador a ser usada
        bit_size: Tamanho do número em bits
        seed: Semente para inicializar o gerador
    
    Returns:
        float: Tempo médio em milissegundos
    """
    # Inicializa o gerador apropriadamente dependendo da classe
    if generator_class == Xorshift32 or generator_class == Xorshift64:
        generator = generator_class(seed=seed)
    elif generator_class == Xorshift128:
        generator = generator_class(seed=[seed, seed+1, seed+2, seed+3])
    elif generator_class == Xorshift128Plus:
        generator = generator_class(seed=[seed, seed+1])
    
    # Mede o tempo total para gerar NUM_SAMPLES números
    start_time = time.time()
    
    for _ in range(NUM_SAMPLES):
        generate_random_bits(generator, bit_size)
    
    end_time = time.time()
    
    # Calcula o tempo médio em milissegundos
    avg_time_ms = ((end_time - start_time) / NUM_SAMPLES) * 1000
    
    return avg_time_ms


def run_experiment():
    """
    Executa o experimento completo e exibe os resultados em formato de tabela.
    """
    generators = [
        ("Xorshift32", Xorshift32),
        ("Xorshift64", Xorshift64),
        ("Xorshift128", Xorshift128),
        ("Xorshift128Plus", Xorshift128Plus)
    ]
    
    # Imprime o cabeçalho da tabela
    print("\n" + "=" * 80)
    print(f"{'Algoritmo':<20} | {'Tamanho do Número':<20} | {'Tempo para gerar (ms)':<20}")
    print("-" * 80)
    
    # Executa o experimento para cada combinação de gerador e tamanho
    for gen_name, gen_class in generators:
        for bit_size in BIT_SIZES:
            # Mede o tempo de geração
            time_ms = measure_generation_time(gen_class, bit_size)
            
            # Formata e imprime o resultado
            print(f"{gen_name:<20} | {bit_size} bits{' ':<14} | {time_ms:.4f} ms")
        
        # Linha separadora entre geradores
        print("-" * 80)
    
    print("=" * 80)
    print(f"Experimento concluído. Cada medição representa a média de {NUM_SAMPLES} execuções.")


def verify_random_number_quality(show_numbers=False):
    """
    Verifica a qualidade dos números gerados por cada algoritmo.
    Esta função é útil para validar se os números gerados têm o tamanho especificado.
    
    Args:
        show_numbers: Se True, exibe os números gerados para inspeção visual
    """
    print("\nVerificação da qualidade dos números gerados:")
    print("-" * 80)
    
    generators = [
        ("Xorshift32", Xorshift32(seed=42)),
        ("Xorshift64", Xorshift64(seed=42)),
        ("Xorshift128", Xorshift128(seed=[42, 43, 44, 45])),
        ("Xorshift128Plus", Xorshift128Plus(seed=[42, 43]))
    ]
    
    # Seleciona alguns tamanhos para testar
    test_sizes = [40, 128, 256, 1024]
    
    for gen_name, generator in generators:
        print(f"\nGerador: {gen_name}")
        
        for bit_size in test_sizes:
            number = generate_random_bits(generator, bit_size)
            
            # Verifica se o número tem o tamanho correto (em bits)
            binary = bin(number)[2:]  # Remove o prefixo '0b'
            actual_bits = len(binary)
            
            print(f"  Tamanho solicitado: {bit_size} bits")
            print(f"  Tamanho real: {actual_bits} bits")
            
            if show_numbers:
                # Limita a exibição para evitar números muito grandes
                if bit_size <= 128:
                    print(f"  Número binário: {binary}")
                else:
                    print(f"  Número binário: {binary[:64]}...{binary[-64:]} (parte inicial e final)")
            
            print("-" * 40)


if __name__ == "__main__":
    # Verifica se o usuário quer executar o experimento com verificação de qualidade
    verify_quality = len(sys.argv) > 1 and sys.argv[1] == "--verify"
    
    if verify_quality:
        verify_random_number_quality(show_numbers=True)
    else:
        print("\nIniciando experimento de geração de números pseudoaleatórios...")
        print(f"Gerando {NUM_SAMPLES} amostras para cada combinação algoritmo-tamanho")
        run_experiment()
        print("\nDica: Execute com a flag --verify para testar a qualidade dos números gerados.")
\end{verbatim}

\section{Blum Blum Shub}\label{apx:bbs-impl}

\subsection{Implementação do Algoritmo}

\begin{verbatim}
"""
Implementação do algoritmo Blum Blum Shub (BBS) para geração de números pseudoaleatórios.

BBS é um gerador de números pseudoaleatórios criptograficamente seguro proposto em 1986
por Lenore Blum, Manuel Blum e Michael Shub. Sua segurança baseia-se na dificuldade
do problema de fatoração de inteiros.
"""

import random
import math
import time
from typing import Tuple, Optional


def is_prime(n: int, k: int = 40) -> bool:
    """
    Verifica se um número é provavelmente primo usando o teste de Miller-Rabin.
    
    Args:
        n: O número a ser testado
        k: Número de iterações para o teste (maior k = maior confiabilidade)
        
    Returns:
        bool: True se n for provavelmente primo, False caso contrário
    """
    # Casos básicos
    if n <= 1:
        return False
    if n <= 3:
        return True
    if n % 2 == 0:
        return False
    
    # Expressando n-1 como 2^r * d
    r, d = 0, n - 1
    while d % 2 == 0:
        r += 1
        d //= 2
    
    # Teste de Miller-Rabin
    for _ in range(k):
        a = random.randint(2, n - 2)
        x = pow(a, d, n)
        if x == 1 or x == n - 1:
            continue
        for _ in range(r - 1):
            x = pow(x, 2, n)
            if x == n - 1:
                break
        else:
            return False
    return True


def generate_prime(bits: int) -> int:
    """
    Gera um número primo com o número especificado de bits.
    
    Args:
        bits: Número de bits do primo a ser gerado
        
    Returns:
        int: Um número primo com o número especificado de bits
    """
    while True:
        # Gera um número aleatório com o número especificado de bits
        p = random.randint(2**(bits-1), 2**bits - 1)
        
        # Garante que o número é ímpar
        if p % 2 == 0:
            p += 1
        
        # Verifica se é primo usando o teste de Miller-Rabin
        if is_prime(p):
            return p


def generate_blum_integer(bits: int) -> Tuple[int, int, int]:
    """
    Gera um inteiro de Blum, que é o produto de dois primos grandes, ambos
    congruentes a 3 mod 4.
    
    Args:
        bits: Número de bits para cada primo (o inteiro de Blum terá aproximadamente 2*bits)
        
    Returns:
        Tuple[int, int, int]: (n, p, q) onde n = p * q é o inteiro de Blum
    """
    # Gera o primeiro primo p tal que p \equiv 3 (mod 4)
    while True:
        p = generate_prime(bits)
        if p % 4 == 3:
            break
    
    # Gera o segundo primo q tal que q \equiv 3 (mod 4)
    while True:
        q = generate_prime(bits)
        if q % 4 == 3 and p != q:  # Garante que p e q são diferentes
            break
    
    # Retorna o inteiro de Blum n = p * q
    return p * q, p, q


def gcd(a: int, b: int) -> int:
    """
    Calcula o Máximo Divisor Comum (MDC) entre dois números.
    
    Args:
        a: Primeiro número
        b: Segundo número
        
    Returns:
        int: O MDC entre a e b
    """
    while b:
        a, b = b, a % b
    return a


class BlumBlumShub:
    """
    Implementação do gerador de números pseudoaleatórios Blum Blum Shub.
    """
    
    def __init__(self, seed: Optional[int] = None, n: Optional[int] = None, 
                 p: Optional[int] = None, q: Optional[int] = None, bits: int = 512):
        """
        Inicializa o gerador BBS.
        
        Args:
            seed: Semente inicial para o gerador (opcional)
            n: O inteiro de Blum a ser usado (opcional)
            p: Primeiro primo utilizado na geração de n (opcional)
            q: Segundo primo utilizado na geração de n (opcional)
            bits: Número de bits para cada um dos primos (se n não for fornecido)
        """
        # Se n não foi fornecido, gera um novo inteiro de Blum
        if n is None:
            self.n, self.p, self.q = generate_blum_integer(bits)
        else:
            # Usa os valores fornecidos
            self.n = n
            self.p = p
            self.q = q
        
        # Se a semente não foi fornecida, gera uma aleatória
        if seed is None:
            # Gera uma semente aleatória coprima com n
            while True:
                seed = random.randint(2, self.n - 1)
                if gcd(seed, self.n) == 1:
                    break
        else:
            # Verifica se a semente fornecida é válida
            if seed <= 1 or seed >= self.n:
                raise ValueError("A semente deve estar no intervalo [2, n-1]")
            if gcd(seed, self.n) != 1:
                raise ValueError("A semente deve ser coprima com n")
        
        # Inicializa o estado com x_0 = s^2 mod n
        self.state = (seed * seed) % self.n
    
    def next_bit(self) -> int:
        """
        Gera o próximo bit pseudoaleatório.
        
        Returns:
            int: 0 ou 1 (o bit menos significativo do novo estado)
        """
        # Calcula x_{i+1} = x_i² mod n
        self.state = (self.state * self.state) % self.n
        
        # Retorna o bit menos significativo
        return self.state % 2
    
    def next_byte(self) -> int:
        """
        Gera o próximo byte (8 bits) pseudoaleatório.
        
        Returns:
            int: Um número entre 0 e 255
        """
        byte = 0
        # Gera 8 bits para formar um byte
        for i in range(8):
            bit = self.next_bit()
            # Constrói o byte bit a bit (do mais significativo para o menos)
            byte = (byte << 1) | bit
        return byte
    
    def random_bits(self, num_bits: int) -> int:
        """
        Gera um número com a quantidade especificada de bits.
        
        Args:
            num_bits: Número de bits a serem gerados
            
        Returns:
            int: Um número pseudoaleatório com o tamanho especificado
        """
        result = 0
        # Gera os bits um a um
        for _ in range(num_bits):
            bit = self.next_bit()
            # Constrói o número bit a bit
            result = (result << 1) | bit
        
        # Garante que o bit mais significativo seja 1 (para garantir exatamente num_bits bits)
        result |= (1 << (num_bits - 1))
        
        return result
    
    def random(self) -> float:
        """
        Gera um número de ponto flutuante pseudoaleatório no intervalo [0.0, 1.0).
        
        Returns:
            float: Um número entre 0.0 e 1.0
        """
        # Usa 53 bits para garantir precisão suficiente para um float
        return self.random_bits(53) / (1 << 53)
\end{verbatim}

\subsection{Implementação do Teste}

\begin{verbatim}
"""
Experimento para avaliar o desempenho do gerador de números pseudoaleatórios Blum Blum Shub.

Este script mede o tempo necessário para gerar números pseudoaleatórios de diferentes
tamanhos (40-4096 bits) usando o algoritmo Blum Blum Shub (BBS).
"""

import time
import sys
from main import BlumBlumShub

# Tamanhos de bits a serem testados
BIT_SIZES = [40, 56, 80, 128, 168, 224, 256, 512, 1024, 2048, 4096]

# Número de amostras para calcular o tempo médio
NUM_SAMPLES = 10  # Menos amostras para BBS porque é muito mais lento


def measure_generation_time(bit_size):
    """
    Mede o tempo médio para gerar números pseudoaleatórios de determinado tamanho.
    
    Args:
        bit_size: Tamanho do número em bits
    
    Returns:
        float: Tempo médio em milissegundos
    """
    # Inicializa o gerador BBS com um módulo de tamanho adequado
    # O tamanho do módulo deve ser pelo menos duas vezes maior que o do número a ser gerado
    module_bits = max(512, bit_size * 2)
    
    # Medimos também o tempo de inicialização
    init_start_time = time.time()
    bbs = BlumBlumShub(bits=module_bits // 2)
    init_time = (time.time() - init_start_time) * 1000  # em ms
    
    # Mede o tempo total para gerar NUM_SAMPLES números
    total_time = 0
    
    for _ in range(NUM_SAMPLES):
        start_time = time.time()
        bbs.random_bits(bit_size)
        end_time = time.time()
        total_time += (end_time - start_time) * 1000  # em ms
    
    # Calcula o tempo médio em milissegundos
    avg_time_ms = total_time / NUM_SAMPLES
    
    return avg_time_ms, init_time


def run_experiment():
    """
    Executa o experimento completo e exibe os resultados em formato de tabela.
    """
    print("\n" + "=" * 80)
    print("Experimento de geração de números pseudoaleatórios com Blum Blum Shub (BBS)")
    print("=" * 80)
    print(f"{'Tamanho do Número':<20} | {'Tempo de Init (ms)':<20} | {'Tempo de Geração (ms)':<25}")
    print("-" * 80)
    
    # Executa o experimento para cada tamanho
    for bit_size in BIT_SIZES:
        try:
            # Mede o tempo de geração
            gen_time_ms, init_time_ms = measure_generation_time(bit_size)
            
            # Formata e imprime o resultado
            print(f"{bit_size} bits{' ':<14} | {init_time_ms:.4f} ms{' ':<10} | {gen_time_ms:.4f} ms")
            
            # Salva o resultado parcial para não perder progresso
            with open("bbs_results.txt", "a") as f:
                f.write(f"{bit_size},{init_time_ms:.4f},{gen_time_ms:.4f}\n")
                
        except Exception as e:
            print(f"{bit_size} bits{' ':<14} | ERRO: {str(e)}")
    
    print("=" * 80)
    print(f"Experimento concluído. Cada medição representa a média de {NUM_SAMPLES} execuções.")


def verify_random_number_quality():
    """
    Verifica a qualidade dos números gerados pelo algoritmo BBS.
    Esta função é útil para validar se os números gerados têm o tamanho especificado.
    """
    print("\nVerificação da qualidade dos números gerados pelo Blum Blum Shub:")
    print("-" * 80)
    
    # Cria um gerador BBS
    bbs = BlumBlumShub(bits=512)
    
    # Seleciona alguns tamanhos para testar
    test_sizes = [40, 128, 256]
    
    for bit_size in test_sizes:
        print(f"\nGerando número de {bit_size} bits...")
        number = bbs.random_bits(bit_size)
        
        # Verifica se o número tem o tamanho correto
        binary = bin(number)[2:]  # Remove o prefixo '0b'
        actual_bits = len(binary)
        
        print(f"  Tamanho solicitado: {bit_size} bits")
        print(f"  Tamanho real: {actual_bits} bits")
        print(f"  Número binário: {binary}")
        print(f"  Número decimal: {number}")
        
        # Verificação simples da distribuição de bits
        ones = binary.count('1')
        zeros = binary.count('0')
        print(f"  Distribuição: {ones} bits '1' ({ones/actual_bits:.2%}), {zeros} bits '0' ({zeros/actual_bits:.2%})")
        
        print("-" * 40)


if __name__ == "__main__":
    # Verifica se o usuário quer executar o experimento ou testar a qualidade
    verify_quality = len(sys.argv) > 1 and sys.argv[1] == "--verify"
    
    if verify_quality:
        verify_random_number_quality()
    else:
        print("\nIniciando experimento com Blum Blum Shub...")
        print(f"Gerando {NUM_SAMPLES} amostras para cada tamanho de bits")
        print("Nota: Este algoritmo é significativamente mais lento que outros PRNGs.")
        print("      O experimento pode levar vários minutos ou até horas para completar.")
        run_experiment()
        print("\nDica: Execute com a flag --verify para testar a qualidade dos números gerados.")
\end{verbatim}
 
\input{appendix/03-fermat-primality-test.tex}
\section{Teste de Primalidade de Miller-Rabin}\label{apx:miller-rabin-impl}

\subsection{Implementação do Algoritmo}

\begin{verbatim}
#!/usr/bin/env python3
"""
Implementação do Teste de Primalidade de Miller-Rabin

Este módulo implementa o algoritmo de teste de primalidade Miller-Rabin,
um algoritmo probabilístico eficiente para determinar se um número é provavelmente primo.

Autor: João Pedro Schmidt Cordeiro
Data: Abril 2024
"""

import random
import time
from typing import Tuple, List


def decompose(n: int) -> Tuple[int, int]:
    """
    Decompõe n-1 como (2^s) * d, onde d é ímpar.
    
    Args:
        n: O número a ser decomposto (n-1)
    
    Returns:
        Uma tupla (s, d) onde s e d satisfazem n-1 = (2^s) * d e d é ímpar
    """
    n_minus_1 = n - 1
    s = 0
    d = n_minus_1
    
    # Enquanto d for par, divide por 2 e incrementa s
    while d % 2 == 0:
        d //= 2
        s += 1
    
    return s, d


def miller_rabin_round(n: int, a: int) -> bool:
    """
    Executa uma rodada do teste de Miller-Rabin com a base 'a'.
    
    Args:
        n: O número a ser testado para primalidade
        a: A base para o teste (2 <= a <= n-2)
    
    Returns:
        True se n passa no teste para a base a, False caso contrário
    """
    if n == a:
        return True
    
    if n % a == 0:
        return False
    
    # Decompõe n-1 como 2^s * d
    s, d = decompose(n)
    
    # Calcula x = a^d mod n
    x = pow(a, d, n)
    
    # Se x == 1 ou x == n-1, n passa no teste para essa base
    if x == 1 or x == n - 1:
        return True
    
    # Calcula x = x^2 mod n para s-1 iterações
    for _ in range(s - 1):
        x = pow(x, 2, n)
        # Se x == n-1, n passa no teste para essa base
        if x == n - 1:
            return True
        # Se x == 1, encontramos uma raiz quadrada não-trivial de 1, então n é composto
        if x == 1:
            return False
    
    # Se chegamos aqui, n é composto para esta base
    return False


def is_prime_miller_rabin(n: int, k: int = 40) -> bool:
    """
    Determina se n é provavelmente primo usando o teste de Miller-Rabin.
    
    Args:
        n: O número a ser testado
        k: O número de rodadas/bases a serem testadas (padrão: 40)
           Quanto maior o valor de k, menor a probabilidade de erro
    
    Returns:
        True se n é provavelmente primo, False se n é definitivamente composto
    """
    # Casos triviais
    if n <= 1:
        return False
    if n <= 3:
        return True
    if n % 2 == 0:
        return False
    
    # Para números pequenos, podemos usar um conjunto fixo de bases
    # que garantem determinismo até certo limite
    # Recomendação de Menezes, van Oorschot e Vanstone (1996)
    if n < 1_373_653:
        # Para n < 1,373,653, é suficiente testar as bases 2, 3
        bases_deterministic = [2, 3]
    elif n < 9_080_191:
        # Para n < 9,080,191, é suficiente testar as bases 31, 73
        bases_deterministic = [31, 73]
    elif n < 25_326_001:
        # Para n < 25,326,001, é suficiente testar as bases 2, 3, 5
        bases_deterministic = [2, 3, 5]
    elif n < 3_215_031_751:
        # Para n < 3,215,031,751, é suficiente testar as bases 2, 3, 5, 7
        bases_deterministic = [2, 3, 5, 7]
    elif n < 4_759_123_141:
        # Para n < 4,759,123,141, é suficiente testar as bases 2, 7, 61
        bases_deterministic = [2, 7, 61]
    elif n < 1_122_004_669_633:
        # Para n < 1,122,004,669,633, é suficiente testar as bases 2, 13, 23, 1662803
        bases_deterministic = [2, 13, 23, 1662803]
    elif n < 2_152_302_898_747:
        # Para n < 2,152,302,898,747, é suficiente testar as bases 2, 3, 5, 7, 11
        bases_deterministic = [2, 3, 5, 7, 11]
    elif n < 3_474_749_660_383:
        # Para n < 3,474,749,660,383, é suficiente testar as bases 2, 3, 5, 7, 11, 13
        bases_deterministic = [2, 3, 5, 7, 11, 13]
    elif n < 341_550_071_728_321:
        # Para n < 341,550,071,728,321, é suficiente testar as bases 2, 3, 5, 7, 11, 13, 17
        bases_deterministic = [2, 3, 5, 7, 11, 13, 17]
    elif n < 2**64:
        # Para n < 2^64, é suficiente testar as bases 2, 3, 5, 7, 11, 13, 17, 19, 23, 29, 31, 37
        # Conforme provado por Pomerance, Selfridge e Wagstaff e ampliado por Jaeschke
        bases_deterministic = [2, 3, 5, 7, 11, 13, 17, 19, 23, 29, 31, 37]
    else:
        # Para números maiores, usamos bases aleatórias
        for _ in range(k):
            a = random.randint(2, n - 2)
            if not miller_rabin_round(n, a):
                return False
        return True
    
    # Teste com as bases determinísticas selecionadas
    for a in bases_deterministic:
        if a >= n:
            break
        if not miller_rabin_round(n, a):
            return False
    
    # Se todas as k bases passaram no teste, n é provavelmente primo
    return True


def generate_probable_prime(bits: int, k: int = 40) -> int:
    """
    Gera um número provavelmente primo com o número especificado de bits.
    
    Args:
        bits: Número de bits do primo a ser gerado
        k: Número de rodadas no teste de Miller-Rabin (padrão: 40)
    
    Returns:
        Um número provavelmente primo com o número especificado de bits
    """
    while True:
        # Gera um número aleatório com o número especificado de bits
        # Garante que o número é ímpar e tem exatamente o número de bits solicitado
        n = random.getrandbits(bits)
        n |= (1 << (bits - 1)) | 1  # Garante que o bit mais significativo é 1 e que é ímpar
        
        if is_prime_miller_rabin(n, k):
            return n
\end{verbatim}

\subsection{Implementação do Teste}

\begin{verbatim}
#!/usr/bin/env python3
"""
Experimento de Geração de Números Primos usando o Teste de Miller-Rabin

Este script realiza experimentos de geração de números primos de diferentes tamanhos
usando o algoritmo de Miller-Rabin. Ele gera uma tabela com o tempo necessário 
para encontrar um número primo para cada tamanho especificado.

Autor: João Pedro Schmidt Cordeiro
Data: Abril 2024
"""

import sys
import time
import random
from typing import Dict, Tuple, List
from datetime import datetime

# Importa as funções do arquivo principal
from main import is_prime_miller_rabin, generate_probable_prime


def find_prime_with_timeout(bits: int, timeout_seconds: float = 300) -> Tuple[int, float, int, bool]:
    """
    Tenta encontrar um número primo com o número especificado de bits,
    com um limite de tempo para evitar que o programa fique preso.
    
    Args:
        bits: Número de bits do primo a ser gerado
        timeout_seconds: Tempo máximo em segundos para tentar
        
    Returns:
        Uma tupla (primo, tempo_em_ms, tentativas, sucesso)
    """
    print(f"Testando geração de primo de {bits} bits...", end="", flush=True)
    
    start_time = time.time()
    end_time = start_time + timeout_seconds
    attempts = 0
    prime = None
    
    while time.time() < end_time:
        attempts += 1
        if attempts % 10 == 0:
            print(".", end="", flush=True)
        
        # Gera um número ímpar aleatório com o número exato de bits
        n = random.getrandbits(bits)
        n |= (1 << (bits - 1)) | 1  # Garante MSB=1 e número ímpar
        
        if is_prime_miller_rabin(n):
            prime = n
            break
    
    duration = (time.time() - start_time) * 1000  # Converte para milissegundos
    success = prime is not None
    
    if success:
        print(f" Encontrado em {attempts} tentativas e {duration:.2f} ms")
    else:
        print(f" Falha após {attempts} tentativas e {timeout_seconds*1000:.2f} ms")
    
    return prime, duration, attempts, success


def format_number_compact(n: int) -> str:
    """
    Formata um número grande de forma compacta, mostrando apenas o início e o fim.
    
    Args:
        n: O número a ser formatado
        
    Returns:
        Uma string com o número formatado
    """
    if n is None:
        return "N/A"
    
    str_n = str(n)
    if len(str_n) <= 20:
        return str_n
    
    return f"{str_n[:10]}...{str_n[-10:]}"


def run_prime_experiment() -> Dict:
    """
    Executa o experimento de geração de números primos.
    
    Returns:
        Um dicionário com os resultados do experimento
    """
    bit_sizes = [40, 56, 80, 128, 168, 224, 256, 512, 1024, 2048, 4096]
    results = {
        "algorithm": "Miller-Rabin",
        "timestamp": datetime.now().strftime("%Y-%m-%d %H:%M:%S"),
        "results": []
    }
    
    print(f"Experimento iniciado em: {results['timestamp']}")
    print("=" * 80)
    
    # Tempo limite aumenta com o tamanho do número
    for bits in bit_sizes:
        # Define um timeout adequado baseado no tamanho
        if bits <= 128:
            timeout = 60  # 1 minuto para números pequenos
        elif bits <= 512:
            timeout = 300  # 5 minutos para números médios
        elif bits <= 1024:
            timeout = 600  # 10 minutos para números grandes
        else:
            timeout = 1200  # 20 minutos para números muito grandes
        
        prime, duration, attempts, success = find_prime_with_timeout(bits, timeout)
        
        result_entry = {
            "bits": bits,
            "prime": prime,
            "time_ms": duration,
            "attempts": attempts,
            "status": "Sucesso" if success else "Falha"
        }
        
        results["results"].append(result_entry)
    
    print("=" * 80)
    print(f"Experimento concluído em: {datetime.now().strftime('%Y-%m-%d %H:%M:%S')}")
    
    return results


def format_results_as_table(results: Dict) -> str:
    """
    Formata os resultados como uma tabela Markdown.
    
    Args:
        results: Os resultados do experimento
        
    Returns:
        Uma string contendo a tabela em formato Markdown
    """
    table = "| Algoritmo | Tamanho do Número | Tentativas | Tempo para gerar |\n"
    table += "|-----------|-------------------|------------|-------------------|\n"
    
    for result in results["results"]:
        algorithm = results["algorithm"]
        bits = f"{result['bits']} bits"
        
        if result["status"] == "Sucesso":
            attempts_str = str(result["attempts"])
            time_str = f"{result['time_ms']:.2f} ms"
        else:
            attempts_str = str(result["attempts"])
            time_str = "Timeout"
        
        table += f"| {algorithm} | {bits} | {attempts_str} | {time_str} |\n"
    
    return table


def generate_report(results: Dict) -> None:
    """
    Gera um relatório completo do experimento.
    
    Args:
        results: Os resultados do experimento
    """
    # Cria o relatório
    report = "# Experimento de Geração de Números Primos com o Teste de Miller-Rabin\n\n"
    report += f"Data/Hora: {results['timestamp']}\n\n"
    
    # Adiciona a tabela
    report += "## Resultados\n\n"
    report += format_results_as_table(results)
    report += "\n\n"
    
    # Análise e observações
    report += "## Observações\n\n"
    
    # Analisa taxa de sucesso
    successful = sum(1 for r in results["results"] if r["status"] == "Sucesso")
    total = len(results["results"])
    success_rate = (successful / total) * 100 if total > 0 else 0
    
    report += f"- Taxa de sucesso: {successful}/{total} ({success_rate:.1f}%)\n"
    
    # Analisa tempos e tentativas
    if successful > 0:
        times = [r["time_ms"] for r in results["results"] if r["status"] == "Sucesso"]
        attempts = [r["attempts"] for r in results["results"] if r["status"] == "Sucesso"]
        
        avg_time = sum(times) / len(times)
        max_time = max(times)
        min_time = min(times)
        
        avg_attempts = sum(attempts) / len(attempts)
        max_attempts = max(attempts)
        min_attempts = min(attempts)
        
        report += f"- Tempo médio: {avg_time:.2f} ms\n"
        report += f"- Tempo mínimo: {min_time:.2f} ms\n"
        report += f"- Tempo máximo: {max_time:.2f} ms\n"
        report += f"- Tentativas médias: {avg_attempts:.2f}\n"
        report += f"- Tentativas mínimas: {min_attempts}\n"
        report += f"- Tentativas máximas: {max_attempts}\n\n"
        
        # Análise da relação entre tamanho de bits e número de tentativas
        sizes = [r["bits"] for r in results["results"] if r["status"] == "Sucesso"]
        if len(sizes) > 1:
            report += "### Relação entre tamanho e esforço\n\n"
            report += "- À medida que o tamanho em bits aumenta, nota-se:\n"
            
            # Ordenar resultados por tamanho de bits para comparação
            sorted_results = sorted([r for r in results["results"] if r["status"] == "Sucesso"], 
                                    key=lambda x: x["bits"])
            
            if len(sorted_results) >= 2:
                smallest = sorted_results[0]
                largest = sorted_results[-1]
                report += f"  - Para {smallest['bits']} bits: {smallest['attempts']} tentativas, {smallest['time_ms']:.2f} ms\n"
                report += f"  - Para {largest['bits']} bits: {largest['attempts']} tentativas, {largest['time_ms']:.2f} ms\n"
                
                attempt_increase = largest['attempts'] / smallest['attempts'] if smallest['attempts'] > 0 else 0
                time_increase = largest['time_ms'] / smallest['time_ms'] if smallest['time_ms'] > 0 else 0
                
                report += f"  - Aumento de tentativas: {attempt_increase:.2f}x\n"
                report += f"  - Aumento de tempo: {time_increase:.2f}x\n\n"
    
    # Analisa desempenho por tamanho
    report += "### Dificuldades encontradas\n\n"
    
    failed_bits = [r["bits"] for r in results["results"] if r["status"] == "Falha"]
    if failed_bits:
        report += f"- Não foi possível gerar números primos para os seguintes tamanhos: {', '.join(str(b) for b in failed_bits)} bits\n"
    
    # Salva o relatório
    filename = "miller_rabin_report.md"
    with open(filename, "w") as f:
        f.write(report)
    
    print(f"\nRelatório completo gerado com sucesso: {filename}")
\end{verbatim}

